\chapter{Concrete Semantics} \label{chapter:concrete_semantics}

We model the memory as a two-dimensional address space $\ZZ$,
where the first element of the address is the \emph{block identifier} and the second element is the \emph{offset} within the block.
Each call to \malloc\ returns an address with a fresh block identifier, and each variable is written in memory in a unique memory block,
which is distinct from the block of any other variable.

The semantics can be described by the following three functions:
\begin{itemize}
    \item $\rho : \X \rightarrow \ZZ$ assigns an address to each variable,
    \item $\nu : \mathcal{A} \rightarrow \ZZ$ assigns an address to each auxiliary, and
    \item $\mu : \ZZ \rightarrow \ZZ$ assigns a value to each address.
\end{itemize}
We are only interested in analyzing the values of pointers and addresses, therefore we interpret each value stored in the memory and in the auxiliaries as an address.

\section{Assignment}

We consider assignments of the form $s_1\,{:=}\,s_2$ where $s_1$ is either an auxiliary $B \in \mathcal{A}$ or a pointer term $*(z+t)$, while $s_2$ is either the symbol $?$, which represents an unknown value, or is of the form $z_1 + t$ for some term $t$, or \malloc, which assigns an address from a fresh block to $s_1$.
A fresh address block is an address block identifier that does not occur as a mapping in any of the functions $\rho$, $\nu$ or $\mu$.
More precisely we define the property $fresh$ as:

\[ \begin{array}{c}
        fresh(a, (\rho,\nu,\mu)) \equiv
        (\nexists A\in \A . bl(\nu(A)) = a      \\
        \land \nexists x\in \X. bl(\rho(x)) = a \\
        \land \nexists a'\in \ZZ . bl(\mu(a')) = a)
    \end{array}
\]

For a set $H$ of models $(\rho,\nu,\mu)$, we define the concrete semantics of the assignment
$s_1\,{:=}\,s_2$ as the transformation
$\sem{s_1\,{:=}\,s_2}\,H$, where:
\[
    \begin{array}{l}
        \sem{B\,{:=}\,?}\,H =
        \\
        \quad \{(\rho,\nu\oplus\{B\mapsto a\},\mu)\mid(\rho,\nu,\mu)\in H,a\in\ZZ\}                   \\
        \sem{*(z+t)\,{:=}\,?}\,H =                                                                    \\
        \quad \{(\rho,\nu,\mu\oplus\{(z+\sem{t}\,(\rho,\nu,\mu))\mapsto a\})
        \mid(\rho,\nu,\mu)\in H,a\in\ZZ\}                                                             \\
        \sem{B\,{:=}\,z_1+s}\,H =                                                                     \\
        \quad \{(\rho,\nu\oplus\{B\mapsto z_1+\sem{s}\,(\rho,\nu,\mu)\},\mu)\mid(\rho,\nu,\mu)\in H\} \\
        \sem{*(z+t)\,{:=}\,z_1+s}\,H =                                                                \\
        \quad \{(\rho,\nu,\mu\oplus\{(z+\sem{t}\,(\rho,\nu,\mu))\mapsto z_1+\sem{s}\,(\rho,\nu,\mu))\}
        \mid(\rho,\nu,\mu)\in H\}                                                                     \\
        \sem{B\,{:=}\,\malloc}\,H =                                                                   \\
        \quad \{(\rho,\nu\oplus\{B\mapsto (a,0)\},\mu)\mid
        (\rho,\nu,\mu)\in H, a\in \Z,
        fresh(a, (\rho,\nu,\mu))\}\}                                                                  \\
        \sem{*(z+t)\,{:=}\,\malloc}\,H =
        \\
        \quad\{(\rho,\nu,\mu\oplus\{(z+\sem{t}\,(\rho,\nu,\mu))\mapsto (a,0))\}\mid
        (\rho,\nu,\mu)\in H, a\in \Z, fresh(a, (\rho, \nu, \mu))\}.
    \end{array}
\]
Here,
$f\oplus\{m\mapsto z\}$ is the mapping obtained
from $f$ by
setting the value of $f$ for $m$ to $z$.

\section{Interprocedural Semantics}

Let $f$ be a function with parameters $p_1, \ldots, p_n$.
Assume that there is an edge of the CFG is labeled with $f(e_1, \ldots, e_n)$,
and the concrete state at the node where the edge starts is $(\rho, \nu, \mu)$.
The concrete semantics of the
function call is defined by the function $\enter\,(\rho, \nu, \mu)$, which returns the state $(\rho', \nu', \mu')$, defined as follows:
\begin{enumerate}
    \item $\rho'\,x = \rho\,x$ for all global variables $x \in \X$.
    For the local variables of the caller, $\rho'$ does not contain mappings.
          Additionally, for each parameter $p_i$ of the function which is not an auxiliary, $\rho'\,p_i = \sem{e_i}$, where $\sem{e_i}$ is the concrete value of the expression $e_i$ in the state $(\rho, \nu, \mu)$.
    \item $\nu'$ is defined correspondingly: it contains the entries of $\nu'\,A = \nu\,A$ for all global variables $A \in \A$ and $\nu'$ does not contain a mapping for the local variables of the caller.
    For each parameter $p_i$ of the function, $\nu'\,p_i=\sem{e_i}$, if $p_i$ is an auxiliary.
    \item $\mu'\,(a,b)=\mu\,(a,b)$ for each address $(a,b)$.
\end{enumerate}

At the end of the function, let the resulting state of the function be $(\rho', \nu', \mu')$.
The concrete semantics of the edge form the return of the function to the caller is defined by the function $\combine\,(\rho, \nu, \mu)\,(\rho', \nu', \mu')$, which returns the state $(\rho'', \nu'', \mu'')$, defined as follows:
\begin{enumerate}
    \item $\rho''\,x = \rho'\,x$ for all global variables $x \in \X$ and $\rho''\,x = \rho\,x$ for the local variables of the caller.
    The local variables of the function are not present in $\rho''$.
    \item Correspondingly, $\nu''\,A = \nu'\,A$ for all global auxiliary variables $A \in \X$ and $\nu''\,A = \nu\,A$ for the local variables of the caller.
    \item $\mu''\,(a,b)=\mu'\,(a,b)$ for each address $(a,b)$.
\end{enumerate}
