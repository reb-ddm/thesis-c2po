\chapter{2-Pointer Logic as an Abstract Domain}\label{chapter:2pointer}

The abstract domain of the \cpo\ analysis consists of finite conjunctions of propositions from the 2-Pointer Logic.
In the following section, we describe this logic and how to represent the propositions to infer all deriving propositions.
Then, we discuss the extension of the domain to include additional terms and the restriction of the domain to a subset of terms, which will be needed later for the analysis.

\todo{
  Move somewhere:
}

It is possible to define an order on terms, given a linear order $<$ on atoms, by defining:
\begin{itemize}
  \item $a < *(z+t)$ whenever $a$ is an atom and
  \item $*(z_1 + t_1) < *(z_2 + t_2)$ whenever either $t_1 < t_2$ or $t_1 \equiv t_2$ and $z_1 < z_2$.
\end{itemize}

\textcite{2pointer} introduced the 2-Pointer Logic, which consists of equalities and disequalities between the terms described in \cref{chapter:concrete_semantics}.
Here, we extend this logic with an additional type of proposition, the \emph{block disequalities}.
These disequalities express that two terms do not belong to the same address block.
The propositions $p$ are defined by the following grammar:
\[
  p\,{::=}\,t_1=z+t_2 \mid t_1\neq z+t_2\mid bl(t_1) \neq bl(t_2),
\]
where $*$ denotes the dereferencing operator and $bl(t)$ represents the address block of $t$.
We use $*t$ as an abbreviation for $*(0+t)$ and $x$ as an abbreviation for $*(0+\&x)$.

Moreover, we define the function $bl : \ZZ \rightarrow \Z$ that returns the address block identifier of an address, where $bl(a,b) = a$.
The validity of a proposition under a concrete state $(\rho,\nu,\mu)$ is defined as:
\[
  \begin{array}{ll}
    (\rho,\nu,\mu)\models t_1 = z+t_2          & \textbf{iff}
    \\ \multicolumn{2}{c}{\quad\quad\quad\quad\sem{t_1}\,(\rho,\nu,\mu) = z+\sem{t_2}\,(\rho,\nu,\mu)}  \\
    (\rho,\nu,\mu)\models t_1 \neq z+t_2       & \textbf{iff}
    \\ \multicolumn{2}{c}{\quad\quad\quad\quad\sem{t_1}\,(\rho,\nu,\mu) \neq z+\sem{t_2}\,(\rho,\nu,\mu)} \\
    (\rho,\nu,\mu)\models bl(t_1) \neq bl(t_2) & \textbf{iff}
    \\ \multicolumn{2}{c}{\quad\quad\quad\quad bl(\sem{t_1}\,(\rho, \nu, \mu)) \neq bl(\sem{t_2}\,(\rho, \nu, \mu))}
  \end{array}
\]

If $(\rho, \nu, \mu)\models p$ for each proposition $p$ in $\Psi$, then we say that $(\rho, \nu, \mu) \models \Psi$.
\todo[inline]{The domain is not this anymore, it's the kernels. Move this.

The domain of 2-Pointer Logic $\mathcal{P}_2[=]$ consists of all finite conjunctions
of propositions over terms up to semantic equivalence.
The concretization $\gamma(\Psi)$ of $\Psi$ is the set of all $(\rho, \nu, \mu)$ with $(\rho, \nu, \mu) \models \Psi$.
}

%\section{Representation of 2-Pointer Logic}

\section{Quantitative congruence closure}

Given a conjunction $\Psi$ of propositions, the quantitative congruence closure infers all equalities that are logically implied by $\Psi$.
As there are infinitely many such equalities, we consider only the equalities that contain terms from a specific set $\T$ of terms.
The set $\T$ must be subterm-closed and contain at least all the terms that occur in $\Psi$.
The equalities implied by $\Psi$ are described by the quantitative equivalence relation $\equivp$,
which is the smallest equivalence relation $\equivp$ that satisfies these rules:
\begin{enumerate}[label={[E\arabic*]}, ref={[E\arabic*]}]
  \setcounter{enumi}{-1}
\item\label{item:persistence} If $t_1 = z+t_2$ occurs in $\Psi$, then $t_1 \equivp z+t_2$;
\item\label{item:quantitative-reflexivity} $t_1 \equivp 0+t_1$ for all $t\in\T$(\emph{quantitative reflexivity});
\item\label{item:quantitative-symmetry} If $t_1 \equivp z+t_2$, then $t_2 \equivp -z+t_1$ (\emph{quantitative symmetry});
\item\label{item:quantitative-transitivity} If $t_1 \equivp z_1+t_2$ and $t_2 \equivp z_2 + t_3$,
		then $t_1 \equivp (z_1+z_2)+t_3$ (\emph{quantitative transitivity});
\item\label{item:dereferencing} If $t_1 \equivp (z_2 - z_1) + t_2$, then $*(z_1 + t_1) \equivp *(z_2 + t_2)$ holds as well, whenever $*(z_1 + t_1),  *(z_2 + t_2)$ are in $\T$ (\emph{dereferencing}).
\end{enumerate}

\subsection{Quantitative union-find}

We can represent the quantitative equivalence relation $\equivp$ by using a data structure similar to union-find, but extended with integer offsets.
The \emph{quantitative union-find} $(\Pi, \tau, \omega)$ consists of a partition $\Pi$ of $\T$,
a function $\tau : \Pi \rightarrow \T$ that assigns a representative term to each equivalence class $Q \in \Pi$ and a mapping $\omega : \T \rightarrow \Z$,
which assigns to each term its offset from the representative of its equivalence class.
A set $Q \in \Pi$ contains all terms that are equivalent up to an integer offset.

There exist efficient algorithms for the union-find data structure,
in particular the well-known representation proposed by~\textcite{uf-tarjan},
which we will present in this section.
We extend this approach to handle integer offsets as well.

The union-find partition is represented as an acyclic directed weighted graph,
where each node represents a term $t \in \T$ and
each connected component is a tree, which represents an equivalence class $Q \in \Pi$.
Each tree has a root, that is the representative $\tau Q$ of the equivalence class $Q$.
All the edges in the tree are directed towards the root.
Each edge from $t_1$ to $t_2$ is weighted with an integer $z$,
which means that $t_1 \equivp z + t_2$.
$\omega t$ can be computed by following the path from $t$ to the root of the tree and summing up the weights of the edges.
The \find{t} operation returns the pair $(\otau\,t, \omega\,t)$ for a given term $t$,
where $\otau\,t$ denotes the representative $\tau\,Q$ of the equivalence class $Q\in\Pi$ that contains $t$.

The union-find data structure is initialized with a partition $\Pi$ that contains all the terms from $\T$ as singletons.
The corresponding graph contains a node for each $t \in \T$ and no edges.
Then for each equality $t_1 = z + t_2$ in $\Psi$, we perform the \union{t_1}{t_2}{z} operation, which merges two trees by adding an edge from $\tau\,t_1$ to $\tau\,t_2$ with weight $z - \omega\,t_1 + \omega\,t_2$.

\begin{example}
Consider the conjunction $\Psi \equiv B = 1 + A \land D = 2 + \&x \land A = 3 + \&x$.
The union-find algorithm computes the following tree after having performed the three union operations:

\begin{tikzpicture}[->, node distance=2cm, auto]
    \node (x) {$\&x$};
    \node (A) [below left of=x] {$A$};
    \node (D) [below right of=x] {$D$};
    \node (B) [below=1cm of A] {$B$};

    \path (A) edge node {3} (x);
    \path (B) edge node {1} (A);
    \path (D) edge node {2} (x);
\end{tikzpicture}

From this representation, we can for example derive that $B = 4 + \&x$.

Depending on the choice of the representatives, the tree can look different.
In a practical implementation, the representatives are chosen in such a way that the trees have the lowest possible height,
thus making the \find{t} operation more efficient.
Another possible optimization is to set the parent node of a term to the representative of the tree,
each time the \find{t} operation is called.
This way, the tree is flattened, which makes the \find{t} operation
more efficient if we call it again for the same term $t$ or a term in the same equivalence class.
\todo{cite}
\end{example}

We define the action $(+): \Z \rightarrow (\Z \times \T) \rightarrow (\Z \times \T)$, where $z_1 + (z_2,t) = (z_1 + z_2, t)$.
An equality $t_1 = z + t_2$ is satisfied by a partition $(\Pi, \tau, \omega)$, i.e., \find{t_1} = z + \find{t_2}, iff $t_1 \equivp z + t_2$.

\subsection{Quantitative finite automaton}

The union-find data structure is useful for representing equalities between atoms
with an addition of integer offsets.
The union operation needs to be modified in order to take into account also the dereferencing operation.
For example if $\T = {A, B, *A, *B}$ and we perform a union of $A$ and $B$, the
closure rule~\ref{item:dereferencing} tells us, that also the equivalence classes of $*A$ and $*B$ should be merged.
In order to efficiently find the equivalence class corresponding to the dereferenced terms of another equivalence class, we introduce the \emph{quantitative finite automaton} (QFA).

The QFA is defined by a triple $(S, \eta, \delta)$, where $S$ is a finite set of states, where each state represents an equivalence class from $\Pi$, $\eta : (\mathcal{A} \cup \{\&x | x \in \X\} \rightarrow \Z \times S)$ is a partial mapping that provides initial offsets and states for atoms, and $\delta : \Z \rightarrow S \rightarrow \Z \times S$ is the partial transition function.
Intuitively, if $\delta(z_1, s_1) = \delta(z_2, s_2)$, this means that for the representatives
$t_1, t_2$ of the equivalence class of $s_1,s_2$, resepectively, it holds that $*(z_1 + t_1) \equivp z_2 + t_2$.

The QFA is constructed from the partition  $(\Pi, \tau, \omega)$ by defining $M[\Psi,\T] = (S, \eta, \delta)$ where $S = \{s_i | Q_i \in \Pi\}$, $\eta a = (\omega a, s_i)$ if $a \in Q_i$ for some $Q_i \in \Pi$, and $\delta z s_i = (\omega t', s_j)$ if there is a $t' \in Q_j$ with $Q_j \in \Pi$, such that $t' = *(z_1 + t_1)$ with $t_1 \in Q_i$ and $z = z_1 + (\omega t_1)$.

We remark that a transition $\delta z s_i$ could derive not only from a single term $t'$ but also from a second term $*(z_2 + t_2) \in \T$ with $t_2 \in Q_i$ and $z = z_2 + (\omega t_2)$.
Since $t_i \equiv (\omega t_i) + (\tau Q_i)$ for $i = 1,2$, it follows that $t_1 \equiv (z_2 - z_1)+ t_2$. Therefore $*(z_2+t_2)\in Q_j$ with offset $\omega(*(z_2+t_2)) = \omega t'$.
This shows that the result of $\delta z s_i$ is well defined and doesn't depend on which term $t'$ we choose for deriving a transition.
Additionally, $\delta$ is defined only for a finite amount of values, given that each term $*(z + t) \in \T$ defines at most one mapping for $\delta$.

We use $\oT$ to denote the set of \emph{all} terms with variable names from $\X$ and auxiliaries from $\A$.
We can extend the mappings $\eta$,$\delta$ to a partial mapping $M : \oT \rightarrow \Z \times S$ where $M[a] = \eta(a)$ for atoms $a$, and $M[*(z+t_1)] = \delta(z+z_1, s)$ for terms $t_1$ if $M[t_1] = (z_1,s)$.

We define $M[\Psi] = M[\Psi,\T_\Psi]$, where $\T_\Psi$ is the set of terms and subterms occurring in $\Psi$.

We also define the set $\L(M)$ of terms $t \in \oT$ for which $M$ is defined.
For each state $s \in S$ we define the set $\L_M(s)$ of terms $t \in \oT$ for which it holds that $M[t] = (z, s)$ for some $z \in \Z$.

Using the automata, the \emph{closure}$(t_1,t_2,z)$ operation, which is the modified version of the union operation, is defined as follows:

\begin{enumerate}
  \item Case 1: $\tau\,t_1 = \tau\,t_2$. If $\omega\,t_1 = z + \omega\,t_2$ we are done.
  If $\omega\,t_1 \neq z + \omega\,t_2$, then the conjunctioin is unsatisfiable.
  \item Case 2: $\tau\,t_1 \neq \tau\,t_2$. We call \union{t_1}{t_2}{z}.
  Now either $t_1$ or $t_2$ has a new representative, so the transitions of the QFA are updated accordingly.
  Let $t$ be the term that has a new representative,
  and let $P = (\Pi, \tau, \omega)$ be the partition before the union operation and $P' = (\Pi', \tau', \omega')$ be the partition after the union operation.
  For each outgoing transition $\delta(z_1, s_1) = \delta(z_2, s_2)$ where $t \in \L(s_1)$,
  we set $\delta'(z_1 - \tau\,t'', s_1) = \delta'(z_2, s_2)$.
  \todo[inline]{Then use merge for the successors.}
  \todo{change $z_2$ in incoming transitions? how deep should I go in the details?}
\end{enumerate}

\todo[inline]{This is the same as congruence closure (cite) but restricted to a unary uninterpreted function symbol $*$ and extended with integer offsets as in (cite Oliveras Nieuwenhuis).}

\begin{theorem}
  % This is Theorem 3 in the other paper
  Assume that $\Psi$ is a satisfiable conjunction of equalities, let $\T$ be a subterm-closed set of terms which contains all terms occuring in $\Psi$.
  The corresponding quantitative partition $(\Pi, \tau,\omega)$ and a corresponding QFA $M = (S, \eta, \delta)$ are constructed by applying the closure operation to all equalities in $\Psi$.
  Then $\T \subseteq \L(M)$ and for every $t_1, t_2 \in \L(M)$ the following statements are equivalent:
  \begin{enumerate}
    \item $M[t_1] = z + M[t_2]$,
    \item $\Psi$ implies $(t_1 = z + t_2)$.
  \end{enumerate}
\end{theorem}

This theorem was proven in~\cite{2pointer} for the case of a one-dimensional memory model.
The proof can easily be adapted for the two-dimensional memory model.

We remark that the only equalities that can be derived from $\Psi$ are derived from $\Psi_=$. We cannot derive equalities from any of the disequalities.

\section{Block Disequalities}


\section{Disequalities}


\section{Kernel Representation of 2-Pointer Logic Conjunctions}
The abstract values of the \cpo\ analysis consist of a representation
of 2-Pointer Logic conjunctions, as detailed in the previous chapter.
This ensures that whenever propositions are added to or removed from the domain,
the closure of all implied propositions is computed.

The representation consists of a tuple $k = \angl{P,M_P,B,D}$,
where $P = (\T, \tau, \omega)$ is a quantitative partition, as described in \cref{subsection:quantitative-union-find}.
Each state of the QFA $M$ represents an equivalence class of $P$, as outlined in \cref{subsection:qfa}.
$B$ and $D$ are lists of explicit block disequalities and disequalities, respectively, as detailed in \cref{section:block-disequalities,section:disequalities}.

The tuple $k$ is called the \emph{kernel} representation of $\Psi$.
For an unsatisfiable formula $\Psi$, the kernel representation is $\bot$.

It is posible to convert a kernel representation $k = \angl{P,M_P,B,D} \neq \bot$ with $M_P = (S, \otau, \eta, \delta)$ to a formula $\F[k]$, that is given by

\[
  \begin{array}{lll}
    \F[k] & \equiv & \bigwedge_{\eta\,a=(z,s)} (a = z + \otau\,s) \land                        \\
          &        & \bigwedge_{\delta\,z\,s=(z',s')} (*(z + \otau\,s) = z' + \otau\,s') \land \\
          &        & \bigwedge_{\{t, t'\} \in B} (bl(t) \neq bl(t')) \land                     \\
          &        & \bigwedge_{(t, z, t') \in D} (t \neq z + t')
  \end{array}
\]

The trivial propositions of the form $t = 0 + t$ are removed from the formula, as well as the repeated equalities.

If $k$ is the kernel of a conjunction $\Psi$, then $\F[k]$ is equivalent to $\Psi$,
even though it may not be syntactically identical.
They could be syntactically different because of a different choice of representative terms.
\todo[inline]{say something about abstraction and concretization}

\section{Extension to a Superset of Terms}

When new terms appear during the analysis, the partition $P$ and the automaton $M_P$ need to be updated to express properties about the new terms.
It is not sufficient to simply add a fresh equivalence class for the new terms, as they sometimes need to be added to existing equivalence classes.

\begin{example}
    Let $\Psi \equiv A = B$ and $\T = \{A, B, *A\}$.
    The equivalence classes of $P[\Psi]$ are $\{A, B\}$ and $\{*A\}$.
    If we want to add the term $*B$ to $\T$, we need to add it to the same equivalence class as $*A$, as the two terms are equivalent according to the rules defined in \cref{chapter:qcc}.
    \todo{The same example is already somewhere else.}
\end{example}

Let $P = (\T, \tau, \omega)$ be a partition with the corresponding QFA $M_P = (S, \otau, \eta, \delta)$ and let $\T'$ be a superset of $\T$ and closed under subterms.
We define the operation $\ext{k}{\T'}$ that extends the kernel $k = \angl{P, M_P, B, D}$ to the set of terms $\T'$, without altering the semantics of the representation.
Only $P$ and $M_P$ need to be modified, while $B$ and $D$ remain unchanged.

The extension is built inductively based on the structure of the terms in $\T'$.
We first describe how to add a new atom and then consider terms of the form $*(z+t)$, where $t$ is already a part of $\T$.
This is repeated for all terms in $\T'$.

In the case that $\T' = \T \cup \{a\}$ for an atom $a \notin \T$, we simply add a new equivalence class with representative $a$ to $P$, by including a new state $s$ to the states $S$ of $M_P$ and defining $\tau\,a = a$, $\omega\,a=0$, $\eta(a) = (0,s)$, $\otau\,s = a$.

If $\T' = \T \cup \{*(z+t)\}$ for a term $t \in \T$, but $*(z + t) \notin \T$, we add a new equivalence class $\{*(z+t)\}$ to $P$ and then determine if it needs to be merged with an existing equivalence class.
We define $\tau\,(*(z+t)) = *(z+t)$ and $\omega\,(*(z+t))=0$.
Let the resulting partition be $P'$ and let $s \in S$ be the state of the subterm $t$, i.e., for which it holds that $\otau\,s = \tau\,t$.
If $\delta(z + \omega\,t,s)$ is not defined, then we add a new state $s'$ to $S$ and define $\delta(z + \omega\,t, s) = (0, s')$ and $\otau\,s' = *(z+t)$.
Otherwise, let $\delta(z + \omega\,t, s) = (z', s')$.
Then we perform the operation $\closure{*(z + t)}{z'}{\otau\,s'}{P'}$.
In this case, we do not need to add a new state to the automaton,
as the new term is added to the existing equivalence class of $s'$.

\section{Restriction}


For the assignment and the widen operation, it will be useful to be able to forget all information about specific sets of terms.
For example, when we assign a value to a term, we need to remove all terms from the domain that may be modified by this assignment, i.e., all the terms that may alias with this term.
In the following, we will describe how to find the set of terms that may alias with a term and how to restrict the automaton and the disequalities to only keep the propositions that contain terms from a specific set.

Given a partition $P = (\Pi, \tau, \omega)$, a QFA $M = (S, \eta, \delta)$, and a term $t$, we want to compute the partition $\restr{P}{\neg t}$ and the QFA $\restr{M}{\neg t}$ that represents all equalities that follow from $M$ and that are still valid after overwriting the value of $t$.

Let $\T = \L(M)$. We construct the subterm-closed set $\restr{\T}{\neg t}$ that contains all terms $t' \in \T$ such that each subterm of $t'$ definitely does not alias with $t$.
This is exactly the set of terms that are not modified when the value of $t$ changes.
We differentiate between two cases: the case where $t$ is an atom and the case where $t$ is a dereferenced term.

If $t$ is an atom, then it is impossible to reach the address of $t$ by dereferencing.
Therefore, $\restr{\T}{\neg t}$ is the set of all terms that do not contain $t$ as a subterm.

If $t \equiv *(z + t')$ is a dereferenced term, then $\restr{\T}{\neg t}$ contains all the terms of $\T$ where for each subterm of the form $*(z' + v)$ it holds that $t' \nequivp (z' - z) + v$.

Given the set $\restr{\T}{\neg t}$, we can find the restricted automaton $\restr{M}{\neg t} = (S', \eta', \delta')$ by conducting a breadth-first search on the automaton $M$ and computing which paths of the automaton are still reachable by using only the terms in $\restr{\T}{\neg t}$.
In order to correctly update the weights in the automaton and later find representative terms of the equivalence classes that derive from the automaton, we will need to choose a representative term for each state of the automaton, which we will denote as the function $\repr : S' \rightarrow \Z \times \restr{\T}{\neg t}$.

We start with the initial states of $M$:
for each atom $a \in \restr{\T}{\neg t}$ where $\eta(a) = (z, s)$, if we haven't already visited $s$, then we need to choose a new representative for $s$. We simply define $\repr s = (z, a)$ and $\eta'(a) = (0, s)$.
If we have already visited $s$, then let $\repr s = (z', t)$. We define $\eta'(a) = (z - z', s)$.

Then for each visited state $s_1$, we consider all outgoing transitions: for each $\delta(z_1, s_1) = (z_2, s_2)$,
we define $\repr s_2$ if we haven't visited $s_2$ yet.
For this, we find a term $*(z+t)$ such that $M[t] = (z_1', s_1)$ and $z = z_1 - z_1'$ for some $z_1' \in \Z$.
If we manage to find such a term, we define $\repr s_2 = *(z+t)$.
Otherwise, this transition is not valid in $\restr{M}{\neg t}$ anymore, and we don't add $s_2$ to the set of visited states.
Now let $\repr s_1 = (z_1', t_1)$ and $\repr s_2 = (z_2', t_2)$.
We can define the corresponding transition $\delta'(z_1 - z_1', s_1) = (z_2 - z_2', s_2)$.
We add $s_2$ to the set of reachable states, and we continue the search for all outgoing transitions for the remaining visited states.

Essentially, we keep the transitions between the states that are still reachable after removing the terms, and we adjust the weights of the transitions to the new representatives.

The new partition $\restr{P}{\neg t} = (\Pi', \tau', \omega')$ is defined as $\Pi' = \{\L(s) \mid s \in S, s \text{ is reachable in $\restr{M}{\neg t}$}\}$,
$\tau' (\L(s)) = t$ if $\repr s = (z, t)$, and
for each $t \in \L(\restr{M}{\neg t})$ we define $\omega'(t) = z'$ if $\restr{M}{\neg t}[t] = (z', t')$.

This definition of restriction can be applied in the same way by using any subterm-closed set $\T' \subseteq \L(M)$ of terms and restricting the automaton $M$ to the automaton $\restr{M}{\T'}$ that only remembers the equalities that only contain terms from $\T'$.

\begin{lemma}\label{restriction}
    Let $M = (S, \eta, \delta)$ be a QFA and let $\restr{M}{\T'} = (S', \eta', \delta')$ and $\repr : S' \rightarrow \Z \times \T'$ be the automaton and the function that are obtained by restricting $M$ as described above.
    Then for each term $t_1 \in \T'$, it holds that $M[t_1] = (z, s)$ iff $\restr{M}{\T'}[t_1] = (z - \repr s, s)$.
\end{lemma}
\begin{proof}
    This lemma can be proven by induction over the structure of the term $t_1$.
\end{proof}

\begin{proposition}
    Let $M = (S, \eta, \delta)$ be a QFA and let $\restr{M}{\T'} = (S', \eta', \delta')$ be the automaton that is obtained by restricting $M$ as described above.
    Then for each term $t_1, t_2 \in \T'$, it holds that $M[t_1] = z + M[t_2]$ iff $\restr{M}{\T'}[t_1] = z + \restr{M}{\T'}[t_2]$.
\end{proposition}
\begin{proof}
    This proposition follows from Lemma~\ref{restriction}.
\end{proof}

We can also find the set of disequalities and the block disequalities that are still valid when considering only the terms $\T'$.
We keep only the propositions between representatives of the equivalence classes that are still reachable in the automaton $\restr{M}{\T'}$, but we need to update the representatives to the new representatives.
The remaining disequalities are removed.

Given a conjunction $\Psi$, we denote the restriction of the automaton and disequalities derived from $\Psi$ to the set of terms $\T'$ by $\restr{\Psi}{\T'}$.
Correspondingly, $\restr{\Psi}{\neg t}$ is the restriction of $\Psi$ to the terms of $\restr{\T}{\neg t}$, where $\T = \L(M)$ and $M$ is the automaton that derives from $\Psi$.

