\chapter{2-Pointer Logic}


The abstract domain of the \cpo\ analysis consists of porpositions from the 2-Pointer Logic.
In the following section, we describe this logic and how it is represented in the implementation.
The logic is composed of finite conjunctions of propositions.
It was first introduced by Seidl et al.\ in~\cite{2pointer} and was later extended with an additional type of proposition.
The propositions are made up of terms, formed from address constants and auxiliaries using the addition of constant offsets and dereferencing.
The address constants represent addresses of program variables.
The auxiliaries are variables whose address is never taken, therefore
we know of these variables that their address cannot be reached from other terms using address arithmetic or dereferencing.
Only terms whose C type is a pointer type are considered by the analysis.
Let $\X$,$\A$ be disjoint finite sets of variables names and auxiliaries.
The terms $t$ are defined by the following grammar:

\[
    t\,{::=}\,A \mid \&x \mid *(z+t)
\]
where $A\in \A$ is an auxiliary, $x \in \X$ is a variable with address $\&x$ and $z \in \Z$ is an integer.
We call terms $\&x$ or $A$ \emph{atoms}.
Given a linear order $<$ on atoms, it can be extended to an order on terms by defining:
\todo{do i need this?}
\begin{itemize}
    \item $a < *(z+t)$ whenever $a$ is an atom and
    \item $*(z_1 + t_1) < *(z_2 + t_2)$ whenever either $t_1 < t_2$ or $t_1 \equiv t_2$ and $z_1 < z_2$.
\end{itemize}

The terms can form three types of propositions: equalities and disequalities between terms and additionally a proposition which states that two terms do not belong to the same address block.
The propositions $p$ are defined by the following grammar:

\[
    p\,{::=}\,t_1=z+t_2 \mid t_1\neq z+t_2\mid bl(t_1) \neq bl(t_2)
\]

Where $*$ denotes the dereferencing operator and $bl(t)$ represents the address block of $t$.
We use $*t$ as an abbreviation for $*(0+t)$ and $x$ as an abbreviation for $*(0+\&x)$.
For a conjunction $\Psi$ of propositions, we denote the conjunctions of all the equalities in $\Psi$ by $\Psi_{=}$, the conjunction of disequalities between terms by $\Psi_{\neq}$ and the set of block disequalities of the type $bl(t_1) \neq bl(t_2)$ by $\Psi_{bl}$.

\section{Semantics}

We model the memory as a two-dimensional address space $\ZZ$,
where the first element of the address is the \emph{block identifier} and the second element is the \emph{offset} within the block.
Each call to \malloc\ returns an address with a fresh block identifier, and each variable is written in memory in a unique memory block,
which is distinct from the block of any other variable.

The semantics can be described by the following three functions:
\begin{itemize}
\item $\rho : \X \rightarrow \ZZ$ assigns an address to each variable,
\item $\nu : \mathcal{A} \rightarrow \ZZ$ assigns an address to each auxiliary, and
\item $\mu : \ZZ \rightarrow \ZZ$ assigns a value to each address.
\end{itemize}
We are only interested in analyzing the values of pointers and addresses, therefore we interpret each value stored in the memory and in the auxiliaries as an address.

We also define the operator $(+) : \Z \rightarrow \ZZ \rightarrow \ZZ$ on addresses, where $z + (a,b) = (a, z+b)$.
Here, $z$ is added to the address offset, thus modeling the fact that it's not possible to leave a memory block using address arithmetic.
Moreover, we define the function $bl : \ZZ \rightarrow \Z$ that returns the address block identifier of an address, where $bl(a,b) = a$.

A term $t$ is interpreted as the value $\sem{t}(\rho, \nu, \mu)$ defined by:
\[
  \begin{array}{lll}
    \sem{\& x}\,(\rho,\nu,\mu)   & = & \rho\,x                          \\
    \sem{A}\,(\rho,\nu,\mu)      & = & \nu\,A                           \\
    \sem{*(z+t)}\,(\rho,\nu,\mu) & = & \mu\,(z+\sem{t}\,(\rho,\nu,\mu)) \\
  \end{array}
\]

The validity is defined as:
\[
  \begin{array}{ll}
    (\rho,\nu,\mu)\models t_1 = z+t_2          & \textbf{iff}
    \\ \multicolumn{2}{c}{\quad\quad\quad\quad\sem{t_1}\,(\rho,\nu,\mu) = z+\sem{t_2}\,(\rho,\nu,\mu)}	\\
    (\rho,\nu,\mu)\models t_1 \neq z+t_2       & \textbf{iff}
    \\ \multicolumn{2}{c}{\quad\quad\quad\quad\sem{t_1}\,(\rho,\nu,\mu) \neq z+\sem{t_2}\,(\rho,\nu,\mu)}	\\
    (\rho,\nu,\mu)\models bl(t_1) \neq bl(t_2) & \textbf{iff}
    \\ \multicolumn{2}{c}{\quad\quad\quad\quad bl(\sem{t_1}\,(\rho, \nu, \mu)) \neq bl(\sem{t_2}\,(\rho, \nu, \mu))}
  \end{array}
\]

If $(\rho, \nu, \mu)\models p$ for each proposition $p$ in $\Psi$, then we say that $(\rho, \nu, \mu) \models \Psi$. The domain of 2-Pointer Logic $\mathcal{P}_2[=]$ consists of all finite conjunctions
of propositions over terms up to semantic equivalence.
The concretization $\gamma(\Psi)$ of $\Psi$ is the set of all $(\rho, \nu, \mu)$ with $(\rho, \nu, \mu) \models \Psi$.




\section{Representation of C Expressions as 2-Pointer Logic Terms}

Since the \cpo\ analysis only considers C expressions with pointer types, they have a uniform size.
For instance, on an x86-64 architecture, pointers occupy 64 bits in the memory.

However, the offsets added to pointers vary depending on the pointer's type.
For example, if we have a pointer $p$ of type $int32_t\,*p$, the expression $p + z$ represents an offset of $32 \cdot z$ bits.
If the same pointer $p$ is cast to a pointer of type $int64_t\,*p$, the offset $z$ is of $64 \cdot z$ bits.
It is essential to consider this to know the relative positioning of
different expressions in the memory and to discover possible overlapping data.
Therefore, in our implementation, the offsets in the abstract terms representing C expressions are expressed in bits rather than the value present in the code.
For instance, if $p$ is a pointer to a 32-bit integer and $q$ is a pointer to a 64-bit integer, then the analysis represents the C expression $*(p + 1)$ as $*(p + 32)$, and $*(q + 1)$ as $*(q + 64)$.

In some cases, it is impossible to determine the offset of a pointer expression, e.g., if the offset is an unknown variable instead of a constant.
Here, the \goblint\ constant propagation analysis is used to infer the offset value.
If the value remains indeterminable, the expression is treated as unknown.

Furthermore, arrays and structs are handled similarly to pointers in C, allowing us to express properties not only about pointers but also about arrays and structs.
For example, in C, the expression $a[1]$ is equivalent to the expression $*(a + 1)$.
Hence, the analysis treats arrays as pointers to their first element.
A two-dimensional array in C with dimensions $n$ and $m$ can be viewed as a one-dimensional array with $n \cdot m$ elements.
The expression $a[i][j]$ is treated as $*(a + (i \cdot m + j) \cdot s)$, where $s$ is the size in bits of the array elements.
Similarly, arrays with more dimensions are treated as pointers to a one-dimensional array.

For a struct $s$, the analysis treats $\&s$ as a pointer to the first field of the struct.
For a field $f$ of the struct $s$ with an offset of $z$ bits,
the expression \textsf{s.f} is treated as $*(\&s + z)$.
In C, the offset of a struct element may not always be a multiple of 8 bits.
Thus, offsets are represented in bits rather than bytes.
The analyzer \goblint\ can determine the bit offsets of fields within a struct in most cases.
\cpo\ leverages \goblint\ to find these offsets, and in the rare instances where \goblint\ cannot determine the exact offset, the expression is treated as unknown.


\begin{example}
    Consider this C struct:
    \begin{minted}{C}
        struct S {
            int32_t first;
            int32_t arr[5];
        };
    \end{minted}
    Let $x$ be a variable of type \textsf{struct S}.
    The expression \textsf{x.arr[4]} is treated as $*(\&x + 160)$.
    Inside the struct, the offset of $arr$ is $32$ and the offset of $arr[4]$ is $32\cdot 4 = 128$.
    In total, the offset is $32 + 128 = 160$.
\end{example}
%struct S x[3][2];
%struct S {
% int first;
% int arr[5];
%   };
%x[2][1].arr[4]
%*(1120+x[Ptr64])[Int32]

\section{Quantitative Congruence Closure}\label{chapter:qcc}

Given a conjunction $\Psi$ of propositions, we want to find all equalities logically implied by $\Psi$.
As infinitely many such equalities exist, we only consider the ones containing terms exclusively from a specific set $\T$.
The set $\T$ must be subterm-closed and contain at least all the terms in $\Psi$.
The equalities implied by $\Psi$ containing terms from $\T$ are described by the quantitative equivalence relation $\equivp$, which is the smallest equivalence relation $\equivp$ satisfying these rules:
\begin{enumerate}[label={[E\arabic*]}, ref={[E\arabic*]}]
  \setcounter{enumi}{-1}
  \item\label{item:persistence} If $t_1 = z+t_2$ occurs in $\Psi$, then $t_1 \equivp z+t_2$;
  \item\label{item:quantitative-reflexivity} $t \equivp 0+t$ for all $t\in\T$ (\emph{quantitative reflexivity});
  \item\label{item:quantitative-symmetry} If $t_1 \equivp z+t_2$, then $t_2 \equivp -z+t_1$ (\emph{quantitative symmetry});
  \item\label{item:quantitative-transitivity} If $t_1 \equivp z_1+t_2$ and $t_2 \equivp z_2 + t_3$,
  then $t_1 \equivp (z_1+z_2)+t_3$ (\emph{quantitative transitivity});
  \item\label{item:dereferencing} If $t_1 \equivp (z_2 - z_1) + t_2$, then $*(z_1 + t_1) \equivp *(z_2 + t_2)$ holds as well, whenever $*(z_1 + t_1),  *(z_2 + t_2)$ are in $\T$ (\emph{dereferencing}).
\end{enumerate}

\subsection{Quantitative Partition}\label{subsection:quantitative-union-find}

We can represent the quantitative equivalence relation $\equivp$ as a partition.
The \emph{quantitative partition} $P = (\T, \tau, \omega)$ consists of a set of terms $\T$,
a function $\tau : \T \rightarrow \T$ that assigns a \emph{representative} to each term $t \in \T$ and a mapping $\omega : \T \rightarrow \Z$,
which assigns to each term its offset from the representative of its equivalence class.
These two functions model equalities of the form $t \equivp \omega\,t + \tau\,t$ for each $t \in \T$.
All terms with the same representative belong to the same equivalence class, i.e., they are equivalent up to an integer offset.

We denote as $P[\Psi]$ the quantitative partition that model the conjunction $\Psi$.
The set of representative terms $rep(P[\Psi])$ is the set of terms $t \in \T$ that occur on the right-hand side of the function $\tau$, i.e., there exists a $t'\in \T$ such that $\tau\,t' = t$.
For each $t\in rep(P[\Psi])$, it holds that $\tau\,t=t$ and $\omega\,t = 0$.

The quantitative partition $P = (\T, \tau, \omega)$ is useful to define a normal form for propositions.
Given an equality $p \equiv t_1 = z + t_2$ and a partition $P = (\Pi, \tau, \omega)$,
the normalized equality $norm_P(p)$ is an equivalent equality that only contains representatives of $P$,
i.e., $\tau\,t_1 = z - \omega\,t_1 + \omega\,t_2 + \tau\,t_2$.
Equivalently, the normalized disequality $norm_P(p)$ for $p \equiv t_1 \neq z + t_2$ is the disequality $\tau\,t_1 = z - \omega\,t_1 + \omega\,t_2 + \tau\,t_2$.
The normalized block disequality $norm_P(p)$ for $p \equiv bl(t_1) \neq bl(t_2)$ is the disequality $bl(\tau\,t_1) \neq bl(\tau\,t_2$).

\subsubsection{Implementation}

The quantitative partition is represented using a data structure similar to union-find but extended with integer offsets.
The union-find data structure is represented as a forest of trees, each representing an equivalence class.
Each node in the tree corresponds to a term $t \in \T$, and the roots are representative elements $\tau\,t$.\cite{uf-tarjan}
All the edges in the tree are directed towards the root.
In order to represent the integer offset, each edge is labeled with an integer weight.

The forest is represented as a map $parent : \T \rightarrow \Z \times \T$ that maps each term $t$ to an integer offset and its parent node in the tree.
The nodes that have themselves as a parent are the representative terms.
The mapping $parent(t_1) = (z,t_2)$ defines an edge from $t_1$ to $t_2$ with weight $z$ in the graph, and it implies the equality $t_1 \equivp z + t_2$.

In order to compute the representative of a term in the tree, there is an operation $\find{t}$.
It returns the pair $(\omega\,t,\tau\,t)$.
The offset $\omega t$ can be computed by following the path from $t$ to the tree's root and summing up the weights of the encountered edges.
The representative $\tau\,t$ is the root of the tree containing $t$.

The union-find tree is built incrementally by adding consecutively equalities.
Initially, each term $t\in\T$ is its own parent, and equalities $t_1 = z + t_2$ between terms can be added by performing the $\union{t_1}{z}{t_2}$ operation.
This operation merges two trees by adding an edge from $\tau\,t_1$ to $\tau\,t_2$ with weight $z - \omega\,t_1 + \omega\,t_2$,
thus merging the two equivalence classes of $t_1$ and $t_2$ with the correct offset.
If the two terms are already equivalent with a different offset, i.e., $\tau\,t_1 = \tau\,t_2$, but $z - \omega\,t_1 + \omega\,t_2 \neq 0$, then the conjunction is unsatisfiable.

\begin{example}
  Consider the conjunction $\Psi \equiv (B = 1 + A) \land (D = 2 + \&x) \land (A = 3 + \&x)$.
  The tree computed by the union-find algorithm is illustrated in \cref{fig:uf-tree}.
  \begin{figure}
    \begin{tikzpicture}[->, node distance=2cm, auto]
      \node (x) {$\&x$};
      \node (A) [below left of=x] {$A$};
      \node (D) [below right of=x] {$D$};
      \node (B) [below=1cm of A] {$B$};

      \path (A) edge node {3} (x);
      \path (B) edge node {1} (A);
      \path (D) edge node {2} (x);
    \end{tikzpicture}
    \caption{Union-find tree corresponding to the conjunction  $(B = 1 + A) \land (D = 2 + \&x) \land (A = 3 + \&x)$}\label{fig:uf-tree}
  \end{figure}
  From this representation, we can, for example, derive that $B = 4 + \&x$.
  The tree can look different depending on the choice of the representatives.
\end{example}

In a practical implementation, the representatives are chosen such that the trees have the lowest possible height, thus making the $\find{t}$ operation more efficient.
This optimization is called \emph{union by rank} and consists of adding the edge from the smallest to the largest equivalence class during the $\union{t_1}{z}{t_2}$ operation.
Another possible optimization is, during the $\find{t}$ operation, to set the parent node of a term to the representative of the tree each time a node is traversed.
This way, the tree is flattened, which makes the $\find{t}$ operation
more efficient if we call it again for the same term $t$ or a term in the same equivalence class.
This is the so-called \emph{path compression}.~\cite{uf-tarjan}

\subsection{Quantitative Finite Automaton}\label{subsection:qfa}

A partition $P = (\T, \tau, \omega)$ groups all terms that are equal up to an integer offset.
The partition takes into account the reflexivity, symmetry, and transitivity rules.
Further, it is necessary to ensure that the equalities deriving from \labelcref{item:dereferencing} are always considered.
Therefore, for each term $t$, we need to keep track of the equivalence classes that contain terms of the form $*(z+t)$ and merge two equivalence classes if they contain terms that are equal according to the rule \labelcref{item:dereferencing}.
For example, if $\T = \{A, B, *A, *B\}$ and we perform a union of $A$ and $B$, the closure rule~\labelcref{item:dereferencing} tells us that also the equivalence classes of $*A$ and $*B$ should be merged.
Thus, we introduce an alternative representation of the partition $P$ as a \emph{quantitative finite automaton} (QFA) $M_P$.

The QFA is defined by a triple $M_P = (S, \otau, \eta, \delta)$. $S$ is a finite set of states, each representing an equivalence class of the partition.
In order to know which state represents which equivalence class of the partition, the mapping $\otau : S \rightarrow \T$ assigns a representative term to each state.
If $\otau\,s = t$ for a state $s \in S$, then $s$ represents the equivalence class of all terms whose representative in $P$ is $t$.
The partial mapping $\eta : (\mathcal{A} \cup \{\&x | x \in \X\} \rightarrow \Z \times S)$ provides initial offsets and states for atoms, meaning that if $\eta\,a = (z,s)$, then $a$ is in the equivalence class represented by $s$.
$\delta : \Z \rightarrow S \rightarrow \Z \times S$ is the partial transition function.
If $\delta\,z_1\,s_1 = (z_2, s_2)$, this means that it holds that $*(z_1 + \otau\,s_1) \equivp z_2 + \otau\,s_2$.

The QFA $M_P = (S, \otau, \eta, \delta)$ is constructed from a partition  $P = (\T, \tau, \omega)$ as follows:
\begin{itemize}
  \item $S$ contains a state $s$ for each representative term $t$ of $P$ and we set $\otau\,s = t$.
  \item $\eta\,a = (\omega\,a, s)$ if $s \in S$ and $\otau\,s = \tau\,a$
  \item $\delta\,z\,s_1 = (\omega\,t', s_2)$ if there is a $\tau\,t'=\otau\,s_2$, such that $t' \equiv *(z_1 + t_1)$ with $\tau \,t_1= \otau\,s_1$ and $z = z_1 + \omega\,t_1$.
\end{itemize}

We remark that following this definition, a transition $\delta\,z\,s_1$ could derive not only from a single term $t'$ but also from a second term $*(z_2 + t_2) \in \T$ with $\tau\,t_2=\otau\,s_1$ and $z = z_2 + \omega\,t_2$.
However, we show that the resulting mapping $\delta\,z\,s_1$  would have the same value in this case.
Since $t_i \equivp (\omega t_i) + (\tau Q_i)$ for $i = 1,2$, it follows that $t_1 \equivp (z_2 - z_1)+ t_2$.
Therefore $\tau\,*(z_2+t_2)=\otau\,s_2$ with offset $\omega(*(z_2+t_2)) = \omega\,t'$.
This shows that the result of $\delta\,z\,s_1$ is well defined and does not depend on which term $t'$ we choose for deriving a transition.
Moreover, we note that $\delta$ is defined only for a finite amount of values, given that each term $*(z + t) \in \T$ defines at most one mapping for $\delta$.

We use $\oT$ to denote the set of \emph{all} terms with variable names from $\X$ and auxiliaries from $\A$.
We can extend the mappings $\eta$ and $\delta$ to a partial mapping $M : \oT \rightarrow \Z \times S$ where $M_P[a] = \eta(a)$ for atoms $a$, and $M_P[*(z+t_1)] = \delta(z+z_1, s)$ for terms $t_1$ if $M_P[t_1] = (z_1,s)$.

Furthermore, we define the set $\L(M_P)$ of terms $t \in \oT$ for which $M$ is defined.
For each state $s \in S$, the set $\L_{M_P}(s)$ is the set of terms $t \in \oT$ for which it holds that $M_P[t] = (z, s)$ for some $z \in \Z$.

Using the automaton, we define the $\closure{t_1}{z}{t_2}{P}$ operation that modifies the partition $P = (\T, \tau, \omega)$ and the automaton $M_P = (S, \otau, \eta, \delta)$ to include the equality $t_1 = z + t_2$ and then computes the closure of the rule \labelcref{item:dereferencing}.
The resulting partition $P' = (\T, \tau', \omega')$ and automaton $M_{P'} = (S', \otau', \eta', \delta')$  are computed as follows:
\todo{reread}
\begin{itemize}
  \item Case 1: $\tau\,t_1 = \tau\,t_2$. If $\omega\,t_1 = z + \omega\,t_2$, then $P' = P$ and $M_P = M_{P'}$.
        If $\omega\,t_1 \neq z + \omega\,t_2$, then the conjunction is unsatisfiable.
  \item Case 2: $\tau\,t_1 \neq \tau\,t_2$. We call $\union{t_1}{z}{t_2}$.
        Afterwards, either $t_1$ or $t_2$ has a new representative, so the transitions of the QFA are updated accordingly.
        W.l.o.g., let $\tau\,t_1$ be the new representative of $t_1$ and $t_2$:
        \begin{itemize}
          \item Let $\otau\,s_1 = \tau\,t_1$ and $\otau\,s_2 = \tau\,t_2$.
                After the union, the state $s_2$ will not be in $S'$, as the equivalence classes of $s_1$ and $s_2$ are merged.
                Therefore, we need to move all outgoing transitions of $s_2$ to be transitions of $s_2$.
                For each transition $\delta(z_2, s_2) = (z_3, s_3)$, we set $\delta'(z_1, s_1) = (z_3, s_3)$, where $z_1 = z_2 - \omega\,t_1 + \omega\,t_2 - z$, if $\delta(z_1,s_1)$ is not already defined.
          \item If $\delta(z_1,s_1) = (z_4, s_4)$ is defined, then the equivalence classes of $s_3$ and $s_4$ need to be merged later.
                We add $(\otau\,s_3 = \omega\,(\otau\,s_4) - \omega\,(\otau\,s_3) + \otau\,s_4)$ to a queue $Q$ of pending equalities.
          \item For each incoming transition $\delta(z_5, s_5) = \delta(z_2', s_2)$, we set $\delta'(z_5, s_5) = (z_2' - z_1, s_2)$.
                The remaining transitions of $\delta$ are added to $\delta'$ without modifications.
          \item Then we call $\closure{t_1}{z}{t_2}{P'}$ for each equality $t_1 = z + t_2 \in Q$.
        \end{itemize}
\end{itemize}

This algorithm is similar to congruence closure \cite{cc-tarjan,cc-shostak}, but it is restricted to a unary uninterpreted function symbol $*$ and extended with integer offsets.

\begin{figure}\begin{subfigure}{0.5\textwidth}
\begin{centering}
  \begin{tikzpicture}[shorten >=1pt, node distance=3cm, on grid, auto, state/.style={rectangle, rounded corners, draw, inner sep=5pt, align=center}]

    \node[state] (A) {$A$};

    \node[state, below=1cm of A] (B) {$B$};

    \node[state, right of=A] (As) {$*A$};

    \node[state, right of=B] (Bs) {$*(1 + B)$};

    \path[->]
    (A) edge[] node {$0,0$} (As)
    (B) edge[] node {$1,0$} (Bs);

    \node[above left=0.5cm and 0.3cm of A] {$s_0$};
    \node[above left=0.5cm and 0.3cm of B] {$s_1$};
    \node[above left=0.5cm and 0.3cm of As] {$s_2$};
    \node[above left=0.5cm and 0.3cm of Bs] {$s_3$};

  \end{tikzpicture}
\end{centering}
  \caption{A QFA representing the empty conjunction.}\label{fig:initial-example}
\end{subfigure}
\begin{subfigure}{0.5\textwidth}
  \begin{centering}

  \begin{tikzpicture}[shorten >=1pt, node distance=3cm, on grid, auto, state/.style={rectangle, rounded corners, draw, inner sep=5pt, align=center}]

  \node[state] (AB) {$\begin{array}{c}
    \boldsymbol{A} = \\
    1 + B
  \end{array}$};

  \node[state, right of=AB] (ABs) {$\begin{array}{c}
    \boldsymbol{*A} = \\
    *(1 + B)
  \end{array}$};

  \path[->]
  (AB) edge[] node {$0,0$} (ABs);

  \node[above left=0.9cm and 0.3cm of AB] {$s_0$};
  \node[above left=0.9cm and 0.3cm of ABs] {$s_2$};
\end{tikzpicture}
\end{centering}

  \caption{The same QFA after the operation $\closure{A}{1}{B}{P}$.}\label{fig:closure-example}
\end{subfigure}

\caption{An example of the $\closure{A}{1}{B}{P}$ operation on a QFA over the set of terms
$\T = \{A, B, *A, *(1 + B)\}$.
The representative of each equivalence class is represented in bold as the mentioned first term.}


\end{figure}

\begin{example}
  A QFA can be visualized by labeling the states with the terms of $\T$ that are in the corresponding equivalence class.
  For example, consider the set of terms $\T = \{A, B, *A, *(1 + B)\}$.
  The edges are labeled with the offsets: for an edge $\delta(s_1, z_1) = (s_2, z_2)$, the edge between $s_1$ and $s_2$ is labeled with $z_1, z_2$.

  The initial automaton would be the one in \cref{fig:initial-example}.
  If the operation $\closure{A}{1}{B}{P}$ is called,
  first the $\union{A}{1}{B}$ is performed and the new representative of $B$ becomes $A$, with offset $1$.
  It would be equivalent to choose $B$ as the new representative, but we choose $A$ here as an example.
  Then, we want to move the outgoing transitions from $s_1$ to $s_0$, as the state $s_1$ is removed.
  However, there already exists a transition $\delta(s_0,0) = (s_2, 0)$.
  Therefore, we add the equality $*(1 + B) = *A$ to a queue of pending equalities.
  Then we recursively call $\closure{*(1 + B)}{0}{*A}{P'}$ and thus perform the union of $*(1 + B)$ and $*A$.
  As before, we choose $*A$ as the new representative, but $*(1+B)$ would also be a possible choice.
  The resulting automaton is visualized in \cref{fig:closure-example}.
\end{example}

\todo[inline]{Either prove or remove this theorem.}
\begin{theorem}
  % This is Theorem 3 in the other paper
  Assume that $\Psi$ is a satisfiable conjunction of equalities, let $\T$ be a subterm-closed set of terms which contains all terms occuring in $\Psi$.
  The corresponding quantitative partition $P = (\T, \tau,\omega)$ and a corresponding QFA $M_P = (S, \otau, \eta, \delta)$ are constructed by applying the closure operation with all equalities in $\Psi$.
  Then $\T \subseteq \L(M_P)$ and for every $t_1, t_2 \in \L(M_P)$ the following statements are equivalent:
  \begin{enumerate}
    \item $M_P[t_1] = z + M_P[t_2]$,
    \item $\tau\,t_1 = \tau\,t_2$ and $\omega\,t_1 = z + \omega\,t_2$,
    \item $\Psi$ implies $(t_1 = z + t_2)$.
  \end{enumerate}
\end{theorem}

This theorem was proven in~\cite{2pointer} for the case of a one-dimensional memory model.
The proof can easily be adapted for the two-dimensional memory model.

We remark that the only equalities that can be derived from $\Psi$ are derived from the equalities in $\Psi$. We cannot derive equalities from any of the disequalities.

\section{Block Disequalities}


\section{Disequalities}\label{disequalities}

In order to implement a more precise abstract transformer for the assignment, it is important to know which terms definitely
point to different addresses. Let $\Psi$ be a conjunction of 2-pointer logic terms.
We define the binary relation $\nequivp$ which represents all disequalities that can be derived from $\Psi$.

\begin{enumerate}
  \item[(D0)] If $t_1 \neq z + t_2 \in \Psi$, then $t_1 \nequivp z + t_2$.
  \item[(D1)] If $t_1 \nequivp z + t_2$, then:
    \begin{enumerate}
      \item $t_2 \nequivp - z + t_1$.
      \item $t'_1\nequivp (z_1 + z) + t_2$ if $t_1' = z_1 + t_1$ is implied by $\Psi$.
      \item If $t_1 = z + t_2$ is implied by $\Psi$, then $\Psi$ is Unsat.
    \end{enumerate}
  \item[(D2)] If $*(z_1 + t_1) \nequivp *(z_2 + t_2)$ then also $t_1 \nequivp (z_2 - z_1) + t_2$.
  \item[(D3)] If $t_1 = z' + t_2$ is implied by $\Psi$ then $t_1 \nequivp z + t_2$ for all $z \neq z'$.
  \item[(D4)] If $bl(t_1) \neq bl(t_2) \in \Psi$, then:
    \begin{enumerate}
      \item $t_1 \nequivp z + t_2$ for all $z \in \Z$.
      \item If $t_1 = z + t_2$ for any $z \in \Z$, then $\Psi$ is Unsat.
    \end{enumerate}
\end{enumerate}

The set of disequalities following from (D3) and (D4) can be infinite, but we don't need to explicitly store them, as they implicitly derive from the congruence closure of $\Psi_=$ or respectively from the propositions in $\Psi_{bl}$.
We construct the set of all disequalities and block disequalities deriving from (D0), (D1), (D2) and (D4),but only those that contain representative terms of components $Q\in\Pi$.
This set is finite, because we only store equalities between different representatives and not the implicit equalities following from (D3) and (D4).
We denote these explicitly stored disequalities as $closure_\neq(\Psi)$.

This set can be computed in polynomial time in the size of the formula $\Psi$ by going through all the
disequalities $t_1 \neq z + t_2$ in $\Psi_\neq$ and storing the disequalities
$\otau t_1  \neq (\omega t_2 - \omega t_1 + z) + \otau t_2$ and $\otau t_2 \neq - (\omega t_2 - \omega t_1 + z) + \otau t_1$.
Then for each disequality of the type $*(z_1 + t_1) \nequivp *(z_2 + t_2)$ that follows from a stored disequality or from the congruence closure of $\Psi_=$ or from a block disequality,
the disequalities propagate backwards using the rule (D2).
This only adds a finite amount of disequalities to the set, given that each equality and each block disequality implies at most one disequality of the type $*(z_1 + t_1) \nequivp *(z_2 + t_2)$.

For each block disequality $bl(t_1) \neq bl(t_2) \in \Psi$, we store the disequalities $bl(\otau t_1) \neq bl(\otau t_2)$.
We call the set of these disequalities $closure_{bl}(\Psi)$.

\begin{lemma}\label{lemma:diseq_types}
  $t_1 \nequivp t_2 + z$ iff one of the following holds:

  \begin{enumerate}
    \item $\otau t_1 \neq (\omega t_1 - \omega t_2 + z) + \otau t_2 \in closure_\neq(\Psi)$.
    \item Either $bl(\otau t_1) \neq bl(\otau t_2) \in closure_{bl}(\Psi)$ or $bl(\otau t_2) \neq bl(\otau t_1) \in closure_{bl}(\Psi)$.
    \item $t_1 = z' + t_2$ for any $z' \neq z$ is implied by $\Psi$.
  \end{enumerate}
  We call the disequalities that follow from 1.\ the \emph{explicit} disequalities and those following from 2.\ and 3.\ the \emph{implicit} disequalities.
\end{lemma}

\section{Extension to a superset of terms}

When new terms appear during the analysis, the set $\T$ and the partition $P$ need to be updated
in order to be able to express properties about the new terms.
It is not sufficient to simply add the new terms to a fresh equivalence class,
as they sometimes need to be added to existing equivalence classes.

\begin{example}
    Let $\Psi \equiv A = B$ and $\T = \{A, B, *A\}$.
    The parttion $\Pi$ is equal to $\{\{A, B\}, \{*A\}\}$.
    If we want to add the term $*B$ to $\T$, we need add it to the same equivalence class as $*A$,
    as they are equivalent according to the rules defined in \cref{chapter:qcc}.
    \todo{The same example is already somewhere else}
\end{example}

Given a partition $P = (\Pi, \tau, \omega)$ of set of terms $\T$ and a set $\T'$, which is a superset of $\T$ and closed under subterms,
we define the operation $\ext{k}{\T'}$ that extends the kernel $k = \angl{\T, P, M, B, D}$ to the set of terms $\T'$,
without altering the semantics of the representation.

The extension is built inductively based on the structure of the terms in $\T'$.
We describe how to add a new atom and a term of the form $*(z+t)$, where $t$ is already a part of $\T$.
This can be repeated for all terms in $\T'$.

In the case that $\T' = \T \cup \{a\}$ for an atom $a \notin \T$, we simply add a new
equivalence class $\{a\}$ to $\Pi$ and a new state $s_i$ to the states $S$ of $M$.
We define $\tau\,a = a$, $\omega\,a=0$, $\eta(a) = (0,s_i)$, $Q_i = \{a\}$.
The remaining elements of $k$ remain unchanged.

Consider $\T' = \T \cup \{*(z+t)\}$ for a term $t \in \T$, but $*(z + t) \notin \T$.
First we add a new equivalence class $\{*(z+t)\}$ to $\Pi$.
We define $\tau\,(*(z+t)) = *(z+t)$ and $\omega\,(*(z+t))=0$.
Let $Q_i \in \Pi$ be the equivalence class that contains $t$.
If $\delta(z + \omega\,t,s_i)$ is not defined, then we add a new state $s_j$ to $S$
and define $\delta(z + \omega\,t, s_i) = (0, s_j)$ and $Q_j = \{*(z+t)\}$.
Otherwise, let $\delta(z + \omega\,t, s_i) = (z', s_j)$.
Then we perform the operation \union{*(z + t)}{\tau\,Q_j}{z'}.
In this case, we do not need to add a new state to the automaton,
as the new term is added to the existing equivalence class $Q_j$.

\section{Restriction to a subset of terms}

\todo[inline]{Change it to a positive set instead of a negative set.}

During an assignment, it is necessary to forget information about a specific set of terms.
For instance, when assigning a value to a term, all terms in the domain that might be modified by this assignment---those that may alias with this term---must be removed.
This section explains how to restrict a kernel $k = \angl{P, M_P, B, D}$
to exclude all propositions containing terms from a specified set.

We begin by defining the restriction of a partition $P = (\T, \tau, \omega)$ and its corresponding automaton $M_P = (S, \otau, \eta, \delta)$ to forget information about the terms in a set $\Set$.
The complement of $\Set$ needs to be closed under subterms, ensuring that the
resulting partition will still contain terms from a set that remains closed under subterms.

Given a set $\Set$, we can find a restricted
automaton $\restr{M_P}{-\Set} = (S', \otau', \eta', \delta')$ such that $\L(\restr{M_P}{-\Set})$ does not include terms from
$\Set$, but retains the equalities between all other terms in $\L(M_P) \setminus \Set$.
From this, we derive the restricted partition $\restr{P}{-\Set} = (\T', \tau', \omega')$, where $\T'$ contains
the terms in $\T \setminus \Set$ and additionally all terms that appear in the codomain of the function $\otau'$.

This is necessary to ensure that for each state $s \in S'$, a representative term in $\T'$ can be found,
even if $\T \setminus \Set$ does not contain any term in $\L_{\restr{M_P}{-\Set}}(s)$.

\begin{example}
    Consider the conjunction $\Psi \equiv \&x = \&z \land *x = \&y$.
    We choose $\T = \{\&x, \&y, \&z, x, *x\}$. The corresponding automaton $M_P$ can be visualized as follows:

    \begin{center}
        \begin{tikzpicture}[shorten >=1pt, node distance=3cm, on grid, auto,
                state/.style={rectangle, rounded corners, draw, inner sep=3pt, align=center}]
            \node[state] (q_0) {$\&x, \&z$};
            \node[below=0.5cm of q_0] {$s_1$};
            \node[state] (q_1) [right=of q_0] {$x$};
            \node[below=0.5cm of q_1] {$s_2$};
            \node[state] (q_2) [right=of q_1] {$*x, \&y$};
            \node[below=0.5cm of q_2] {$s_3$}; % Label for the third state

            \path[->]
            (q_0) edge node {} (q_1)
            (q_1) edge node {} (q_2);
        \end{tikzpicture}
    \end{center}
    Let $\Set$ be the set of terms that contain the variable $x$.
    If we simply remove all terms from $\Set$ from the automaton, the state $s_2$ would
    not contain any terms from $\T \setminus \Set$.
    However, the set $\L_{M_P}(s_2)$ contains not only $x$, but also $z$, which is not in $\Set$.
    Therefore, we can choose $z$ as the new representative for $s_2$.
    The new set of terms after the restriction is $\T' = (\T \setminus \Set) \cup \{z\} = \{\&y, \&z, z\}$.
    The resulting automaton can be visualized as follows:
    \begin{center}
        \begin{tikzpicture}[shorten >=1pt, node distance=3cm, on grid, auto,
                state/.style={rectangle, rounded corners, draw, inner sep=3pt, align=center}]
            \node[state] (q_0) {$\&z$};
            \node[below=0.5cm of q_0] {$s_1$};
            \node[state] (q_1) [right=of q_0] {$z$};
            \node[below=0.5cm of q_1] {$s_2$};
            \node[state] (q_2) [right=of q_1] {$\&y$};
            \node[below=0.5cm of q_2] {$s_3$};

            \path[->]
            (q_0) edge node {} (q_1)
            (q_1) edge node {} (q_2);
        \end{tikzpicture}
    \end{center}
    It represents the formula $*z = \&y$.
    \todo{add edge weights}
\end{example}

The restriction $\restr{M_P}{-\Set} = (S', \otau', \eta', \delta')$ is computed using a breadth-first search (BFS) on the automaton $M_P$
to determine which paths remain reachable after excluding all terms in $\Set$.
For some states $s \in S$, the representative $\otau(s)$ might be in $\Set$.
If this is the case, we need to find a new representative for $s$ from the set $\L_{M_P}(s) \setminus \Set$.
If this set is empty, the state $s$ must be removed from the automaton.

The algorithm proceeds as follows, with $\V$ representing the set of visited states and $Q$ being a queue of states to be processed:
\begin{enumerate}
\item Start with the initial states of $M_P$ and consider each atom $a \in \T \setminus \Set$ where $\eta(a) = (z, s)$:
   \begin{itemize}
       \item If $s$ has not been visited, choose a new representative for $s$.
       Set $\otau'(s) = a$ and $\eta'(a) = (0, s)$.
       Add $s$ to $\V$ and to $Q$.
       \item If $s$ has already been visited, there exists a $t$ such that $\otau'(s) = t$.
       Let $M_P[t] = (z', s)$.
       Define $\eta'(a) = (z - z', s)$.
   \end{itemize}

\item Remove a state $s_1$ from the queue $Q$.
Consider all outgoing transitions $\delta(z_1, s_1) = (z_2, s_2)$.
   \begin{itemize}
       \item If a term $*(z+t) \in \L(M) \setminus \Set$ can be found such that $M_P[t] = (z_1', s_1)$ and $z = z_1 - z_1'$,
       the transition remains valid.
           \begin{itemize}
               \item If $s_2$ has not been visited, set $\otau'(s_2) = *(z+t)$ and add $s_2$ to $\V$ and to $Q$.
               \item If $s_2$ has been visited, $\otau'(s_2)$ is already defined.
               \item Let $M_P[\otau'(s_1)] = (z_3, s_1)$ and $M_P[\otau'(s_2)] = (z_4, s_2)$.
               Define the corresponding transition $\delta'(z_1 - z_3, s_1) = (z_2 - z_4, s_2)$.
           \end{itemize}
       \item If no such term is found, the transition is no longer valid in $\restr{M_P}{-\Set}$, and $s_2$ is not added to the set of visited states.
   \end{itemize}
\item If $Q$ is not empty, return to step 2 and continue processing.
\end{enumerate}

Essentially, the algorithm retains the transitions between the states that remain reachable after removing the terms and adjusts the weights of these transitions to the new representatives.
The set $S'$ is equivalent to the set of visited states $\V$.

The new partition $\restr{P}{-\Set} = (\T', \tau', \omega')$ is defined as follows:
\begin{itemize}
    \item $\T' = (\T \setminus \Set) \cup \{t \mid \exists s \in S' . \otau'\,s = t\}$,
    \item For each $t \in \T'$, if $\restr{M_P}{-\Set}[t] = (z, s)$, then $\tau'\,t = \otau'\,s$ and $\omega'\,t = z$.
\end{itemize}

\begin{theorem}\label{restriction}
    Let $M_P = (S, \otau, \eta, \delta)$ be a QFA and let $\restr{M_P}{-\Set} = (S', \otau', \eta', \delta')$ be the automaton that is obtained by restricting $M_P$ as described above.
    Then,
    \begin{enumerate}
        \item\label{item:lemma-restriction} for each term $t \in \L(\restr{M_P}{-\Set})$, it holds that $M_P[t] = (z, s)$ and $M[\otau'\,s] = (z', s')$ iff $\restr{M_P}{-\Set}[t] = (z - z', s)$ and $s = s'$,
        \item $\L(\restr{M_P}{-\Set}) = \L(M_P) \setminus \Set$ and
              \item\label{item:correctness-restriction} for each term $t_1, t_2 \in \L(\restr{M_P}{-\Set})$, it holds that $M_P[t_1] = z + M_P[t_2]$ iff $\restr{M_P}{-\Set}[t_1] = z + \restr{M_P}{-\Set}[t_2]$.
    \end{enumerate}
\end{theorem}
\begin{proof}
    The first point can be proven by induction over the structure of the term $t$.
    \todo[inline]{Proof of point 2}
    Point~\labelcref{item:correctness-restriction} follows from point~\labelcref{item:lemma-restriction}.
\end{proof}
%--------------
\subsubsection{(Block-)Disequalities}
We can also find the set of disequalities and the block disequalities that are still valid when considering only the terms $\T'$.
We keep only the propositions between representatives of the equivalence classes that
still have a corresponding state in the automaton $\restr{M_P}{-\Set}$, but we need to update the representatives to the new representatives.
The remaining disequalities are removed.

For $k = \bot$, the restriction $\restr{k}{-\Set}$ is also $\bot$.

Given a conjunction $\Psi$, we denote the restriction of the corresponding kernel $k$ to forget propostions about terms in $\Set$ by $\restr{\Psi}{-\Set}$.\todo{do you need this in the end?}

\section{Equality}

Given two kernels, $k_1$ and $k_2$, we want to decide whether they are semantically equivalent.
This section presents two methods for computing the \emph{equal} operation.
Their main difference is in how the equivalence of two partitions $P_1$ and $P_2$ is decided.
The first method compares the equivalence classes of the partitions $P_1$ and $P_2$.
The second method is based on computing a normal form of the conjunctions represented by the kernels $k_1$ and $k_2$.
It then suffices to compare the normal forms syntactically.
This approach has the advantage that the normal form can be computed once for every kernel, and there is no need to recompute it for every comparison.

\subsection{Comparing Equivalence Classes}

Let $P_1 = (\T_1, \tau_1, \omega_1)$ and $P_2 = (\T_2, \tau_2, \omega_2)$ be two partitions.
First, we extend both partitions to the set $\T = \T_1 \cup \T_2$.
Then we compare the equivalence classes of the resulting partitions $\ext{P_1}{\T} = (\T, \tau_1', \omega_1')$ and $\ext{P_2}{\T} = (\T, \tau_2', \omega_2')$.

For this, we need to check for each element $t \in \T$ if all equalities implied by $P_1$ are also implied by $P_2$.
Let $t' \equiv \tau_1'\,t$ and $z = \omega_1'\,t$.
It follows that $P_1$ implies the equality $t = z + t'$.
If $\omega_2'\,t + \tau_2'\,t = z + \omega_2'\,t' + \tau_2'\,t'$ holds, then the same equality is also implied by $P_2$.

Afterward, the disequalities and block disequalities are compared.
For two kernels $k_1 = \angl{P_1, M_{P_1}, B_1, D_1}$ and $k_2 = \angl{P_2, M_{P_2}, B_2, D_2}$,
we rewrite the disequalities $B_2$ and $D_2$ to be about the representatives of $P_1$.
For each block disequality $\{t_1,t_2\} \in B_2$, we convert it to the disequality $norm_{P_1}(bl(t_1) = bl(t_2))$ and for each disequality in $D_2$ of the form $(t_1, z, t_2) \in D$, we convert it to the disequality $norm_{P_1}(t_1 = z + t_2)$.
The normalization is possible because we previously extended the two partitions to the same set $\T$, i.e., $\tau_1'\,t$ is defined for each term $t$ occurring in $B_2$ and $D_2$.
Then, we check if the resulting set of block disequalities is equal to the set $B_1$ and if the resulting set of disequalities equals the set $D_1$.

\subsection{Compute Normal Form}

\textcite{2pointer} presents a different approach to deciding the equivalence of two partitions.
This is done by computing a normal form $\nf(\Psi)$ of a conjunction $\Psi$, such that $\Psi$ is semantically equivalent to $\nf(\Psi)$ and such that two conjunctions $\Psi_1$ and $\Psi_2$ are semantically equivalent iff $\nf(\Psi_1)$ and $\nf(\Psi_2)$ are syntactically equivalent.

The main idea is to find \emph{minimal representatives} for each state $s$ in the QFA $M_P$,
which corresponds to the smallest term in $\L_{M_P}(s)$, according to the order defined in \cref{chapter:2pointer}.
Then, the automaton is transformed to a formula that utilizes only the minimal representatives, thus obtaining a normal form that is independent of the chosen set $\T$ and on the chosen representatives.

Let $M_P = (S, \otau, \eta, \delta)$ be a QFA.\@
For each state $s \in S$, we compute the minimal term $m_s$ and the corresponding offset $z_s$ such that $M[m_s] = (z_s,s)$.
This can be computed using a variation of Dijktra's shortest path algorithm, which is described in the following.

Let $P$ be an initially empty FIFO queue, which we use to store all states $s$ for which $(m_s,z_s)$ is already computed but where the outgoing edges still have to be processed.

\begin{enumerate}
    \item
          First, we consider all atoms $a$ for which $\eta$ is defined in ascending order. For each $\eta\,a = (z,s)$, if $(m_s,z_s)$ is not defined yet, we set
          \[
              (m_s,z_s) = (a,z)
          \]
          and add $s$ to $P$.
    \item
          Then, we process the queue $P$ in a FIFO manner until it is empty.
          For each state $s$ in $P$, we consider all outgoing edges $\delta\,z\,s = (z',s')$ in order of ascending $z$.
          For each such edge where $(m_{s'},z_{s'})$ is not defined yet, we set
          \[
              (m_{s'},z_{s'}) = (*((z-z_s)+m_s),z')
          \]
          and add $s'$ to $P$.
\end{enumerate}

The correctness is given by the fact that longer paths always result in larger terms than shorter paths.
Thus, we do not need to update a pair once it was computed.

\begin{example}\label{ex:min-repr}
    Let $\Psi \equiv B = -1 + \&x \land *B = 2 + A \land *A = 3 + x$.
    The automaton $M_P$\todo{what is P} has can be visualized with the following graph:
    \begin{center}
        \begin{tikzpicture}[shorten >=1pt, node distance=3cm, on grid, auto, state/.style={rectangle, rounded corners, draw, inner sep=5pt, align=center}]

            \node[state] (Q1) {$\begin{array}{c}
                        \boldsymbol{\&x}, \\
                        1 + B
                    \end{array}$};

            \node[state, right=4.5cm of Q1] (Q2) {$\begin{array}{c}
                        \boldsymbol{*B}, 2 + A, \\
                        *(-1+\&x )
                    \end{array}$};

            \node[state, right=5.6cm of Q2] (Q3) {$\begin{array}{c}
                        \boldsymbol{x}, *(1 + B), -3 + *A, \\
                        -3 + *(-2+*B),...
                    \end{array}$};

            \path[->]
            (Q1) edge[] node {$-1,0$} (Q2)
            (Q2) edge[] node {$-2,3$} (Q3)
            (Q1) edge[bend right] node {$0,0$} (Q3);

            \node [above left=1cm and 0.3cm of Q1] {$s_0$};
            \node [above left=1cm and 0.5cm of Q2] {$s_1$};
            \node [above left=1cm and 0.7cm of Q3] {$s_2$};

        \end{tikzpicture}
    \end{center}

    The minimal representatives are computed by first considering the atoms in ascending order.
    Assuming that the linear order on atoms is $\&x < A < B$,
    we define first $m_{s_0} = \&x$ and $z_{s_0} = 0$.
    Then we define $m_{s_1} = A$ and $z_{s_1} = -2$.
    There is nothing to do for $B$ because we already defined $m_{s_0}$.
    The queue $P$ is now $\{s_0, s_1\}$.
    We consider the outgoing edge of $s_0$ with $z = -1$.
    There is nothing to do for $s_1$, as $m_{s_1}$ is already defined.
    Then we consider the outgoing edge of $s_0$ with $z = 0$
    and set $m_{s_2} = x$ and $z_{s_2} = 0$.
    Thus, we have found the minimal representative for each state.
\end{example}
Using the minimal representatives, we can now transform the kernel $k = \angl{P,M_P,B,D} \neq \bot$ with $M_P = (S, \otau, \eta, \delta)$ to a normal form $\F_{normal}[k]$,
defined as follows:

\[
    \begin{array}{lll}
        \F_{normal}[k] & \equiv & \bigwedge_{\eta\,a=(z,s)} (a = (z - z_s) + m_s) \land                                                                     \\
                       &        & \bigwedge_{\delta\,z\,s=(z',s')} (*((z-z_s) + m_s) = ((z'-z_{s'}) + m_{s'}) \land                                         \\
                       &        & \bigwedge_{s,s' \in S \land \{\otau\,s,\otau\,s'\}\in B} (bl(m_{s}) \neq bl(m_{s'})) \land           \\
                       &        & \bigwedge_{s,s' \in S \land (\otau\,s, z,\otau\,s') \in D} (m_{s} \neq (z+ z_{s} -z_{s'}) + m_{s'})
    \end{array}
\]

The trivial propositions of the form $t = 0 + t$ are removed from the formula, as well as the repeated equalities and the implicit disequalities of the form $\&x \neq z + \&y$ and $bl(\&x) \neq bl(\&y)$.

The definition of $\F_{normal}[k]$ is analogous to the formula representation $\F[k]$, but it expresses the proposition using the minimal representatives instead of the union-find representatives.
The advantage of this normal form is that the formula representation depends neither on the chosen representatives nor the chosen set $\T$.
Thus, there is a unique normal form for each semantic equivalence class of kernels, even though multiple kernel representations of the same conjunction exist.
Therefore, in order to decide the equivalence of two kernels $k_1$ and $k_2$, it suffices to compare the normal forms $\F_{normal}[k_1]$ and $\F_{normal}[k_2]$ syntactically.

\begin{example}
    Consider the conjunction $\Psi$ from \cref{ex:min-repr}.
    The normal form of the kernel $k$ representing $\Psi$ is
    \[
        (B = -1+\&x) \land (*(-1 + \&x) = 2 + A) \land (*A = 3 + x)
    \]
    where $B = -1 + \&x$ originates from $\eta\,B = (-1,s_0)$.
    The equalities originating from $\eta\,\&x = (0,s_0)$ and $\eta\,A = (0, s_1)$ are removed, because they are trivial.
    The edge $\delta\,(-1,s_0) = (0,s_0)$ gives rise to the equality $*(-1 + \&x) = 2 + A$ and $\delta\,(-2,s_1) = (3,s_2)$ generates the equality $*A = 3 + x$.
    The last edge $\delta\,(0,s_0) = (0,s_2)$ generates the trivial equality $x = 0 + x$.
\end{example}

\todo[inline]{Correctness proof}

\todo[inline]{Proof that they are equivalent}

The advantage of computing the normal form instead of comparing the equivalence classes is that it only needs to be computed once per kernel, and the resulting normal form can be reused for each equality check.
In the implementation, it is possible to configure which of the two algorithms for equality is used by the analysis.
If the normal form algorithm is chosen, then the kernel contains an additional field containing the normal form.
This normal form is \emph{lazily} computed, i.e., it will only be calculated when needed, and it is not computed if it is never used.
Once calculated, the result is stored directly in the kernel and can be reused.


\section{Partial Order}

The natural partial ordering between conjunctions is the semantic implication, i.e., $\Psi_1 \rightarrow \Psi_2$ if and only if $(\rho, \nu, \mu) \models \Psi_1$ whenever $(\rho, \nu, \mu) \models \Psi_2$.
This is the case if and only if $\Psi_1 \land \Psi_2$ is semantically equal to $\Psi_1$.
Therefore, it is possible to reduce the \emph{less equal} operation to a \emph{meet} and an \emph{equal} operation.
A kernel $k_1$ that represents a conjunction $\Psi_1$ is \emph{less or equal} to a kernel $k_2$ representing $\Psi_2$ if and only if $k_1 \meet k_2$ is semantically equal to $k_1$.
We have already described how semantic equality between two kernels can be decided.
In the next section, the \emph{meet} operation ($\meet$) is presented.

