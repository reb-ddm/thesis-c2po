\chapter{2-Pointer Logic}\label{chapter:2pointer}


The abstract domain of the \cpo\ analysis consists of finite conjunctions of propositions from the 2-Pointer Logic.
In the following section, we describe this logic and how to represent the propositions in order to
infer also all deriving propositions.
It was first introduced by Seidl et al.\ in~\cite{2pointer} and was later extended with an additional type of proposition.
The propositions are made up of terms, formed from address constants and auxiliaries using the addition of constant offsets and dereferencing.
The address constants represent addresses of program variables.
The auxiliaries are variables whose address is never taken, therefore
we know of these variables that their address cannot be reached from other terms using address arithmetic or dereferencing.
Only terms whose C type is a pointer type are considered by the analysis.
Let $\X$,$\A$ be disjoint finite sets of variables names and auxiliaries.
The terms $t$ are defined by the following grammar:

\[
    t\,{::=}\,A \mid \&x \mid *(z+t)
\]
where $A\in \A$ is an auxiliary, $x \in \X$ is a variable with address $\&x$ and $z \in \Z$ is an integer.
We call terms $\&x$ or $A$ \emph{atoms}.
Given a linear order $<$ on atoms, it can be extended to an order on terms by defining:
\begin{itemize}
    \item $a < *(z+t)$ whenever $a$ is an atom and
    \item $*(z_1 + t_1) < *(z_2 + t_2)$ whenever either $t_1 < t_2$ or $t_1 \equiv t_2$ and $z_1 < z_2$.
\end{itemize}

The terms can form three types of propositions: equalities and disequalities between terms and additionally a proposition which states that two terms do not belong to the same address block.
The propositions $p$ are defined by the following grammar:

\[
    p\,{::=}\,t_1=z+t_2 \mid t_1\neq z+t_2\mid bl(t_1) \neq bl(t_2)
\]

Where $*$ denotes the dereferencing operator and $bl(t)$ represents the address block of $t$.
We use $*t$ as an abbreviation for $*(0+t)$ and $x$ as an abbreviation for $*(0+\&x)$.
For a conjunction $\Psi$ of propositions, we denote the conjunctions of all the equalities in $\Psi$ by $\Psi_{=}$, the conjunction of disequalities between terms by $\Psi_{\neq}$ and the set of block disequalities of the type $bl(t_1) \neq bl(t_2)$ by $\Psi_{bl}$.




\section{Representation of C Expressions as 2-Pointer Logic Terms}

Since the \cpo\ analysis only considers C expressions with pointer types, they have a uniform size.
For instance, on an x86-64 architecture, pointers occupy 64 bits in the memory.

However, the offsets added to pointers vary depending on the pointer's type.
For example, if we have a pointer $p$ of type $int32_t\,*p$, the expression $p + z$ represents an offset of $32 \cdot z$ bits.
If the same pointer $p$ is cast to a pointer of type $int64_t\,*p$, the offset $z$ is of $64 \cdot z$ bits.
It is essential to consider this to know the relative positioning of
different expressions in the memory and to discover possible overlapping data.
Therefore, in our implementation, the offsets in the abstract terms representing C expressions are expressed in bits rather than the value present in the code.
For instance, if $p$ is a pointer to a 32-bit integer and $q$ is a pointer to a 64-bit integer, then the analysis represents the C expression $*(p + 1)$ as $*(p + 32)$, and $*(q + 1)$ as $*(q + 64)$.

In some cases, it is impossible to determine the offset of a pointer expression, e.g., if the offset is an unknown variable instead of a constant.
Here, the \goblint\ constant propagation analysis is used to infer the offset value.
If the value remains indeterminable, the expression is treated as unknown.

Furthermore, arrays and structs are handled similarly to pointers in C, allowing us to express properties not only about pointers but also about arrays and structs.
For example, in C, the expression $a[1]$ is equivalent to the expression $*(a + 1)$.
Hence, the analysis treats arrays as pointers to their first element.
A two-dimensional array in C with dimensions $n$ and $m$ can be viewed as a one-dimensional array with $n \cdot m$ elements.
The expression $a[i][j]$ is treated as $*(a + (i \cdot m + j) \cdot s)$, where $s$ is the size in bits of the array elements.
Similarly, arrays with more dimensions are treated as pointers to a one-dimensional array.

For a struct $s$, the analysis treats $\&s$ as a pointer to the first field of the struct.
For a field $f$ of the struct $s$ with an offset of $z$ bits,
the expression \textsf{s.f} is treated as $*(\&s + z)$.
In C, the offset of a struct element may not always be a multiple of 8 bits.
Thus, offsets are represented in bits rather than bytes.
The analyzer \goblint\ can determine the bit offsets of fields within a struct in most cases.
\cpo\ leverages \goblint\ to find these offsets, and in the rare instances where \goblint\ cannot determine the exact offset, the expression is treated as unknown.


\begin{example}
    Consider this C struct:
    \begin{minted}{C}
        struct S {
            int32_t first;
            int32_t arr[5];
        };
    \end{minted}
    Let $x$ be a variable of type \textsf{struct S}.
    The expression \textsf{x.arr[4]} is treated as $*(\&x + 160)$.
    Inside the struct, the offset of $arr$ is $32$ and the offset of $arr[4]$ is $32\cdot 4 = 128$.
    In total, the offset is $32 + 128 = 160$.
\end{example}
%struct S x[3][2];
%struct S {
% int first;
% int arr[5];
%   };
%x[2][1].arr[4]
%*(1120+x[Ptr64])[Int32]

\section{Quantitative Congruence Closure}\label{chapter:qcc}

Given a conjunction $\Psi$ of propositions, we want to find all equalities logically implied by $\Psi$.
As infinitely many such equalities exist, we only consider the ones containing terms exclusively from a specific set $\T$.
The set $\T$ must be subterm-closed and contain at least all the terms in $\Psi$.
The equalities implied by $\Psi$ containing terms from $\T$ are described by the quantitative equivalence relation $\equivp$, which is the smallest equivalence relation $\equivp$ satisfying these rules:
\begin{enumerate}[label={[E\arabic*]}, ref={[E\arabic*]}]
  \setcounter{enumi}{-1}
  \item\label{item:persistence} If $t_1 = z+t_2$ occurs in $\Psi$, then $t_1 \equivp z+t_2$;
  \item\label{item:quantitative-reflexivity} $t \equivp 0+t$ for all $t\in\T$ (\emph{quantitative reflexivity});
  \item\label{item:quantitative-symmetry} If $t_1 \equivp z+t_2$, then $t_2 \equivp -z+t_1$ (\emph{quantitative symmetry});
  \item\label{item:quantitative-transitivity} If $t_1 \equivp z_1+t_2$ and $t_2 \equivp z_2 + t_3$,
  then $t_1 \equivp (z_1+z_2)+t_3$ (\emph{quantitative transitivity});
  \item\label{item:dereferencing} If $t_1 \equivp (z_2 - z_1) + t_2$, then $*(z_1 + t_1) \equivp *(z_2 + t_2)$ holds as well, whenever $*(z_1 + t_1),  *(z_2 + t_2)$ are in $\T$ (\emph{dereferencing}).
\end{enumerate}

\subsection{Quantitative Partition}\label{subsection:quantitative-union-find}

We can represent the quantitative equivalence relation $\equivp$ as a partition.
The \emph{quantitative partition} $P = (\T, \tau, \omega)$ consists of a set of terms $\T$,
a function $\tau : \T \rightarrow \T$ that assigns a \emph{representative} to each term $t \in \T$ and a mapping $\omega : \T \rightarrow \Z$,
which assigns to each term its offset from the representative of its equivalence class.
These two functions model equalities of the form $t \equivp \omega\,t + \tau\,t$ for each $t \in \T$.
All terms with the same representative belong to the same equivalence class, i.e., they are equivalent up to an integer offset.

We denote as $P[\Psi]$ the quantitative partition that model the conjunction $\Psi$.
The set of representative terms $rep(P[\Psi])$ is the set of terms $t \in \T$ that occur on the right-hand side of the function $\tau$, i.e., there exists a $t'\in \T$ such that $\tau\,t' = t$.
For each $t\in rep(P[\Psi])$, it holds that $\tau\,t=t$ and $\omega\,t = 0$.

The quantitative partition $P = (\T, \tau, \omega)$ is useful to define a normal form for propositions.
Given an equality $p \equiv t_1 = z + t_2$ and a partition $P = (\Pi, \tau, \omega)$,
the normalized equality $norm_P(p)$ is an equivalent equality that only contains representatives of $P$,
i.e., $\tau\,t_1 = z - \omega\,t_1 + \omega\,t_2 + \tau\,t_2$.
Equivalently, the normalized disequality $norm_P(p)$ for $p \equiv t_1 \neq z + t_2$ is the disequality $\tau\,t_1 = z - \omega\,t_1 + \omega\,t_2 + \tau\,t_2$.
The normalized block disequality $norm_P(p)$ for $p \equiv bl(t_1) \neq bl(t_2)$ is the disequality $bl(\tau\,t_1) \neq bl(\tau\,t_2$).

\subsubsection{Implementation}

The quantitative partition is represented using a data structure similar to union-find but extended with integer offsets.
The union-find data structure is represented as a forest of trees, each representing an equivalence class.
Each node in the tree corresponds to a term $t \in \T$, and the roots are representative elements $\tau\,t$.\cite{uf-tarjan}
All the edges in the tree are directed towards the root.
In order to represent the integer offset, each edge is labeled with an integer weight.

The forest is represented as a map $parent : \T \rightarrow \Z \times \T$ that maps each term $t$ to an integer offset and its parent node in the tree.
The nodes that have themselves as a parent are the representative terms.
The mapping $parent(t_1) = (z,t_2)$ defines an edge from $t_1$ to $t_2$ with weight $z$ in the graph, and it implies the equality $t_1 \equivp z + t_2$.

In order to compute the representative of a term in the tree, there is an operation $\find{t}$.
It returns the pair $(\omega\,t,\tau\,t)$.
The offset $\omega t$ can be computed by following the path from $t$ to the tree's root and summing up the weights of the encountered edges.
The representative $\tau\,t$ is the root of the tree containing $t$.

The union-find tree is built incrementally by adding consecutively equalities.
Initially, each term $t\in\T$ is its own parent, and equalities $t_1 = z + t_2$ between terms can be added by performing the $\union{t_1}{z}{t_2}$ operation.
This operation merges two trees by adding an edge from $\tau\,t_1$ to $\tau\,t_2$ with weight $z - \omega\,t_1 + \omega\,t_2$,
thus merging the two equivalence classes of $t_1$ and $t_2$ with the correct offset.
If the two terms are already equivalent with a different offset, i.e., $\tau\,t_1 = \tau\,t_2$, but $z - \omega\,t_1 + \omega\,t_2 \neq 0$, then the conjunction is unsatisfiable.

\begin{example}
  Consider the conjunction $\Psi \equiv (B = 1 + A) \land (D = 2 + \&x) \land (A = 3 + \&x)$.
  The tree computed by the union-find algorithm is illustrated in \cref{fig:uf-tree}.
  \begin{figure}
    \begin{tikzpicture}[->, node distance=2cm, auto]
      \node (x) {$\&x$};
      \node (A) [below left of=x] {$A$};
      \node (D) [below right of=x] {$D$};
      \node (B) [below=1cm of A] {$B$};

      \path (A) edge node {3} (x);
      \path (B) edge node {1} (A);
      \path (D) edge node {2} (x);
    \end{tikzpicture}
    \caption{Union-find tree corresponding to the conjunction  $(B = 1 + A) \land (D = 2 + \&x) \land (A = 3 + \&x)$}\label{fig:uf-tree}
  \end{figure}
  From this representation, we can, for example, derive that $B = 4 + \&x$.
  The tree can look different depending on the choice of the representatives.
\end{example}

In a practical implementation, the representatives are chosen such that the trees have the lowest possible height, thus making the $\find{t}$ operation more efficient.
This optimization is called \emph{union by rank} and consists of adding the edge from the smallest to the largest equivalence class during the $\union{t_1}{z}{t_2}$ operation.
Another possible optimization is, during the $\find{t}$ operation, to set the parent node of a term to the representative of the tree each time a node is traversed.
This way, the tree is flattened, which makes the $\find{t}$ operation
more efficient if we call it again for the same term $t$ or a term in the same equivalence class.
This is the so-called \emph{path compression}.~\cite{uf-tarjan}

\subsection{Quantitative Finite Automaton}\label{subsection:qfa}

A partition $P = (\T, \tau, \omega)$ groups all terms that are equal up to an integer offset.
The partition takes into account the reflexivity, symmetry, and transitivity rules.
Further, it is necessary to ensure that the equalities deriving from \labelcref{item:dereferencing} are always considered.
Therefore, for each term $t$, we need to keep track of the equivalence classes that contain terms of the form $*(z+t)$ and merge two equivalence classes if they contain terms that are equal according to the rule \labelcref{item:dereferencing}.
For example, if $\T = \{A, B, *A, *B\}$ and we perform a union of $A$ and $B$, the closure rule~\labelcref{item:dereferencing} tells us that also the equivalence classes of $*A$ and $*B$ should be merged.
Thus, we introduce an alternative representation of the partition $P$ as a \emph{quantitative finite automaton} (QFA) $M_P$.

The QFA is defined by a triple $M_P = (S, \otau, \eta, \delta)$. $S$ is a finite set of states, each representing an equivalence class of the partition.
In order to know which state represents which equivalence class of the partition, the mapping $\otau : S \rightarrow \T$ assigns a representative term to each state.
If $\otau\,s = t$ for a state $s \in S$, then $s$ represents the equivalence class of all terms whose representative in $P$ is $t$.
The partial mapping $\eta : (\mathcal{A} \cup \{\&x | x \in \X\} \rightarrow \Z \times S)$ provides initial offsets and states for atoms, meaning that if $\eta\,a = (z,s)$, then $a$ is in the equivalence class represented by $s$.
$\delta : \Z \rightarrow S \rightarrow \Z \times S$ is the partial transition function.
If $\delta\,z_1\,s_1 = (z_2, s_2)$, this means that it holds that $*(z_1 + \otau\,s_1) \equivp z_2 + \otau\,s_2$.

The QFA $M_P = (S, \otau, \eta, \delta)$ is constructed from a partition  $P = (\T, \tau, \omega)$ as follows:
\begin{itemize}
  \item $S$ contains a state $s$ for each representative term $t$ of $P$ and we set $\otau\,s = t$.
  \item $\eta\,a = (\omega\,a, s)$ if $s \in S$ and $\otau\,s = \tau\,a$
  \item $\delta\,z\,s_1 = (\omega\,t', s_2)$ if there is a $\tau\,t'=\otau\,s_2$, such that $t' \equiv *(z_1 + t_1)$ with $\tau \,t_1= \otau\,s_1$ and $z = z_1 + \omega\,t_1$.
\end{itemize}

We remark that following this definition, a transition $\delta\,z\,s_1$ could derive not only from a single term $t'$ but also from a second term $*(z_2 + t_2) \in \T$ with $\tau\,t_2=\otau\,s_1$ and $z = z_2 + \omega\,t_2$.
However, we show that the resulting mapping $\delta\,z\,s_1$  would have the same value in this case.
Since $t_i \equivp (\omega t_i) + (\tau Q_i)$ for $i = 1,2$, it follows that $t_1 \equivp (z_2 - z_1)+ t_2$.
Therefore $\tau\,*(z_2+t_2)=\otau\,s_2$ with offset $\omega(*(z_2+t_2)) = \omega\,t'$.
This shows that the result of $\delta\,z\,s_1$ is well defined and does not depend on which term $t'$ we choose for deriving a transition.
Moreover, we note that $\delta$ is defined only for a finite amount of values, given that each term $*(z + t) \in \T$ defines at most one mapping for $\delta$.

We use $\oT$ to denote the set of \emph{all} terms with variable names from $\X$ and auxiliaries from $\A$.
We can extend the mappings $\eta$ and $\delta$ to a partial mapping $M : \oT \rightarrow \Z \times S$ where $M_P[a] = \eta(a)$ for atoms $a$, and $M_P[*(z+t_1)] = \delta(z+z_1, s)$ for terms $t_1$ if $M_P[t_1] = (z_1,s)$.

Furthermore, we define the set $\L(M_P)$ of terms $t \in \oT$ for which $M$ is defined.
For each state $s \in S$, the set $\L_{M_P}(s)$ is the set of terms $t \in \oT$ for which it holds that $M_P[t] = (z, s)$ for some $z \in \Z$.

Using the automaton, we define the $\closure{t_1}{z}{t_2}{P}$ operation that modifies the partition $P = (\T, \tau, \omega)$ and the automaton $M_P = (S, \otau, \eta, \delta)$ to include the equality $t_1 = z + t_2$ and then computes the closure of the rule \labelcref{item:dereferencing}.
The resulting partition $P' = (\T, \tau', \omega')$ and automaton $M_{P'} = (S', \otau', \eta', \delta')$  are computed as follows:
\todo{reread}
\begin{itemize}
  \item Case 1: $\tau\,t_1 = \tau\,t_2$. If $\omega\,t_1 = z + \omega\,t_2$, then $P' = P$ and $M_P = M_{P'}$.
        If $\omega\,t_1 \neq z + \omega\,t_2$, then the conjunction is unsatisfiable.
  \item Case 2: $\tau\,t_1 \neq \tau\,t_2$. We call $\union{t_1}{z}{t_2}$.
        Afterwards, either $t_1$ or $t_2$ has a new representative, so the transitions of the QFA are updated accordingly.
        W.l.o.g., let $\tau\,t_1$ be the new representative of $t_1$ and $t_2$:
        \begin{itemize}
          \item Let $\otau\,s_1 = \tau\,t_1$ and $\otau\,s_2 = \tau\,t_2$.
                After the union, the state $s_2$ will not be in $S'$, as the equivalence classes of $s_1$ and $s_2$ are merged.
                Therefore, we need to move all outgoing transitions of $s_2$ to be transitions of $s_2$.
                For each transition $\delta(z_2, s_2) = (z_3, s_3)$, we set $\delta'(z_1, s_1) = (z_3, s_3)$, where $z_1 = z_2 - \omega\,t_1 + \omega\,t_2 - z$, if $\delta(z_1,s_1)$ is not already defined.
          \item If $\delta(z_1,s_1) = (z_4, s_4)$ is defined, then the equivalence classes of $s_3$ and $s_4$ need to be merged later.
                We add $(\otau\,s_3 = \omega\,(\otau\,s_4) - \omega\,(\otau\,s_3) + \otau\,s_4)$ to a queue $Q$ of pending equalities.
          \item For each incoming transition $\delta(z_5, s_5) = \delta(z_2', s_2)$, we set $\delta'(z_5, s_5) = (z_2' - z_1, s_2)$.
                The remaining transitions of $\delta$ are added to $\delta'$ without modifications.
          \item Then we call $\closure{t_1}{z}{t_2}{P'}$ for each equality $t_1 = z + t_2 \in Q$.
        \end{itemize}
\end{itemize}

This algorithm is similar to congruence closure \cite{cc-tarjan,cc-shostak}, but it is restricted to a unary uninterpreted function symbol $*$ and extended with integer offsets.

\begin{figure}\begin{subfigure}{0.5\textwidth}
\begin{centering}
  \begin{tikzpicture}[shorten >=1pt, node distance=3cm, on grid, auto, state/.style={rectangle, rounded corners, draw, inner sep=5pt, align=center}]

    \node[state] (A) {$A$};

    \node[state, below=1cm of A] (B) {$B$};

    \node[state, right of=A] (As) {$*A$};

    \node[state, right of=B] (Bs) {$*(1 + B)$};

    \path[->]
    (A) edge[] node {$0,0$} (As)
    (B) edge[] node {$1,0$} (Bs);

    \node[above left=0.5cm and 0.3cm of A] {$s_0$};
    \node[above left=0.5cm and 0.3cm of B] {$s_1$};
    \node[above left=0.5cm and 0.3cm of As] {$s_2$};
    \node[above left=0.5cm and 0.3cm of Bs] {$s_3$};

  \end{tikzpicture}
\end{centering}
  \caption{A QFA representing the empty conjunction.}\label{fig:initial-example}
\end{subfigure}
\begin{subfigure}{0.5\textwidth}
  \begin{centering}

  \begin{tikzpicture}[shorten >=1pt, node distance=3cm, on grid, auto, state/.style={rectangle, rounded corners, draw, inner sep=5pt, align=center}]

  \node[state] (AB) {$\begin{array}{c}
    \boldsymbol{A} = \\
    1 + B
  \end{array}$};

  \node[state, right of=AB] (ABs) {$\begin{array}{c}
    \boldsymbol{*A} = \\
    *(1 + B)
  \end{array}$};

  \path[->]
  (AB) edge[] node {$0,0$} (ABs);

  \node[above left=0.9cm and 0.3cm of AB] {$s_0$};
  \node[above left=0.9cm and 0.3cm of ABs] {$s_2$};
\end{tikzpicture}
\end{centering}

  \caption{The same QFA after the operation $\closure{A}{1}{B}{P}$.}\label{fig:closure-example}
\end{subfigure}

\caption{An example of the $\closure{A}{1}{B}{P}$ operation on a QFA over the set of terms
$\T = \{A, B, *A, *(1 + B)\}$.
The representative of each equivalence class is represented in bold as the mentioned first term.}


\end{figure}

\begin{example}
  A QFA can be visualized by labeling the states with the terms of $\T$ that are in the corresponding equivalence class.
  For example, consider the set of terms $\T = \{A, B, *A, *(1 + B)\}$.
  The edges are labeled with the offsets: for an edge $\delta(s_1, z_1) = (s_2, z_2)$, the edge between $s_1$ and $s_2$ is labeled with $z_1, z_2$.

  The initial automaton would be the one in \cref{fig:initial-example}.
  If the operation $\closure{A}{1}{B}{P}$ is called,
  first the $\union{A}{1}{B}$ is performed and the new representative of $B$ becomes $A$, with offset $1$.
  It would be equivalent to choose $B$ as the new representative, but we choose $A$ here as an example.
  Then, we want to move the outgoing transitions from $s_1$ to $s_0$, as the state $s_1$ is removed.
  However, there already exists a transition $\delta(s_0,0) = (s_2, 0)$.
  Therefore, we add the equality $*(1 + B) = *A$ to a queue of pending equalities.
  Then we recursively call $\closure{*(1 + B)}{0}{*A}{P'}$ and thus perform the union of $*(1 + B)$ and $*A$.
  As before, we choose $*A$ as the new representative, but $*(1+B)$ would also be a possible choice.
  The resulting automaton is visualized in \cref{fig:closure-example}.
\end{example}

\todo[inline]{Either prove or remove this theorem.}
\begin{theorem}
  % This is Theorem 3 in the other paper
  Assume that $\Psi$ is a satisfiable conjunction of equalities, let $\T$ be a subterm-closed set of terms which contains all terms occuring in $\Psi$.
  The corresponding quantitative partition $P = (\T, \tau,\omega)$ and a corresponding QFA $M_P = (S, \otau, \eta, \delta)$ are constructed by applying the closure operation with all equalities in $\Psi$.
  Then $\T \subseteq \L(M_P)$ and for every $t_1, t_2 \in \L(M_P)$ the following statements are equivalent:
  \begin{enumerate}
    \item $M_P[t_1] = z + M_P[t_2]$,
    \item $\tau\,t_1 = \tau\,t_2$ and $\omega\,t_1 = z + \omega\,t_2$,
    \item $\Psi$ implies $(t_1 = z + t_2)$.
  \end{enumerate}
\end{theorem}

This theorem was proven in~\cite{2pointer} for the case of a one-dimensional memory model.
The proof can easily be adapted for the two-dimensional memory model.

We remark that the only equalities that can be derived from $\Psi$ are derived from the equalities in $\Psi$. We cannot derive equalities from any of the disequalities.

\section{Block Disequalities}


\section{Disequalities}\label{disequalities}

In order to implement a more precise abstract transformer for the assignment, it is important to know which terms definitely
point to different addresses. Let $\Psi$ be a conjunction of 2-pointer logic terms.
We define the binary relation $\nequivp$ which represents all disequalities that can be derived from $\Psi$.

\begin{enumerate}
  \item[(D0)] If $t_1 \neq z + t_2 \in \Psi$, then $t_1 \nequivp z + t_2$.
  \item[(D1)] If $t_1 \nequivp z + t_2$, then:
    \begin{enumerate}
      \item $t_2 \nequivp - z + t_1$.
      \item $t'_1\nequivp (z_1 + z) + t_2$ if $t_1' = z_1 + t_1$ is implied by $\Psi$.
      \item If $t_1 = z + t_2$ is implied by $\Psi$, then $\Psi$ is Unsat.
    \end{enumerate}
  \item[(D2)] If $*(z_1 + t_1) \nequivp *(z_2 + t_2)$ then also $t_1 \nequivp (z_2 - z_1) + t_2$.
  \item[(D3)] If $t_1 = z' + t_2$ is implied by $\Psi$ then $t_1 \nequivp z + t_2$ for all $z \neq z'$.
  \item[(D4)] If $bl(t_1) \neq bl(t_2) \in \Psi$, then:
    \begin{enumerate}
      \item $t_1 \nequivp z + t_2$ for all $z \in \Z$.
      \item If $t_1 = z + t_2$ for any $z \in \Z$, then $\Psi$ is Unsat.
    \end{enumerate}
\end{enumerate}

The set of disequalities following from (D3) and (D4) can be infinite, but we don't need to explicitly store them, as they implicitly derive from the congruence closure of $\Psi_=$ or respectively from the propositions in $\Psi_{bl}$.
We construct the set of all disequalities and block disequalities deriving from (D0), (D1), (D2) and (D4),but only those that contain representative terms of components $Q\in\Pi$.
This set is finite, because we only store equalities between different representatives and not the implicit equalities following from (D3) and (D4).
We denote these explicitly stored disequalities as $closure_\neq(\Psi)$.

This set can be computed in polynomial time in the size of the formula $\Psi$ by going through all the
disequalities $t_1 \neq z + t_2$ in $\Psi_\neq$ and storing the disequalities
$\otau t_1  \neq (\omega t_2 - \omega t_1 + z) + \otau t_2$ and $\otau t_2 \neq - (\omega t_2 - \omega t_1 + z) + \otau t_1$.
Then for each disequality of the type $*(z_1 + t_1) \nequivp *(z_2 + t_2)$ that follows from a stored disequality or from the congruence closure of $\Psi_=$ or from a block disequality,
the disequalities propagate backwards using the rule (D2).
This only adds a finite amount of disequalities to the set, given that each equality and each block disequality implies at most one disequality of the type $*(z_1 + t_1) \nequivp *(z_2 + t_2)$.

For each block disequality $bl(t_1) \neq bl(t_2) \in \Psi$, we store the disequalities $bl(\otau t_1) \neq bl(\otau t_2)$.
We call the set of these disequalities $closure_{bl}(\Psi)$.

\begin{lemma}\label{lemma:diseq_types}
  $t_1 \nequivp t_2 + z$ iff one of the following holds:

  \begin{enumerate}
    \item $\otau t_1 \neq (\omega t_1 - \omega t_2 + z) + \otau t_2 \in closure_\neq(\Psi)$.
    \item Either $bl(\otau t_1) \neq bl(\otau t_2) \in closure_{bl}(\Psi)$ or $bl(\otau t_2) \neq bl(\otau t_1) \in closure_{bl}(\Psi)$.
    \item $t_1 = z' + t_2$ for any $z' \neq z$ is implied by $\Psi$.
  \end{enumerate}
  We call the disequalities that follow from 1.\ the \emph{explicit} disequalities and those following from 2.\ and 3.\ the \emph{implicit} disequalities.
\end{lemma}

