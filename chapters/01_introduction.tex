% !TeX root = ../main.tex
% Add the above to each chapter to make compiling the PDF easier in some editors.

\chapter{Introduction}\label{chapter:introduction}

The verification of software is crucial in safety-critical systems, in order to ensure the absence of bugs that could lead to catastrophic failures.
Many safety critical programs are written in C, a language that is known for its low-level features and its lack of safety guarantees.
Especially complex pointer manipulations could lead to bugs which are hard to detect.
Automatic verification tools can help to find these bugs by using static analysis of programs,
i.e, formally analyzing the behavior of the programs without executing them.
% TODO: more concrete examples, like in the c2po paper

Existing methods for static analysis of pointers handle different aspects of the analysis.
A common approach is to infer for each variable in the program the set of possible address values it can take.
There exist fast, but relatively imprecise methods, as the one by \textcite{Steensgaard} and the one by \textcite{Andersen}, but also more precise approaches that compute for each program point and each variable the set of possible addresses it can point to, as the ones implemented, e.g., in \textsc{Mopsa} \cite{mopsa} and \goblint \cite{goblint}.
On the other hand, there are extremely precise analyses, which can precisely analyze the shape of dynamic data structures in the heap,
with the drawback of being less efficient and thus not applicable to large programs
\cite{rivalpapers,kreiker,predator}.

As a compromise between precision and efficiency,
weakly relational analyses have been considered \cite{octagon,SeidlETS2023}.
Relational analyses track relations between variables, such as equalities and disequalities.
They are only \emph{weakly} relation as they only considers relations between pairs of variables and not an arbitrary number of variables.
Thus, the advantage of these analyses is its simplicity, which allows for a polynomial-time implementation.

This master's thesis describes the design and implementation in \goblint\ of a weakly-relational pointer analysis,
which we called \cpo, as it is an analysis on C programs and it is based on 2-Pointer Logic \cite{2pointer}.
\cpo\ infers for each program point a set of disequalities that always hold at specific program points,
for any possible execution of the program.
The inferred propositions are expressed in the 2-Pointer Logic introduced by \textcite{2pointer}, which
are equalities or disequalities built from pairs of terms that can
include variable names, dereferencing, and pointer arithmetic.
An example of such a property is \todo{example program}
\begin{equation}\label{example:introduction}
    (x = \&y + 3) \land (*(\&z + 2) \neq \&y + 1),
\end{equation}
where $*$ is the dereferencing operator and $\&$ is the address-of operator.

Additionally to the propositions introduced by \textcite{2pointer}, so-called \emph{block disequalities}
are introduced, which express that two pointers point to different memory objects, that we call \emph{blocks}.
Memory objects are for example distinct variables or distinct memory blocks allocated by \malloc.
For two variables $x$ and $y$, we know that their addresses $\&x$ and $\&y$ belong to different memory block, thus we can infer that $bl(\&x) \neq bl(\&y)$, where $bl(\cdot)$ represents the memory block of a pointer.
For two variables $x$ and $z$ that are initialized with different calls to \malloc, we can infer that $bl(\&x) \neq bl(\&z)$.

Analyzing diseuqualities in addition to equalities is crucial in order to have a precise
analysis during an assigment.
Given that assigning to a dereferenced pointer $*p$ overwrites a value in memory, after the assignment
all values that are written at the same address as $p$ have changed.
Therefore, all propositions that contain any value that may be equal to $p$ are no longer valid.
The equalities inferred by \cpo\ are must-equalities that always hold at a specific program point,
therefore there is no way to know if two pointers \emph{may} point to the same address.
The disequalities allow us to know which pointer \emph{may not} point to the same address,
thus allowing us to keep more information after an assignment.
Additional disequalities are inferred from the block disequalities and also by
exploiting information that derives from other non-relational pointer analyses.
This is easily possible, as we implemented the \cpo\ analysis in \goblint, which already includes a non-relational pointer analysis.

\section{Thesis Structure}

This master's thesis is structured as follows:
\begin{itemize}
    \item \cref{chapter:background} gives an overview on the abstract interpretation method, which builds the theoretical foundation on which the \cpo\ analysis is designed.
    It also includes a brief description of the abstract interpreter \goblint\, where the \cpo\ analysis is implemented, and other analyses that collaborate with \cpo\ to augment the precision of the analysis.
    \item \cref{chapter:concrete_semantics} introduces a concrete semantics, which models C pointers and the content of the memory.
    \item \cref{chapter:2pointer} recalls the 2-Pointer Logic introduced in~\cite{2pointer} and extends it with block disequalities.
    A representation of the conjunctions is introduced, which automatically discovers the \emph{closure} of the conjunction,
    i.e., the set of all propositions that are implied by the conjunction.
    For example the proposition $*(\&z + 2) \neq x - 2$ is implied by the conjunction~\labelcref{example:introduction}.
    This includes the \emph{quantitative congruence closure} of all the equalities that are present in the conjunction and
    the closure of the disequalities, using the rule that when $*t_1 \neq *t_2$, this implies that the pointers are also different, i.e., $t_1 \neq t_2$.
    \item \cref{chapter:analysis} describes the abstract domain used in the analysis, and the corresponding operations that are needed to implement an abstract interpretation analysis, including the \emph{equals}, \emph{meet}, \emph{join}, \emph{widening} and \emph{narrowing} operations.
    Two possible algorithms of the \emph{equals} and the \emph{join} operation are presented, which differ in
    their precision and efficiency.
    Transfer functions are described, which define the behavior of the analysis during an assignment and a function call in the analyzed C code.
    \item \cref{chapter:evaluation} evaluates the implementation of the \cpo\ analysis in \goblint\ on a set of benchmarks.,
    comparing the analysis with the existing non-relational pointer analysis in \goblint, and also
    evaluating the differences between the two algorithms for the \emph{join} operation and the two possibilities for the \emph{equals} operation.
    \item \cref{chapter:related_work} gives an overview of the related work in the field of congruence closure, pointer analysis and weakly relational analyses.
    \item \cref{chapter:conclusion} concludes the thesis and gives an outlook on future work.
\end{itemize}

The implementation of \cpo\ is available under \todo{TODO}. Parts of this thesis have been submitted to \dots.
