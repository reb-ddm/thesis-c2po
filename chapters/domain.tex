\chapter{Operations on the Abstract Domain}

In this chapter we present the representation of the abstract domain used for the \cpo\ analysis, alongside the operations that are fundamental for implementing an analysis based on abstract interpretation.
This includes a partial order and the operations \emph{meet}, \emph{join}, \emph{widening} and \emph{narrowing}.
Two different versions of the \emph{join} and of the \emph{equal} operation are introduced,
differing in precision and efficiency.
Prior to defining these operations, we discuss the extension of the domain to include additional terms and the restriction of the domain to a subset of terms,
both crucial for the subsequent operations.


\section{Concrete Semantics}

We model the memory as a two-dimensional address space $\ZZ$,
where the first element of the address is the \emph{block identifier} and the second element is the \emph{offset} within the block.
Each call to \malloc\ returns an address with a fresh block identifier, and each variable is written in memory in a unique memory block,
which is distinct from the block of any other variable.

The semantics can be described by the following three functions:
\begin{itemize}
\item $\rho : \X \rightarrow \ZZ$ assigns an address to each variable,
\item $\nu : \mathcal{A} \rightarrow \ZZ$ assigns an address to each auxiliary, and
\item $\mu : \ZZ \rightarrow \ZZ$ assigns a value to each address.
\end{itemize}
We are only interested in analyzing the values of pointers and addresses, therefore we interpret each value stored in the memory and in the auxiliaries as an address.

We also define the operator $(+) : \Z \rightarrow \ZZ \rightarrow \ZZ$ on addresses, where $z + (a,b) = (a, z+b)$.
Here, $z$ is added to the address offset, thus modeling the fact that it's not possible to leave a memory block using address arithmetic.
Moreover, we define the function $bl : \ZZ \rightarrow \Z$ that returns the address block identifier of an address, where $bl(a,b) = a$.

A term $t$ is interpreted as the value $\sem{t}(\rho, \nu, \mu)$ defined by:
\[
  \begin{array}{lll}
    \sem{\& x}\,(\rho,\nu,\mu)   & = & \rho\,x                          \\
    \sem{A}\,(\rho,\nu,\mu)      & = & \nu\,A                           \\
    \sem{*(z+t)}\,(\rho,\nu,\mu) & = & \mu\,(z+\sem{t}\,(\rho,\nu,\mu)) \\
  \end{array}
\]

The validity is defined as:
\[
  \begin{array}{ll}
    (\rho,\nu,\mu)\models t_1 = z+t_2          & \textbf{iff}
    \\ \multicolumn{2}{c}{\quad\quad\quad\quad\sem{t_1}\,(\rho,\nu,\mu) = z+\sem{t_2}\,(\rho,\nu,\mu)}	\\
    (\rho,\nu,\mu)\models t_1 \neq z+t_2       & \textbf{iff}
    \\ \multicolumn{2}{c}{\quad\quad\quad\quad\sem{t_1}\,(\rho,\nu,\mu) \neq z+\sem{t_2}\,(\rho,\nu,\mu)}	\\
    (\rho,\nu,\mu)\models bl(t_1) \neq bl(t_2) & \textbf{iff}
    \\ \multicolumn{2}{c}{\quad\quad\quad\quad bl(\sem{t_1}\,(\rho, \nu, \mu)) \neq bl(\sem{t_2}\,(\rho, \nu, \mu))}
  \end{array}
\]

If $(\rho, \nu, \mu)\models p$ for each proposition $p$ in $\Psi$, then we say that $(\rho, \nu, \mu) \models \Psi$.
\todo[inline]{The domain is not this any more, it's the kernels.}
The domain of 2-Pointer Logic $\mathcal{P}_2[=]$ consists of all finite conjunctions
of propositions over terms up to semantic equivalence.
The concretization $\gamma(\Psi)$ of $\Psi$ is the set of all $(\rho, \nu, \mu)$ with $(\rho, \nu, \mu) \models \Psi$.

\section{Abstract Domain}
The abstract values of the \cpo\ analysis consist of a representation
of 2-Pointer Logic conjunctions, as detailed in the previous chapter.
This ensures that whenever propositions are added to or removed from the domain,
the closure of all implied propositions is computed.

The representation consists of a tuple $k = \angl{P,M_P,B,D}$,
where $P = (\T, \tau, \omega)$ is a quantitative partition, as described in \cref{subsection:quantitative-union-find}.
Each state of the QFA $M$ represents an equivalence class of $P$, as outlined in \cref{subsection:qfa}.
$B$ and $D$ are lists of explicit block disequalities and disequalities, respectively, as detailed in \cref{section:block-disequalities,section:disequalities}.

The tuple $k$ is called the \emph{kernel} representation of $\Psi$.
For an unsatisfiable formula $\Psi$, the kernel representation is $\bot$.

It is posible to convert a kernel representation $k = \angl{P,M_P,B,D} \neq \bot$ with $M_P = (S, \otau, \eta, \delta)$ to a formula $\F[k]$, that is given by

\[
    \begin{array}{lll}
        \F[k] & \equiv & \bigwedge_{\eta\,a=(z,s)} (a = z + \otau\,s) \land                        \\
              &        & \bigwedge_{\delta\,z\,s=(z',s')} (*(z + \otau\,s) = z' + \otau\,s') \land \\
              &        & \bigwedge_{\{t, t'\} \in B} (bl(t) \neq bl(t')) \land                     \\
              &        & \bigwedge_{(t, z, t') \in D} (t \neq z + t')
    \end{array}
\]

The trivial propositions of the form $t = 0 + t$ are removed from the formula, as well as the repeated equalities.

If $k$ is the kernel of a conjunction $\Psi$, then $\F[k]$ is equivalent to $\Psi$,
even though it may not be syntactically identical.
They could be syntactically different because of a different choice of representative terms.

\section{Extension to a Superset of Terms}

When new terms appear during the analysis, the partition $P$ and the automaton $M_P$ need to be updated to express properties about the new terms.
It is not sufficient to simply add a fresh equivalence class for the new terms, as they sometimes need to be added to existing equivalence classes.

\begin{example}
    Let $\Psi \equiv A = B$ and $\T = \{A, B, *A\}$.
    The equivalence classes of $P[\Psi]$ are $\{A, B\}$ and $\{*A\}$.
    If we want to add the term $*B$ to $\T$, we need to add it to the same equivalence class as $*A$, as the two terms are equivalent according to the rules defined in \cref{chapter:qcc}.
    \todo{The same example is already somewhere else.}
\end{example}

Let $P = (\T, \tau, \omega)$ be a partition with the corresponding QFA $M_P = (S, \otau, \eta, \delta)$ and let $\T'$ be a superset of $\T$ and closed under subterms.
We define the operation $\ext{k}{\T'}$ that extends the kernel $k = \angl{P, M_P, B, D}$ to the set of terms $\T'$, without altering the semantics of the representation.
Only $P$ and $M_P$ need to be modified, while $B$ and $D$ remain unchanged.

The extension is built inductively based on the structure of the terms in $\T'$.
We first describe how to add a new atom and then consider terms of the form $*(z+t)$, where $t$ is already a part of $\T$.
This is repeated for all terms in $\T'$.

In the case that $\T' = \T \cup \{a\}$ for an atom $a \notin \T$, we simply add a new equivalence class with representative $a$ to $P$, by including a new state $s$ to the states $S$ of $M_P$ and defining $\tau\,a = a$, $\omega\,a=0$, $\eta(a) = (0,s)$, $\otau\,s = a$.

If $\T' = \T \cup \{*(z+t)\}$ for a term $t \in \T$, but $*(z + t) \notin \T$, we add a new equivalence class $\{*(z+t)\}$ to $P$ and then determine if it needs to be merged with an existing equivalence class.
We define $\tau\,(*(z+t)) = *(z+t)$ and $\omega\,(*(z+t))=0$.
Let the resulting partition be $P'$ and let $s \in S$ be the state of the subterm $t$, i.e., for which it holds that $\otau\,s = \tau\,t$.
If $\delta(z + \omega\,t,s)$ is not defined, then we add a new state $s'$ to $S$ and define $\delta(z + \omega\,t, s) = (0, s')$ and $\otau\,s' = *(z+t)$.
Otherwise, let $\delta(z + \omega\,t, s) = (z', s')$.
Then we perform the operation $\closure{*(z + t)}{z'}{\otau\,s'}{P'}$.
In this case, we do not need to add a new state to the automaton,
as the new term is added to the existing equivalence class of $s'$.

\section{Restriction}


For the assignment and the widen operation, it will be useful to be able to forget all information about specific sets of terms.
For example, when we assign a value to a term, we need to remove all terms from the domain that may be modified by this assignment, i.e., all the terms that may alias with this term.
In the following, we will describe how to find the set of terms that may alias with a term and how to restrict the automaton and the disequalities to only keep the propositions that contain terms from a specific set.

Given a partition $P = (\Pi, \tau, \omega)$, a QFA $M = (S, \eta, \delta)$, and a term $t$, we want to compute the partition $\restr{P}{\neg t}$ and the QFA $\restr{M}{\neg t}$ that represents all equalities that follow from $M$ and that are still valid after overwriting the value of $t$.

Let $\T = \L(M)$. We construct the subterm-closed set $\restr{\T}{\neg t}$ that contains all terms $t' \in \T$ such that each subterm of $t'$ definitely does not alias with $t$.
This is exactly the set of terms that are not modified when the value of $t$ changes.
We differentiate between two cases: the case where $t$ is an atom and the case where $t$ is a dereferenced term.

If $t$ is an atom, then it is impossible to reach the address of $t$ by dereferencing.
Therefore, $\restr{\T}{\neg t}$ is the set of all terms that do not contain $t$ as a subterm.

If $t \equiv *(z + t')$ is a dereferenced term, then $\restr{\T}{\neg t}$ contains all the terms of $\T$ where for each subterm of the form $*(z' + v)$ it holds that $t' \nequivp (z' - z) + v$.

Given the set $\restr{\T}{\neg t}$, we can find the restricted automaton $\restr{M}{\neg t} = (S', \eta', \delta')$ by conducting a breadth-first search on the automaton $M$ and computing which paths of the automaton are still reachable by using only the terms in $\restr{\T}{\neg t}$.
In order to correctly update the weights in the automaton and later find representative terms of the equivalence classes that derive from the automaton, we will need to choose a representative term for each state of the automaton, which we will denote as the function $\repr : S' \rightarrow \Z \times \restr{\T}{\neg t}$.

We start with the initial states of $M$:
for each atom $a \in \restr{\T}{\neg t}$ where $\eta(a) = (z, s)$, if we haven't already visited $s$, then we need to choose a new representative for $s$. We simply define $\repr s = (z, a)$ and $\eta'(a) = (0, s)$.
If we have already visited $s$, then let $\repr s = (z', t)$. We define $\eta'(a) = (z - z', s)$.

Then for each visited state $s_1$, we consider all outgoing transitions: for each $\delta(z_1, s_1) = (z_2, s_2)$,
we define $\repr s_2$ if we haven't visited $s_2$ yet.
For this, we find a term $*(z+t)$ such that $M[t] = (z_1', s_1)$ and $z = z_1 - z_1'$ for some $z_1' \in \Z$.
If we manage to find such a term, we define $\repr s_2 = *(z+t)$.
Otherwise, this transition is not valid in $\restr{M}{\neg t}$ anymore, and we don't add $s_2$ to the set of visited states.
Now let $\repr s_1 = (z_1', t_1)$ and $\repr s_2 = (z_2', t_2)$.
We can define the corresponding transition $\delta'(z_1 - z_1', s_1) = (z_2 - z_2', s_2)$.
We add $s_2$ to the set of reachable states, and we continue the search for all outgoing transitions for the remaining visited states.

Essentially, we keep the transitions between the states that are still reachable after removing the terms, and we adjust the weights of the transitions to the new representatives.

The new partition $\restr{P}{\neg t} = (\Pi', \tau', \omega')$ is defined as $\Pi' = \{\L(s) \mid s \in S, s \text{ is reachable in $\restr{M}{\neg t}$}\}$,
$\tau' (\L(s)) = t$ if $\repr s = (z, t)$, and
for each $t \in \L(\restr{M}{\neg t})$ we define $\omega'(t) = z'$ if $\restr{M}{\neg t}[t] = (z', t')$.

This definition of restriction can be applied in the same way by using any subterm-closed set $\T' \subseteq \L(M)$ of terms and restricting the automaton $M$ to the automaton $\restr{M}{\T'}$ that only remembers the equalities that only contain terms from $\T'$.

\begin{lemma}\label{restriction}
    Let $M = (S, \eta, \delta)$ be a QFA and let $\restr{M}{\T'} = (S', \eta', \delta')$ and $\repr : S' \rightarrow \Z \times \T'$ be the automaton and the function that are obtained by restricting $M$ as described above.
    Then for each term $t_1 \in \T'$, it holds that $M[t_1] = (z, s)$ iff $\restr{M}{\T'}[t_1] = (z - \repr s, s)$.
\end{lemma}
\begin{proof}
    This lemma can be proven by induction over the structure of the term $t_1$.
\end{proof}

\begin{proposition}
    Let $M = (S, \eta, \delta)$ be a QFA and let $\restr{M}{\T'} = (S', \eta', \delta')$ be the automaton that is obtained by restricting $M$ as described above.
    Then for each term $t_1, t_2 \in \T'$, it holds that $M[t_1] = z + M[t_2]$ iff $\restr{M}{\T'}[t_1] = z + \restr{M}{\T'}[t_2]$.
\end{proposition}
\begin{proof}
    This proposition follows from Lemma~\ref{restriction}.
\end{proof}

We can also find the set of disequalities and the block disequalities that are still valid when considering only the terms $\T'$.
We keep only the propositions between representatives of the equivalence classes that are still reachable in the automaton $\restr{M}{\T'}$, but we need to update the representatives to the new representatives.
The remaining disequalities are removed.

Given a conjunction $\Psi$, we denote the restriction of the automaton and disequalities derived from $\Psi$ to the set of terms $\T'$ by $\restr{\Psi}{\T'}$.
Correspondingly, $\restr{\Psi}{\neg t}$ is the restriction of $\Psi$ to the terms of $\restr{\T}{\neg t}$, where $\T = \L(M)$ and $M$ is the automaton that derives from $\Psi$.

\section{Equality}

Given two kernels $k_1$ and $k_2$, we want to decide whether they are semantically equivalent.
In this section, two methods for computing the \emph{equal} operation are presented.
Their main difference is in how the equivalence of two partitions $P_1$ and $P_2$ is decided.
The first method is based on comparing the equivalence classes of the partitions $P_1$ and $P_2$.
The second method is based on computing a normal form of the conjunctions represented by the kernels $k_1$ and $k_2$.
It then suffices to compare the normal forms syntactically. This has the advantage that the normal form can be computed once
for every kernel and there is no need to recompute it for every comparison.

\subsection{Comparing Equivalence Classes}

Let $P_1 = (\T_1, \tau_1, \omega_1)$ and $P_2 = (\T_2, \tau_2, \omega_2)$ be two partitions.
First, we extend both partitions to the set $\T = \T_1 \cup \T_2$.
Then we compare the equivalence classes of the resulting partitions $P_1' = (\T, \tau_1', \omega_1')$ and $P_2' = (\T, \tau_2', \omega_2')$ of $\ext{P_1}{\T}$ and $\ext{P_2}{\T}$.

For this, we need to check for each element $t \in \T$, if $t = \omega_1'\,t + \tau_1'\,t$ also holds in $P_2$, i.e.,
if $\omega_2'\,t + \tau_2'\,t = \omega_1'\,t + \omega_2'\,(\tau_1'\,t) + \tau_2'\,(\tau_1'\,t)$.

Afterwards, the disequalities and block disequalities are compared.
For two kernels $k_1 = \angl{P_1, M_{P_1}, B_1, D_1}$ and $k_2 = \angl{P_2, M_{P_2}, B_2, D_2}$,
we rewrite the disequalities $B_2$ and $D_2$ to be about the representatives of $P_1$.
I.e., for each disequality $t_1 \neq z + t_2$ in $D_2$, we convert it to the disequality $\tau_1'\,t_1 \neq (z + \omega_1'\,t_2 - \omega_1'\,t_1) + \tau_1'\,t_2$
and for each block disequality $bl(t_1) \neq bl(t_2)$ in $B_2$, we convert it to the disequality $bl(\tau_1'\,t_1) \neq bl(\tau_1'\,t_2)$.
This is possible, because we previously extended the partitions to the same set $\T$,
i.e., $\tau_1'\,t$ is defined for each term $t$ occurring in $B_2$ and $D_2$.
Then we check if the resulting set of block disequalities is equal to $B_1$ and if the resulting set of disequalities is equal to $D_1$.

\subsection{Compute Normal Form}

In~\cite{2pointer} a different approach is presented to decide the equivalence of two partitions.
This is done by computing a normal form $\nf(\Psi)$ of a conjunction $\Psi$, such that $\Psi$ is semantically equivalent to $\nf(\Psi)$ and such that two conjunctions $\Psi_1$ and $\Psi_2$ are semantically equivalent iff $\nf(\Psi_1)$ and $\nf(\Psi_2)$ are syntactically equivalent.

The main idea is to find \emph{minimal representatives} for each state $s$ in the QFA $M_P$,
which corresponds to the smallest term in $\L_{M_P}(s)$, according to the order defined in \cref{chapter:2pointer}.
Then the automaton is transformed to a formula that utilizes only the minimal representatives, thus obtaining a normal form
that is independent on the chosen set $\T$ and on the chosen representatives.

Let $M_P = (S, \otau, \eta, \delta)$ be a QFA.\@
For each state $s \in S$, we compute the minimal term $m_s$ and the corresponding offset $z_s$ such that $M[m_s] = (z_s,s)$.
This can be computed by using a variantion of Dijktra's shortest path algorithm,
that will be described in the following.
We use an initially empty FIFO queue $P$ to store all states $s$ for which $(m_s,z_s)$ is already computed,
but where the outgoing edges still have to be processed.
First, we consider all atoms $a$ for which $\eta$ is defined, in ascending order. For each $\eta\,a = (z,s)$ we do:
\begin{enumerate}
    \item If $(m_s,z_s)$ is not defined yet, set $(m_s,z_s) = (a,z)$.
    \item Add $s$ to $P$.
\end{enumerate}
Then we process the queue $P$ in a FIFO manner, until it is empty.
For each state $s$ in $P$, we consider all outgoing edges $\delta\,z\,s = (z',s')$ in order of ascending $z$.
For each such edge, we do:
\begin{enumerate}
    \item If $(m_{s'},z_{s'})$ is not defined yet, set $(m_{s'},z_{s'}) = (*((z-z_s)+m_s),z')$.
    \item Add $s'$ to $P$.
\end{enumerate}
The algorithm is correct because longer paths always result in larger terms than shorter paths.
Thus, we do not need to update a pair once it was computed.

\begin{example}\label{ex:min-repr}
    Let $\Psi \equiv B = -1 + \&x \land *B = 2 + A \land *A = 3 + x$.
    The automaton $M_P$\todo{what is P} has can be visualized with the following graph:
    \begin{center}
        \begin{tikzpicture}[shorten >=1pt, node distance=3cm, on grid, auto, state/.style={rectangle, rounded corners, draw, inner sep=5pt, align=center}]

            \node[state] (Q1) {$\begin{array}{c}
                        \boldsymbol{\&x}, \\
                        1 + B
                    \end{array}$};

            \node[state, right=4.5cm of Q1] (Q2) {$\begin{array}{c}
                        \boldsymbol{*B}, 2 + A, \\
                        *(-1+\&x )
                    \end{array}$};

            \node[state, right=5.6cm of Q2] (Q3) {$\begin{array}{c}
                        \boldsymbol{x}, *(1 + B), -3 + *A, \\
                        -3 + *(-2+*B),...
                    \end{array}$};

            \path[->]
            (Q1) edge[] node {$-1,0$} (Q2)
            (Q2) edge[] node {$-2,3$} (Q3)
            (Q1) edge[bend right] node {$0,0$} (Q3);

            \node [above left=1cm and 0.3cm of Q1] {$s_0$};
            \node [above left=1cm and 0.5cm of Q2] {$s_1$};
            \node [above left=1cm and 0.7cm of Q3] {$s_2$};

        \end{tikzpicture}
    \end{center}
    The minimal representatives are computed by first considering the atoms in ascending order.
    Assuming that the linear order on atoms is $\&x < A < B$,
    we define first $m_{s_0} = \&x$ and $z_{s_0} = 0$.
    Then we define $m_{s_1} = A$ and $z_{s_1} = -2$.
    There is nothing to do for $B$, because we already defined $m_{s_0}$.
    The queue $P$ is now $\{s_0, s_1\}$.
    We consider the outgoing edge of $s_0$ with $z = -1$.
    There is nothing to do for $s_1$, as $m_{s_1}$ is already defined.
    Then we consider the outgoing edge of $s_0$ with $z = 0$
    and set $m_{s_2} = x$ and $z_{s_2} = 0$.
    Then we are done.
\end{example}
Using the minimal representatives, we can now transform the kernel $k = \angl{P,M_P,B,D} \neq \bot$ with $M_P = (S, \otau, \eta, \delta)$ to a normal formula $\F_{normal}[k]$,
defined as follows:

\[
    \begin{array}{lll}
        \F_{normal}[k] & \equiv & \bigwedge_{\eta\,a=(z,s)} (a = (z - z_s) + m_s) \land                                                           \\
                         &        & \bigwedge_{\delta\,z\,s=(z',s')} (*((z-z_s) + m_s) = (z'-z_{s'} + m_{s'}) \land                                 \\
                         &        & \bigwedge_{s,s' \in S \land \{\otau\,s, \otau\,s'\} \in B} (bl(m_{s}) \neq bl(m_{s'})) \land           \\
                         &        & \bigwedge_{s,s' \in S \land (\otau\,s, z,\otau\,s') \in D} (m_{s} \neq (z+ z_{s} -z_{s'}) + m_{s'})
    \end{array}
\]

The trivial propositions of the form $t = 0 + t$ are removed from the formula, as well as the repeated equalities.

This corresponds to the formula representation $\F[k]$, but it expresses the proposition using the minimal representatives instead of the representatives of the union-find.
The advantage of this normal form is that the formula representation does not depend on the chosen representatives and also not of the chosen set $\T$.
Thus, there is a unique normal form for each semantic equivalence class of kernels,
even though there exists multiple different kernel representations of the same conjunction.
Therefore, in order to decide the equivalence of two kernels $k_1$ and $k_2$, it suffices to compare the normal forms $\F_{normal}[k_1]$ and $\F_{normal}[k_2]$ syntactically.

\begin{example}
    Consider the conjunction $\Psi$ from \cref{ex:min-repr}.
    The normal form of the kernel $k$ representing $\Psi$ is
    \[
    (B = -1+\&x) \land (*(-1 + \&x) = 2 + A) \land (*A = 3 + x)
    \]
    where $B = -1 + \&x$ originates from $\eta\,B = (-1,s_0)$.
    The equalities originating from $\eta\,\&x = (0,s_0)$ and $\eta\,A = (0, s_1)$ are removed, because they are trivial.
    The edge $\delta\,(-1,s_0) = (0,s_0)$ gives the equality $*(-1 + \&x) = 2 + A$ and
    $\delta\,(-2,s_1) = (3,s_2)$ gives the equality $*A = 3 + x$.
    The last edge $\delta\,(0,s_0) = (0,s_2)$ gives the trivial equality $x = 0 + x$.
\end{example}

\todo[inline]{Correctness proof}

\todo[inline]{Proof that they are equivalent}

\todo[inline]{Implementation detail: it is lazily computed.}
\subsection{Partial Order}

The natural partial ordering between conjunctions is semantic implication, i.e.,
$\Psi_1 \rightarrow \Psi_2$
iff $(\rho, \nu, \mu) \models \Psi_1$ whenever $(\rho, \nu, \mu) \models \Psi_2$.
This is the case iff $\Psi_1 \land \Psi_2$ is semantically equal to $\Psi_1$.
Therefore it is possible to reduce the \emph{less equal} operation to a \emph{meet} and an \emph{equal} operation.
A kernel $k_1$ that represents a conjunction $\Psi_1$ is \emph{less or equal} to a kernel $k_2$ representing $\Psi_2$ iff
$k_1 \meet k_2$ is semantically equal to $k_1$.
We already described how to decide semantic equality between two kernels.
In the next section, the \emph{meet} operation ($\meet$) is presented.

\section{Meet}\label{section:meet}

Semantically, the \emph{meet} of two conjunctions consists in taking the conjunction of the conjunctions:
$\Psi_1 \meet \Psi_2 \equiv \Psi_1 \land \Psi_2$.

Given two kernels $k_1$ of $\Psi_1$ and $k_2$ of $\Psi_2$,
we want to compute the kernel $k_1 \meet k_2$ that represents the conjunction $\Psi_1 \land \Psi_2$.
If either $k_1 = \bot$ or $k_2 = \bot$, then $k_1 \meet k_2 = \bot$.
Otherwise, we consider the formula representation $\F[k_2]$ of $k_2$ and add each proposition of $\F[k_2]$ to the kernel $k_1$.
First, all quantitative equalities are added, then all block disequalities, and finally, all disequalities.
Let $k = \angl{P,M,B,D}$ and $P = (\T, \tau, \omega)$ and $M = (S, \otau, \eta, \delta)$.
Assuming that all the terms occurring in $p$ have already been added to $\T$ by extending the partition $P$,
the conjunction $p \meet k$ of a kernel $k$ and a proposition $p$ is defined as follows:
\begin{itemize}
    \item Let $p \equiv (t_1 = z + t_2)$ and let $t_1'=\tau\,t_1$ and $t_2'=\tau\,t_2$.
          \begin{itemize}
              \item If $t_1' \equiv t_2'$ and $\omega\,t_1 = z + \omega\,t_2$,
                    then the equality is already implied by $P$ and we set $p \meet k = k$.
              \item If $t_1' \equiv t_2'$, but $\omega\,t_1 \neq z + \omega\,t_2$,
                    or if $\{t_1',t_2'\}\in B$, or if $t_1' \neq (z + \omega\,t_2 - \omega\,t_1) + t_2'$ is in $D$,
                    then there is a contradiction and we set $p \meet k = \bot$.
              \item Otherwise, we compute the $\closure{t_1}{z}{t_2}{P}$.
                    Let $P' = (\T, \tau', \omega')$ and $M'$ be the partition and QFA resulting from the closure operation.
                    Now, we need to update the block and quantitative disequalities to be about the representatives of $P'$.
                    For each block disequality $\{t_1,t_2\}$ in $B$, we convert it to the disequality $bl(\tau'\,t_1) \neq bl(\tau'\,t_2)$.
                    $B'$ is the result of updating all terms of $B$ to the new representatives:
                    \[
                        B' = \{\{\tau'\,t_1, \tau'\,t_2\} \mid \{t_1,t_2\} \in B\}.
                    \]
                    We set $D' = D[\F[k_1] \land p]$.
                    If a contradiction occurs during the construction of $D'$, we set $p \meet k = \bot$.
                    We set $p \meet k = \angl{P',M',B',D'}$.
          \end{itemize}
    \item Let $p \equiv (bl(t_1) \neq bl(t_2))$. Again, let $t_1'=\tau\,t_1$ and $t_2'=\tau\,t_2$.
          \begin{itemize}
              \item If $t_1' \equiv t_2'$, then $p \meet k = \bot$.
                    Otherwise, let $t_1' \nequiv t_2'$. We define $p \meet k = \angl{P,M,B', D'}$, where $B'$ and $D'$ are described in the following:
              \item If $t_1$' and $t_2'$ are address expressions, then $bl(t_1') \neq bl(t_2')$ is trivially true.
                    Therefore, we set $B'= B$ and $D'=D$.
              \item Otherwise, we add the disequality $\{t_1',t_2'\}$ to $B$.
                    We set $D' = D[\F[k_1] \land p]$.
          \end{itemize}
    \item Let $p \equiv (t_1 \neq z + t_2)$, $t_1'=\tau\,t_1$ and $t_2'=\tau\,t_2$.
          \begin{itemize}
              \item If $t_1'$ and $t_2'$ are both address expressions, then the disequality is trivially true.
                    The same holds if $t_1' \equiv t_2'$ and $\omega\,t_1 = z + \omega\,t_2$
                    or if $\{t_1',t_2'\}\in B$. In these cases, we set $p \meet k = k$.
              \item If $t_1' \equiv t_2'$, but $\omega\,t_1 \neq z + \omega\,t_2$, then
                    there is a contradiction and we set $p \meet k = \bot$.
              \item We set $p \meet k = \angl{P,M,B,D'}$, where $D' = D[\F[k_1] \land p]$.
          \end{itemize}
\end{itemize}

As an optimization, the quantitative disequalities can be computed only at the end, and not after each added proposition, by setting $D' = D[\F[k_1] \land \F[k_2]]$

\todo[inline]{some kind of proof?}


\section{Join}

Given two conjunctions $\Psi_1$ and $\Psi_2$, the least upper bound is a conjunction implied by both $\Psi_1$ and $\Psi_2$ that is smallest with respect to the partial order.
Intuitively, the least upper bound is $\Psi_1 \lor \Psi_2$.
However, disjunctions are not expressible in the 2-Pointer Logic, so we are looking for the most precise conjunction that is implied by $\Psi_1 \lor \Psi_2$.
The conjunction of all propositions implied by a disjunction is not always representable as a conjunction of finite size~\cite{join,2pointer}. Therefore, it is impossible to compute the exact least upper bound of two conjunctions.


\begin{example}\label{example:infinite-join}
    Adapting the original example by \textcite{join} to \cpo, consider the conjunctions $\Psi_1 \equiv (A = V)$ and $\Psi_2 \equiv (*A = A)\land (*V=V) \land (*(1 + A) = *(1 + V))$, represented by the automata $M_1$ and $M_2$ shown in \cref{qfa-infinite-join}.
    \begin{figure}
\centering
        \begin{minipage}{0.3\textwidth}
            \centering
            \begin{tikzpicture}[shorten >=1pt, node distance=3cm, on grid, auto, state/.style={rectangle, rounded corners, draw, inner sep=5pt, align=center}]

                \node[state] (Q1) {$\begin{array}{c}
                            A,V
                        \end{array}$};
                \node [left=1.5cm of Q1] {$M_1$:};
            \end{tikzpicture}
        \end{minipage}
        \begin{minipage}{0.65\textwidth}
            \centering
            \begin{tikzpicture}[shorten >=1pt, node distance=3cm, on grid, auto, state/.style={rectangle, rounded corners, draw, inner sep=5pt, align=center}]

                \node[state] (Q1) {$\begin{array}{c}
                            A,*A,..
                        \end{array}$};

                \node[state, right=3cm of Q1] (Q2) {$\begin{array}{c}
                            V,*V,..
                        \end{array}$};

                \node[state, below right=2cm and 1.5cm of Q1] (Q3) {$\begin{array}{c}
                            *(1+A), *(1 + V),...
                        \end{array}$};

                \path[->]
                (Q1) edge[loop below] node[left] {$0,0$} (Q1)
                (Q2) edge[loop below] node[right] {$0,0$} (Q2)
                (Q1) edge[] node[right] {$1,0$} (Q3)
                (Q2) edge[] node[above left] {$1,0$} (Q3);

                \node [below left=1cm and 1.5cm of Q1] {$M_2$:};

            \end{tikzpicture}
        \end{minipage}
    \caption{QFAs for $\Psi_1 \equiv (A = V)$ and $\Psi_2 \equiv (*A = A)\land (*V=V) \land (*(1 + A) = *(1 + V))$}\label{qfa-infinite-join}
\end{figure}
    Then for each $n \geq 0$, the equality $*(1+*^n A) = *(1 + *^n V)$ is implied both by $\Psi_1$ and $\Psi_2$, where $*^m$ denotes $m$-fold dereference.
    It can be shown that the set of quantitative equalities implied both by $\Psi_1$ and $\Psi_2$ cannot be represented by a finite conjunction~\cite{join, 2pointer}.
\end{example}

This example shows that it is not possible to compute the exact least upper bound of two domain elements because of the limitations in the expressiveness of the congruence closure representation.
However, it is possible to define a sound over-approximation of the join operation.

Given that equalities only depend on the automaton and cannot follow from the disequalities, we first consider how to construct an automaton that represents the conjunction of two automata.
Later, we will consider the (block-)disequalities that follow from $\Psi_1$ and $\Psi_2$.

As we only compute an approximated join operation, multiple possibilities exist to define the join.
A compromise must be made between the precision of the join and the computational complexity.
Two different approaches will be presented in the following sections.
The first one computes the join based on the automata representation of the equalities, while the second only considers the partition $P$ of the terms $\T$.
This first approach is less efficient but more precise, as all terms of the set $\L(M)$ are considered, while in the second approach, only the terms of the set $\T$ are considered.
In particular, circular dependencies between terms are lost in the second approach.
However, these circular dependencies do not often occur in practice.
Therefore, we expect that the difference in precision is not significant.

\subsection{Join Using the Automaton}

We propose an implementation of the join algorithm of two quantitative finite automata.
It is based on the work of ~\textcite{join}, where they discuss join algorithms for congruence closure data structures.

Given two partitions $P_1 = (\T_1, \tau_1, \omega_1)$ and $P_2 = (\T_2, \tau_2, \omega_2)$ and the corresponding automata $M_1$ and $M_2$, we want to compute the partition $P_3$ and the automaton $M_3$ that represent the conjunction of the two automata.
The first step is to extend the set of terms of $P_1$ and $P_2$ to $\T = \T_1 \cup \T_2$.
Henceforth, we assume that both partitions are defined over the same set of terms $\T$,
i.e., $P_1 = (\T, \tau_1, \omega_1)$ and $P_2 = (\T, \tau_2, \omega_2)$.
Let $M_1$ and $M_2$ be the corresponding automata.

The join is defined as the product automaton of $M_1$ and $M_2$.

\begin{definition}
    Let $M_1 = (S_1, \otau_1, \eta_1, \delta_1)$ and $M_2 = (S_2, \otau_2, \eta_2, \delta_2)$ be two quantitative finite automata.
    The \emph{quantitative product automaton} $M_3 = (S_3, \otau_3, \eta_3, \delta_3)$ is defined as:
    \begin{itemize}
        \item If $\eta_1(a) = (z_1, s_1)$ and $\eta_2(a) = (z_2, s_2)$, then
              $\eta_3(a) = (z_1, \angl{s_1, s_2, z_2-z_1})$.
        \item If $\delta_1(z_1, s_1) = (z'_1, s'_1)$ and $\delta_2(z_2, s_2) = (z'_2, s'_2)$, then $\delta_3(z_1, \angl{s_1, s_2, z_2 - z_1}) =  (z'_1, \angl{s'_1, s'_2, z'_2 -z'_1})$.
        \item $S_3 \subseteq S_1 \times S_2 \times \Z$. $S_3$ contains all states that are reachable in $M_3$.
        \item For each $s \in S_3$, $\otau_3(s)$ is any term in $\L_{M_3}(s)$.
    \end{itemize}
\end{definition}

The product automaton is finite, as there are only a finite amount of reachable states in $S_1 \times S_2 \times \Z$.
The remaining states are unreachable, i.e., they neither occur in a return value of $\eta_3$ nor $\delta_3$.
This follows from the fact that $\eta_i$ and $\delta_i$ for $i = 1,2$ are defined only for a finite amount of values, as discussed in \cref{subsection:qfa}.

The product can be computed in polynomial time, as we need to add an initial state for all pairs of initial states of $M_1$ and $M_2$ and then add a transition for each pair of transitions of $M_1$ and $M_2$. Therefore, the runtime is equal to the size of $M_1$ times that of $M_2$.

\begin{lemma}\label{lemma:join-automata}
    Let $M_3$ be the product automaton of two quantitative automata $M_1$ and $M_2$.
    Then, for each term $t \in \L(M_1) \cap \L(M_2)$ it holds that $M_1[t] = (z_1, s_1)$ and $M_2[t] = (z_2, s_2)$ iff $M_3[t] = (z_1, \angl{s_1, s_2, z_2 - z_1})$.
\end{lemma}
\begin{proof}
    We prove it by induction over the structure of the term $t$.
\end{proof}

As shown in \cref{example:infinite-join}, it is not possible to define a join algorithm that is complete.
In fact, we can only show for our definition of the join algorithm that it is at least complete over the terms in $\L(M_1) \cap \L(M_2)$.
The following proposition shows that our definition of join is complete over the set of terms $\L(M_1) \cap \L(M_2)$ and that it is sound.

\begin{proposition}\label{prop:join-automata}
    Let $M_3$ be the product automaton of two quantitative automata $M_1$ and $M_2$.
    Then, for all terms $t_1, t_2 \in \L(M_1) \cap \L(M_2)$ it holds that $M_3[t_1] = z + M_3[t_2]$ iff $M_1[t_1] = z + M_1[t_2]$ and $M_2[t_1] = z + M_2[t_2]$.
\end{proposition}
\begin{proof}
    This proposition follows from Lemma~\ref{lemma:join-automata}.
\end{proof}

As we mentioned before, we added all terms of $\T = \T_1 \cup \T_2$ to both automata before computing the product automaton, and after having added these terms, it holds that $\T \subseteq \L(M_1) \cap \L(M_2)$.
Therefore, the join of two automata is complete at least over the terms in $\T$.

Using the automaton $M_3$, we can compute the corresponding partition $P_3$.
Let $\T' = \{t \mid \exists s \in S_3 . \otau_3(s) = t\}$ be the set of terms chosen as representatives for the states of $M_3$.
The partition $P_3$ is defined as $P_3 = (\T' \cup \T, \tau_3, \omega_3)$,
where for each $t \in \T' \cup \T$, $\tau_3\,t = \otau_3\,(s)$ and $\omega_3\,t = z$ if $M_3[t] = (z,s)$.

\begin{example}\label{join-example}
    We consider again the two automata of \cref{example:infinite-join}.
    First, we add the terms of the set $\T = \{A, V, *A,*V,*(1+A)$, $*(1 + V)\}$ to the automaton $M_1$, resulting in the automata $M_1'$.
    Let $M_3$ be the product automata of $M_1'$ and $M_2$.
    The automata $M_1'$ and $M_3$ are shown in \cref{m1-m3-qfa}
\begin{figure}
    \centering
        \begin{minipage}{0.45\textwidth}
            \centering
            \begin{tikzpicture}[shorten >=1pt, node distance=3cm, on grid, auto, state/.style={rectangle, rounded corners, draw, inner sep=5pt, align=center}]

                \node[state] (Q1) {$\begin{array}{c}
                            A, V
                        \end{array}$};
                \node[state, below=2cm of Q1] (Q2) {$\begin{array}{c}
                            *(1+A), *(1+V)
                        \end{array}$};
                \node[state, below left=1cm and 2cm of Q1] (Q3) {$\begin{array}{c}
                            *A
                        \end{array}$};
                \node[state, below right=1cm and 2cm of Q1] (Q4) {$\begin{array}{c}
                            *V
                        \end{array}$};
                \path[->]
                (Q1) edge node {$1,0$} (Q2)
                (Q1) edge node {$0,0$} (Q3)
                (Q1) edge node {$0,0$} (Q4);


                \node [below left=-0.5cm and 1.7cm of Q1] {$M_1'$:};
            \end{tikzpicture}
        \end{minipage}
        \begin{minipage}{0.45\textwidth}
            \centering
            \begin{tikzpicture}[shorten >=1pt, node distance=3cm, on grid, auto, state/.style={rectangle, rounded corners, draw, inner sep=5pt, align=center}]

                \node[state] (Q1) {$\begin{array}{c}
                            A
                        \end{array}$};

                \node[state, right=2cm of Q1] (Q2) {$\begin{array}{c}
                            V
                        \end{array}$};

                \node[state, below right=2cm and 1cm of Q1] (Q3) {$\begin{array}{c}
                            *(1+A), *(1 + V)
                        \end{array}$};
                \node[state, below left=1cm and 2cm of Q1] (Q4) {$\begin{array}{c}
                            *A
                        \end{array}$};
                \node[state, below right=1.1cm and 2cm of Q2] (Q5) {$\begin{array}{c}
                            *V
                        \end{array}$};
                \path[->]
                (Q1) edge[] node[left] {$1,0$} (Q3)
                (Q2) edge[] node[right] {$1,0$} (Q3)
                (Q1) edge[] node[above left=0cm and -0.2cm] {$1,0$} (Q4)
                (Q2) edge[] node[above] {$1,0$} (Q5);

                \node [below left=-0.5cm and 1.7cm of Q1] {$M_3$:};

            \end{tikzpicture}
        \end{minipage}

    \caption{Visualization of the automata $M_1'$ and $M_3$.}\label{m1-m3-qfa}
\end{figure}
    From $M_3$ we can infer the equality $*(1 + A) = *(1 + V)$, while the equalities $*(1 + *^n A) = *(1 + *^n V)$ for $n \geq 1$ are implied by $M_1'$ and $M_2$, but not by $M_3$.
\end{example}

\subsection{Join Using the Partition}\label{subsection:join-partition}

We propose a second approach to compute the join of two quantitative automata.
This approach is less precise than the previous one, as it only considers the partition of the terms $\T$ and not the automaton.

As before, we first extend the two partitions to the same set of terms $\T$ that contains all terms of both partitions.

Let $P_1 = (\T, \tau_1, \omega_1)$ and $P_2 = (\T, \tau_2, \omega_2)$ be two partitions.
We define the partition $P_3 = (\T, \tau_3, \omega_3)$ that represents the join of $P_1$ and $P_2$.
The terms of $\T$ are partitioned in sets $Q_i$ such that for each two terms $t_1$ and $t_2$ in the same set, it holds that $\tau_1\,t_1 = \tau_1\,t_2$, $\tau_2\,t_1 = \tau_2\,t_2$ and $\omega_2\,t_1 -\omega_1\,t_1 = \omega_2\,t_2 - \omega_1\,t_2$.
Then for each equivalence class $Q_i$ we choose a representative term $t_i\in Q_i$ and
for each $t \in Q_i$ we define $\tau_3\,t = t_i$ and $\omega_3\,t = \omega_1\,t - \omega_1\,t_i$.

The corresponding automaton $M_3$ can be computed starting from the partition $P_3$,
as described in \cref{subsection:qfa}.


\begin{example}\label{ex:join-cycle}

\begin{figure}\begin{subfigure}{0.5\textwidth}
    \centering
      \begin{tikzpicture}[shorten >=1pt, node distance=3cm, on grid, auto, state/.style={rectangle, rounded corners, draw, inner sep=5pt, align=center}]

            \node[state] (Q1) {$\&x, *x$};
            \node[state, right=of Q1] (Q2) {$x, {*}{*}x$};

            \path[->]
              (Q1) edge[bend left, above] node {$0,0$} (Q2)
              (Q2) edge[bend left, below] node {$0,0$} (Q1);

          \end{tikzpicture}


      \caption{The QFA of the conjunction $\&x = *x$.}\label{fig:join-diff-example}
    \end{subfigure}
    \begin{subfigure}{0.5\textwidth}
        \centering
          \begin{tikzpicture}[shorten >=1pt, node distance=3cm, on grid, auto, state/.style={rectangle, rounded corners, draw, inner sep=5pt, align=center}]

                \node[state] (Q1) {$\&x, {*}{*}x$};
                \node[state, right=of Q1] (Q2) {$x$};
                \node[state, below right=of Q1] (Q3) {$*x$};


                \path[->]
                  (Q1) edge[bend left] node {$0,0$} (Q2)
                  (Q2) edge[bend left] node {$0,0$} (Q3)
                  (Q3) edge[bend left] node {$0,0$} (Q1);

              \end{tikzpicture}


          \caption{The QFA of the conjunction $\&x = {*}{*}x$.}\label{fig:join-diff-example2}
        \end{subfigure}
        \caption[An example of quantitative finite automata.]{QFAs for the conjunctions $\&x = *x$ and $\&x = {*}{*}x$.}
    \end{figure}

    The main difference between the two join algorithms is the behavior when the automata contain cycles.
    For example, let $\Psi_1 \equiv \&x = *x$ and $\Psi_2 \equiv \&x = {*}{*}x$.
    Assume that $\T = \{\&x, x, *x, {*}{*}x\}$.
    The automaton $M[\Psi_1]$ is a cycle of length 2 (\cref{fig:join-diff-example}) and the automaton $M[\Psi_2]$ is a cycle of length 3 (\cref{fig:join-diff-example2}).
    If we use the join of the automata, the resulting automaton is a cycle of length 6.
    However, the join of the partitions only considers the terms in $\T$, which are all in distinct equivalence classes in the join.
    Therefore, all information about any equalities are lost with the partitions-join.

\end{example}

\subsection{Join of Two Kernels}

Given two kernels $k_1 = (P_1, M_1,B_1, D_1)$ and $k_2 = (P_2, M_2,B_2, D_2)$, we want to compute the kernel $k_3 = (P_3, M_3,B_3, D_3)$ that represents the join of the two kernels.
We already discussed two different methods of computing the partition $P_3$ and the automaton $M_3$ that represents the join of the two automata.
Now we consider the (block-)disequalities that follow from both kernels,
assuming that $P_3$ and $M_3$ are the join of $P_1$ and $P_2$ and $M_1$ and $M_2$, respectively,
using either of the two methods described above.

\subsubsection{Block Disequalities}

In order to infer the block disequalities that follow from the conjunction of two kernels $k_1$ and $k_2$,
we rewrite the block disequalities of $k_1$ and $k_2$ to use the new representatives of the new equivalence classes of $P_3 = (\T, \tau_3, \omega_3)$.
Then, we take the intersection of these two sets of block disequalities.

Thus, $B_3$ is defined as
\[
    \begin{array}{ll}
        B_3 & = \{\{\tau_3\,t_1, \tau_3\,t_2\}  \mid                                                \\
            & (\{\tau_1\,t_1, \tau_1\,t_2\} \in B_1 \land \{\tau_2\,t_1, \tau_2\,t_2\} \in B_2) \}.
    \end{array}
\]

$B_3$ is computed by rewriting each equality $\{t, t'\} \in B_1$ to use the new representatives of the equivalence classes in $P_3$.
We consider the disequality $\{t_1, t_2\}$
for each terms $t_1$ and $t_2$ that are representatives in $P_3$ and for
which $t_1$ is in the same equivalence class of $t$ and $t_2$ is in the same equivalence class of $t'$ in both partition $P_1$.
There could be multiple such sets $\{t_1, t_2\}$,
as the equivalence classes of $P_1$ might be divided into multiple equivalence classes in $P_3$.
We add each of these disequalities to $B_3$ if $\{\tau_2\,t_1, \tau_2\,t_2\} \in B_2$.

\subsubsection{Disequalities}

In \cref{section:disequalities}, we defined the two types of disequalities that follow from a conjunction:
the set of disequalities that follow from the explicitly stored disequalities,
and the set of implicit disequalities that follow from the equalities or the block disequalities.
As before for the equalities, it is also impossible to define a join operation that computes all disequalities that follow from two conjunctions, as is shown in the following example.

\begin{example}
    Let $\Psi_1 \equiv A = V + 1$ and $\Psi_2 \equiv A = V + 2$. There are infinitely many equalities implied by $\Psi_1$ and $\Psi_2$, i.e., $A \neq V + z$ for all $z \in \Z \setminus \{1,2\}$.
    There is no way to implicitly represent these equalities via equalities or block disequalities, and we can only store a finite amount of explicit disequalities.
    Therefore, it is not always possible to compute the exact set of disequalities implied by two conjunctions.
\end{example}

We have seen that the intersection between the set of implicit disequalities that follow from two conjunctions $\Psi_1$ and $\Psi_2$ could be infinite.
Therefore, we define the join to keep only those disequalities that follow from an explicit disequality in $\Psi_1$ and from an explicit or implicit disequality in $\Psi_2$ or vice versa.

Let $P_3 = (\T, \tau_3, \omega_3)$.
We define
\[
    \begin{array}{ll}
        D_3 & = \{(\tau_3\,t_1,  z + \omega_3\,t_2 - \omega_3\,t_1, \tau_3\,t_2) \mid                                                                            \\
            & ((\tau_1\,t_1, \tau_1\,t_2, z + \omega_1\,t_2 - \omega_1\,t_1) \in D_1 \lor (\tau_2\,t_1, \tau_2\,t_2, z + \omega_2\,t_2 - \omega_2\,t_1) \in D_2) \\
            & \land t_1 \nequiv_{k_1} t_2 + z \land t_1 \nequiv_{k_2} t_2 + z\},
    \end{array}
\]

where $\nequiv_{k}$ symbolizes the disequalities that are (implicitly or explicitly) implied by $k$, i.e.,
it is equal to $\nequiv_{\F[k],\T}$.
They can be determined by following the rules described in \cref{section:disequalities}.

$D_3$ can be computed in polynomial time in the number of explicitly stored disequalities by first constructing the set containing all equalities of $D_1$ by computing their closure with respect to the \emph{closure under quantitative equalities} rule \labelcref{item:closure-under-quantitative-equalities}.
This is necessary because each equivalence class of $P_1$ and $P_2$ might now be divided into multiple equivalence classes in $P_3$.
Then, for each resulting disequality $t_1 \neq z + t_2$,
we need to rewrite it to use the new representatives of the equivalence classes in $P_3$,
so we consider the disequality $\tau_3\,t_1 \neq z' + \tau_3\,t_2$,
where $z' = z + \omega_3\,t_2 - \omega_3\,t_1$.
If this equality is implied (implicitly or explicitly) by $k_2$, then it is added to $D_3$.

The same is done for the disequalities $D_2$.

The only disequalities that are lost by this definition of join are those that are implied only by the implicit disequalities of $\Psi_1$ and the implicit disequalities of $\Psi_2$.

Thus, we can define the join of two kernels $k_1 = (P_1, M_1,B_1, D_1)$ and $k_2 = (P_2, M_2,B_2, D_2)$ as $k_1 \join k_2 \equiv (P_3, M_3,B_3, D_3)$.

\section{Widening}

\subsection{Widening with automaton}


The partial order on $\mathcal{P}_2[=]$ has infinite strictly descending chains, as well as
infinite strictly ascending chains. Therefore we need to define a widening operator.

\begin{example}
    Consider the sequence of conjunctions
    \[
        \Psi_n \equiv (*^{2^n} \&x = \&x)\hspace{6pt} (n\geq 0).
    \]
   This could for example occur in a list structure, where each list element points to the next element of the list, and the last element points again to the first element of the list.
   Then we have for each $n \geq 0$, $\Psi_n \Longrightarrow \Psi_{n+1}$ and in particular the product automaton between $M_P[\Psi_n]$ and $M_P[\Psi_{n+1}]$ is $M_P[\Psi_{n+1}]$.
   The computation of this product automaton for $n = 0$ and $n = 1$ can be found in Example~\ref{example:join-subset}. This is an example of an infinite strictly ascending chain in $\mathcal{P}_2[=]$.
\end{example}

We decide to define the widening operator in such a way that the added result can't contain terms of arbitrary size, such that there can't be ascending chains of arbitrary size. Therefore we define the widening operator as first computing the join and then restricting the set of terms of the result to the set of terms that occur in the first input conjunction.

\begin{definition}
    For two conjunctions $\Psi_1$ and $\Psi_2$ over the sets of terms $\T_1$ and $\T_2$, respectively, we define the widening operation
     $\Psi_1 \widen \Psi_2 = \restr{(\Psi_1 \join \Psi_2)}{\T_1}$.
\end{definition}

For a given set $\T$ of terms, there is only a finite number of possible automata and sets of disequations that contain only terms of $\T$.
The set of terms doesn't change after having applied the widening operation, therefore the widening operator will always reach a fixpoint after a finite amount of iterations.
\ignore{
\begin{tikzpicture}[shorten >=1pt, node distance=2cm, on grid, auto]

    \node[state] (Q0) {$\&x$};

    \path[->]
      (Q0) edge [loop above] node {$0$} ();

  \end{tikzpicture}

  \begin{tikzpicture}[shorten >=1pt, node distance=2cm, on grid, auto]

    \node[state] (Q1) {$\&x$};
    \node[state, right=of Q1] (Q2) {$x$};

    \path[->]
      (Q1) edge[bend left, above] node {$0$} (Q2)
      (Q2) edge[bend left, below] node {$0$} (Q1);

  \end{tikzpicture}

% Automaton M1
\begin{tikzpicture}[shorten >=1pt, node distance=2cm, on grid, auto]

    \node[state] (Q0) {$A; B+1; D+2$};
  \end{tikzpicture}

  % Automaton M2
  \begin{tikzpicture}[shorten >=1pt, node distance=2cm, on grid, auto]

    \node[state] (Q1) {$A; B+1,D+2$};
  \end{tikzpicture}

  % Automaton M3
  \begin{tikzpicture}[shorten >=1pt, node distance=2cm, on grid, auto]

    \node[state] (Q0Q1_0) {$A$};
    \node[state, right=of Q0Q1_0] (Q0Q1_1) {$B; C-1$};
  \end{tikzpicture}
}

\subsection{Widening without automaton}


\section{Narrowing}

