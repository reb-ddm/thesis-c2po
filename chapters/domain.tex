\chapter{Abstract Domain}

In this chapter we present the representation of the abstract domain used for the \cpo\ analysis, alongside the operations that are fundamental for implementing an analysis based on abstract interpretation.
This includes a partial order and the operations \emph{meet}, \emph{join}, \emph{widening} and \emph{narrowing}.
Two different versions of the \emph{join} and of the \emph{equal} operation are introduced,
differing in precision and efficiency.
Prior to defining these operations, we discuss the extension of the domain to include additional terms and the restriction of the domain to a subset of terms,
both crucial for the subsequent operations.

The abstract values of the \cpo\ analysis consist of a representation
of 2-Pointer Logic conjunctions, as detailed in the previous chapter.
This ensures that whenever propositions are added to or removed from the domain,
the closure of all implied propositions is computed.

The representation consists of a tuple $k = \angl{P,M_P,B,D}$,
where $P = (\T, \tau, \omega)$ is a quantitative partition, as described in \cref{subsection:quantitative-union-find}.
Each state of the QFA $M$ represents an equivalence class of $P$, as outlined in \cref{subsection:qfa}.
$B$ and $D$ are lists of explicit block disequalities and disequalities, respectively, as detailed in \cref{section:block-disequalities,section:disequalities}.

The tuple $k$ is called the \emph{kernel} representation of $\Psi$.
For an unsatisfiable formula $\Psi$, the kernel representation is $\bot$.

It is posible to convert a kernel representation $k = \angl{P,M_P,B,D} \neq \bot$ with $M = (S, \eta, \delta)$ to a formula $\F[k]$, that is given by

\[
\begin{array}{lll}
    \F[k] & \equiv & \bigwedge_{\eta\,a=(z,s_i)} (a = z + \otau\,s_i) \land \\
    & & \bigwedge_{\delta\,(z_i,s_i)=(z_j,s_j)} (*(z_i + \otau\,s_i) = z_j + \otau\,s_j) \land \\
    & & \bigwedge_{\{t_1, t_2\} \in B} (bl(t_1) \neq bl(t_2)) \land \\
    & & \bigwedge_{(t_1, z, t_2) \in D} (t_1 \neq z + t_2)
\end{array}
\]

If $k$ is the kernel of a conjunction $\Psi$, then $\F[k]$ is equivalent to $\Psi$,
even though it may not be syntactically identical.
They could be syntactically different because of a different choice of representative terms.

\section{Extension to a superset of terms}

When new terms appear during the analysis, the set $\T$ and the partition $P$ need to be updated
in order to be able to express properties about the new terms.
It is not sufficient to simply add the new terms to a fresh equivalence class,
as they sometimes need to be added to existing equivalence classes.

\begin{example}
    Let $\Psi \equiv A = B$ and $\T = \{A, B, *A\}$.
    The parttion $\Pi$ is equal to $\{\{A, B\}, \{*A\}\}$.
    If we want to add the term $*B$ to $\T$, we need add it to the same equivalence class as $*A$,
    as they are equivalent according to the rules defined in \cref{chapter:qcc}.
    \todo{The same example is already somewhere else}
\end{example}

Given a partition $P = (\Pi, \tau, \omega)$ of set of terms $\T$ and a set $\T'$, which is a superset of $\T$ and closed under subterms,
we define the operation $\ext{k}{\T'}$ that extends the kernel $k = \angl{\T, P, M, B, D}$ to the set of terms $\T'$,
without altering the semantics of the representation.

The extension is built inductively based on the structure of the terms in $\T'$.
We describe how to add a new atom and a term of the form $*(z+t)$, where $t$ is already a part of $\T$.
This can be repeated for all terms in $\T'$.

In the case that $\T' = \T \cup \{a\}$ for an atom $a \notin \T$, we simply add a new
equivalence class $\{a\}$ to $\Pi$ and a new state $s_i$ to the states $S$ of $M$.
We define $\tau\,a = a$, $\omega\,a=0$, $\eta(a) = (0,s_i)$, $Q_i = \{a\}$.
The remaining elements of $k$ remain unchanged.

Consider $\T' = \T \cup \{*(z+t)\}$ for a term $t \in \T$, but $*(z + t) \notin \T$.
First we add a new equivalence class $\{*(z+t)\}$ to $\Pi$.
We define $\tau\,(*(z+t)) = *(z+t)$ and $\omega\,(*(z+t))=0$.
Let $Q_i \in \Pi$ be the equivalence class that contains $t$.
If $\delta(z + \omega\,t,s_i)$ is not defined, then we add a new state $s_j$ to $S$
and define $\delta(z + \omega\,t, s_i) = (0, s_j)$ and $Q_j = \{*(z+t)\}$.
Otherwise, let $\delta(z + \omega\,t, s_i) = (z', s_j)$.
Then we perform the operation \union{*(z + t)}{\tau\,Q_j}{z'}.
In this case, we do not need to add a new state to the automaton,
as the new term is added to the existing equivalence class $Q_j$.

\section{Restriction to a subset of terms}

\todo[inline]{Change it to a positive set instead of a negative set.}

During an assignment, it is necessary to forget information about a specific set of terms.
For instance, when assigning a value to a term, all terms in the domain that might be modified by this assignment---those that may alias with this term---must be removed.
This section explains how to restrict a kernel $k = \angl{P, M_P, B, D}$
to exclude all propositions containing terms from a specified set.

We begin by defining the restriction of a partition $P = (\T, \tau, \omega)$ and its corresponding automaton $M_P = (S, \otau, \eta, \delta)$ to forget information about the terms in a set $\Set$.
The complement of $\Set$ needs to be closed under subterms, ensuring that the
resulting partition will still contain terms from a set that remains closed under subterms.

Given a set $\Set$, we can find a restricted
automaton $\restr{M_P}{-\Set} = (S', \otau', \eta', \delta')$ such that $\L(\restr{M_P}{-\Set})$ does not include terms from
$\Set$, but retains the equalities between all other terms in $\L(M_P) \setminus \Set$.
From this, we derive the restricted partition $\restr{P}{-\Set} = (\T', \tau', \omega')$, where $\T'$ contains
the terms in $\T \setminus \Set$ and additionally all terms that appear in the codomain of the function $\otau'$.

This is necessary to ensure that for each state $s \in S'$, a representative term in $\T'$ can be found,
even if $\T \setminus \Set$ does not contain any term in $\L_{\restr{M_P}{-\Set}}(s)$.

\begin{example}
    Consider the conjunction $\Psi \equiv \&x = \&z \land *x = \&y$.
    We choose $\T = \{\&x, \&y, \&z, x, *x\}$. The corresponding automaton $M_P$ can be visualized as follows:

    \begin{center}
        \begin{tikzpicture}[shorten >=1pt, node distance=3cm, on grid, auto,
                state/.style={rectangle, rounded corners, draw, inner sep=3pt, align=center}]
            \node[state] (q_0) {$\&x, \&z$};
            \node[below=0.5cm of q_0] {$s_1$};
            \node[state] (q_1) [right=of q_0] {$x$};
            \node[below=0.5cm of q_1] {$s_2$};
            \node[state] (q_2) [right=of q_1] {$*x, \&y$};
            \node[below=0.5cm of q_2] {$s_3$}; % Label for the third state

            \path[->]
            (q_0) edge node {} (q_1)
            (q_1) edge node {} (q_2);
        \end{tikzpicture}
    \end{center}
    Let $\Set$ be the set of terms that contain the variable $x$.
    If we simply remove all terms from $\Set$ from the automaton, the state $s_2$ would
    not contain any terms from $\T \setminus \Set$.
    However, the set $\L_{M_P}(s_2)$ contains not only $x$, but also $z$, which is not in $\Set$.
    Therefore, we can choose $z$ as the new representative for $s_2$.
    The new set of terms after the restriction is $\T' = (\T \setminus \Set) \cup \{z\} = \{\&y, \&z, z\}$.
    The resulting automaton can be visualized as follows:
    \begin{center}
        \begin{tikzpicture}[shorten >=1pt, node distance=3cm, on grid, auto,
                state/.style={rectangle, rounded corners, draw, inner sep=3pt, align=center}]
            \node[state] (q_0) {$\&z$};
            \node[below=0.5cm of q_0] {$s_1$};
            \node[state] (q_1) [right=of q_0] {$z$};
            \node[below=0.5cm of q_1] {$s_2$};
            \node[state] (q_2) [right=of q_1] {$\&y$};
            \node[below=0.5cm of q_2] {$s_3$};

            \path[->]
            (q_0) edge node {} (q_1)
            (q_1) edge node {} (q_2);
        \end{tikzpicture}
    \end{center}
    It represents the formula $*z = \&y$.
    \todo{add edge weights}
\end{example}

The restriction $\restr{M_P}{-\Set} = (S', \otau', \eta', \delta')$ is computed using a breadth-first search (BFS) on the automaton $M_P$
to determine which paths remain reachable after excluding all terms in $\Set$.
For some states $s \in S$, the representative $\otau(s)$ might be in $\Set$.
If this is the case, we need to find a new representative for $s$ from the set $\L_{M_P}(s) \setminus \Set$.
If this set is empty, the state $s$ must be removed from the automaton.

The algorithm proceeds as follows, with $\V$ representing the set of visited states and $Q$ being a queue of states to be processed:
\begin{enumerate}
\item Start with the initial states of $M_P$ and consider each atom $a \in \T \setminus \Set$ where $\eta(a) = (z, s)$:
   \begin{itemize}
       \item If $s$ has not been visited, choose a new representative for $s$.
       Set $\otau'(s) = a$ and $\eta'(a) = (0, s)$.
       Add $s$ to $\V$ and to $Q$.
       \item If $s$ has already been visited, there exists a $t$ such that $\otau'(s) = t$.
       Let $M_P[t] = (z', s)$.
       Define $\eta'(a) = (z - z', s)$.
   \end{itemize}

\item Remove a state $s_1$ from the queue $Q$.
Consider all outgoing transitions $\delta(z_1, s_1) = (z_2, s_2)$.
   \begin{itemize}
       \item If a term $*(z+t) \in \L(M) \setminus \Set$ can be found such that $M_P[t] = (z_1', s_1)$ and $z = z_1 - z_1'$,
       the transition remains valid.
           \begin{itemize}
               \item If $s_2$ has not been visited, set $\otau'(s_2) = *(z+t)$ and add $s_2$ to $\V$ and to $Q$.
               \item If $s_2$ has been visited, $\otau'(s_2)$ is already defined.
               \item Let $M_P[\otau'(s_1)] = (z_3, s_1)$ and $M_P[\otau'(s_2)] = (z_4, s_2)$.
               Define the corresponding transition $\delta'(z_1 - z_3, s_1) = (z_2 - z_4, s_2)$.
           \end{itemize}
       \item If no such term is found, the transition is no longer valid in $\restr{M_P}{-\Set}$, and $s_2$ is not added to the set of visited states.
   \end{itemize}
\item If $Q$ is not empty, return to step 2 and continue processing.
\end{enumerate}

Essentially, the algorithm retains the transitions between the states that remain reachable after removing the terms and adjusts the weights of these transitions to the new representatives.
The set $S'$ is equivalent to the set of visited states $\V$.

The new partition $\restr{P}{-\Set} = (\T', \tau', \omega')$ is defined as follows:
\begin{itemize}
    \item $\T' = (\T \setminus \Set) \cup \{t \mid \exists s \in S' . \otau'\,s = t\}$,
    \item For each $t \in \T'$, if $\restr{M_P}{-\Set}[t] = (z, s)$, then $\tau'\,t = \otau'\,s$ and $\omega'\,t = z$.
\end{itemize}

\begin{theorem}\label{restriction}
    Let $M_P = (S, \otau, \eta, \delta)$ be a QFA and let $\restr{M_P}{-\Set} = (S', \otau', \eta', \delta')$ be the automaton that is obtained by restricting $M_P$ as described above.
    Then,
    \begin{enumerate}
        \item\label{item:lemma-restriction} for each term $t \in \L(\restr{M_P}{-\Set})$, it holds that $M_P[t] = (z, s)$ and $M[\otau'\,s] = (z', s')$ iff $\restr{M_P}{-\Set}[t] = (z - z', s)$ and $s = s'$,
        \item $\L(\restr{M_P}{-\Set}) = \L(M_P) \setminus \Set$ and
              \item\label{item:correctness-restriction} for each term $t_1, t_2 \in \L(\restr{M_P}{-\Set})$, it holds that $M_P[t_1] = z + M_P[t_2]$ iff $\restr{M_P}{-\Set}[t_1] = z + \restr{M_P}{-\Set}[t_2]$.
    \end{enumerate}
\end{theorem}
\begin{proof}
    The first point can be proven by induction over the structure of the term $t$.
    \todo[inline]{Proof of point 2}
    Point~\labelcref{item:correctness-restriction} follows from point~\labelcref{item:lemma-restriction}.
\end{proof}
%--------------
\subsubsection{(Block-)Disequalities}
We can also find the set of disequalities and the block disequalities that are still valid when considering only the terms $\T'$.
We keep only the propositions between representatives of the equivalence classes that
still have a corresponding state in the automaton $\restr{M_P}{-\Set}$, but we need to update the representatives to the new representatives.
The remaining disequalities are removed.

For $k = \bot$, the restriction $\restr{k}{-\Set}$ is also $\bot$.

Given a conjunction $\Psi$, we denote the restriction of the corresponding kernel $k$ to forget propostions about terms in $\Set$ by $\restr{\Psi}{-\Set}$.\todo{do you need this in the end?}

\section{Equality}

Given two kernels, $k_1$ and $k_2$, we want to decide whether they are semantically equivalent.
This section presents two methods for computing the \emph{equal} operation.
Their main difference is in how the equivalence of two partitions $P_1$ and $P_2$ is decided.
The first method compares the equivalence classes of the partitions $P_1$ and $P_2$.
The second method is based on computing a normal form of the conjunctions represented by the kernels $k_1$ and $k_2$.
It then suffices to compare the normal forms syntactically.
This approach has the advantage that the normal form can be computed once for every kernel, and there is no need to recompute it for every comparison.

\subsection{Comparing Equivalence Classes}

Let $P_1 = (\T_1, \tau_1, \omega_1)$ and $P_2 = (\T_2, \tau_2, \omega_2)$ be two partitions.
First, we extend both partitions to the set $\T = \T_1 \cup \T_2$.
Then we compare the equivalence classes of the resulting partitions $\ext{P_1}{\T} = (\T, \tau_1', \omega_1')$ and $\ext{P_2}{\T} = (\T, \tau_2', \omega_2')$.

For this, we need to check for each element $t \in \T$ if all equalities implied by $P_1$ are also implied by $P_2$.
Let $t' \equiv \tau_1'\,t$ and $z = \omega_1'\,t$.
It follows that $P_1$ implies the equality $t = z + t'$.
If $\omega_2'\,t + \tau_2'\,t = z + \omega_2'\,t' + \tau_2'\,t'$ holds, then the same equality is also implied by $P_2$.

Afterward, the disequalities and block disequalities are compared.
For two kernels $k_1 = \angl{P_1, M_{P_1}, B_1, D_1}$ and $k_2 = \angl{P_2, M_{P_2}, B_2, D_2}$,
we rewrite the disequalities $B_2$ and $D_2$ to be about the representatives of $P_1$.
For each block disequality $\{t_1,t_2\} \in B_2$, we convert it to the disequality $norm_{P_1}(bl(t_1) = bl(t_2))$ and for each disequality in $D_2$ of the form $(t_1, z, t_2) \in D$, we convert it to the disequality $norm_{P_1}(t_1 = z + t_2)$.
The normalization is possible because we previously extended the two partitions to the same set $\T$, i.e., $\tau_1'\,t$ is defined for each term $t$ occurring in $B_2$ and $D_2$.
Then, we check if the resulting set of block disequalities is equal to the set $B_1$ and if the resulting set of disequalities equals the set $D_1$.

\subsection{Compute Normal Form}

\textcite{2pointer} presents a different approach to deciding the equivalence of two partitions.
This is done by computing a normal form $\nf(\Psi)$ of a conjunction $\Psi$, such that $\Psi$ is semantically equivalent to $\nf(\Psi)$ and such that two conjunctions $\Psi_1$ and $\Psi_2$ are semantically equivalent iff $\nf(\Psi_1)$ and $\nf(\Psi_2)$ are syntactically equivalent.

The main idea is to find \emph{minimal representatives} for each state $s$ in the QFA $M_P$,
which corresponds to the smallest term in $\L_{M_P}(s)$, according to the order defined in \cref{chapter:2pointer}.
Then, the automaton is transformed to a formula that utilizes only the minimal representatives, thus obtaining a normal form that is independent of the chosen set $\T$ and on the chosen representatives.

Let $M_P = (S, \otau, \eta, \delta)$ be a QFA.\@
For each state $s \in S$, we compute the minimal term $m_s$ and the corresponding offset $z_s$ such that $M[m_s] = (z_s,s)$.
This can be computed using a variation of Dijktra's shortest path algorithm, which is described in the following.

Let $P$ be an initially empty FIFO queue, which we use to store all states $s$ for which $(m_s,z_s)$ is already computed but where the outgoing edges still have to be processed.

\begin{enumerate}
    \item
          First, we consider all atoms $a$ for which $\eta$ is defined in ascending order. For each $\eta\,a = (z,s)$, if $(m_s,z_s)$ is not defined yet, we set
          \[
              (m_s,z_s) = (a,z)
          \]
          and add $s$ to $P$.
    \item
          Then, we process the queue $P$ in a FIFO manner until it is empty.
          For each state $s$ in $P$, we consider all outgoing edges $\delta\,z\,s = (z',s')$ in order of ascending $z$.
          For each such edge where $(m_{s'},z_{s'})$ is not defined yet, we set
          \[
              (m_{s'},z_{s'}) = (*((z-z_s)+m_s),z')
          \]
          and add $s'$ to $P$.
\end{enumerate}

The correctness is given by the fact that longer paths always result in larger terms than shorter paths.
Thus, we do not need to update a pair once it was computed.

\begin{example}\label{ex:min-repr}
    Let $\Psi \equiv B = -1 + \&x \land *B = 2 + A \land *A = 3 + x$.
    The automaton $M_P$\todo{what is P} has can be visualized with the following graph:
    \begin{center}
        \begin{tikzpicture}[shorten >=1pt, node distance=3cm, on grid, auto, state/.style={rectangle, rounded corners, draw, inner sep=5pt, align=center}]

            \node[state] (Q1) {$\begin{array}{c}
                        \boldsymbol{\&x}, \\
                        1 + B
                    \end{array}$};

            \node[state, right=4.5cm of Q1] (Q2) {$\begin{array}{c}
                        \boldsymbol{*B}, 2 + A, \\
                        *(-1+\&x )
                    \end{array}$};

            \node[state, right=5.6cm of Q2] (Q3) {$\begin{array}{c}
                        \boldsymbol{x}, *(1 + B), -3 + *A, \\
                        -3 + *(-2+*B),...
                    \end{array}$};

            \path[->]
            (Q1) edge[] node {$-1,0$} (Q2)
            (Q2) edge[] node {$-2,3$} (Q3)
            (Q1) edge[bend right] node {$0,0$} (Q3);

            \node [above left=1cm and 0.3cm of Q1] {$s_0$};
            \node [above left=1cm and 0.5cm of Q2] {$s_1$};
            \node [above left=1cm and 0.7cm of Q3] {$s_2$};

        \end{tikzpicture}
    \end{center}

    The minimal representatives are computed by first considering the atoms in ascending order.
    Assuming that the linear order on atoms is $\&x < A < B$,
    we define first $m_{s_0} = \&x$ and $z_{s_0} = 0$.
    Then we define $m_{s_1} = A$ and $z_{s_1} = -2$.
    There is nothing to do for $B$ because we already defined $m_{s_0}$.
    The queue $P$ is now $\{s_0, s_1\}$.
    We consider the outgoing edge of $s_0$ with $z = -1$.
    There is nothing to do for $s_1$, as $m_{s_1}$ is already defined.
    Then we consider the outgoing edge of $s_0$ with $z = 0$
    and set $m_{s_2} = x$ and $z_{s_2} = 0$.
    Thus, we have found the minimal representative for each state.
\end{example}
Using the minimal representatives, we can now transform the kernel $k = \angl{P,M_P,B,D} \neq \bot$ with $M_P = (S, \otau, \eta, \delta)$ to a normal form $\F_{normal}[k]$,
defined as follows:

\[
    \begin{array}{lll}
        \F_{normal}[k] & \equiv & \bigwedge_{\eta\,a=(z,s)} (a = (z - z_s) + m_s) \land                                                                     \\
                       &        & \bigwedge_{\delta\,z\,s=(z',s')} (*((z-z_s) + m_s) = ((z'-z_{s'}) + m_{s'}) \land                                         \\
                       &        & \bigwedge_{s,s' \in S \land \{\otau\,s,\otau\,s'\}\in B} (bl(m_{s}) \neq bl(m_{s'})) \land           \\
                       &        & \bigwedge_{s,s' \in S \land (\otau\,s, z,\otau\,s') \in D} (m_{s} \neq (z+ z_{s} -z_{s'}) + m_{s'})
    \end{array}
\]

The trivial propositions of the form $t = 0 + t$ are removed from the formula, as well as the repeated equalities and the implicit disequalities of the form $\&x \neq z + \&y$ and $bl(\&x) \neq bl(\&y)$.

The definition of $\F_{normal}[k]$ is analogous to the formula representation $\F[k]$, but it expresses the proposition using the minimal representatives instead of the union-find representatives.
The advantage of this normal form is that the formula representation depends neither on the chosen representatives nor the chosen set $\T$.
Thus, there is a unique normal form for each semantic equivalence class of kernels, even though multiple kernel representations of the same conjunction exist.
Therefore, in order to decide the equivalence of two kernels $k_1$ and $k_2$, it suffices to compare the normal forms $\F_{normal}[k_1]$ and $\F_{normal}[k_2]$ syntactically.

\begin{example}
    Consider the conjunction $\Psi$ from \cref{ex:min-repr}.
    The normal form of the kernel $k$ representing $\Psi$ is
    \[
        (B = -1+\&x) \land (*(-1 + \&x) = 2 + A) \land (*A = 3 + x)
    \]
    where $B = -1 + \&x$ originates from $\eta\,B = (-1,s_0)$.
    The equalities originating from $\eta\,\&x = (0,s_0)$ and $\eta\,A = (0, s_1)$ are removed, because they are trivial.
    The edge $\delta\,(-1,s_0) = (0,s_0)$ gives rise to the equality $*(-1 + \&x) = 2 + A$ and $\delta\,(-2,s_1) = (3,s_2)$ generates the equality $*A = 3 + x$.
    The last edge $\delta\,(0,s_0) = (0,s_2)$ generates the trivial equality $x = 0 + x$.
\end{example}

\todo[inline]{Correctness proof}

\todo[inline]{Proof that they are equivalent}

The advantage of computing the normal form instead of comparing the equivalence classes is that it only needs to be computed once per kernel, and the resulting normal form can be reused for each equality check.
In the implementation, it is possible to configure which of the two algorithms for equality is used by the analysis.
If the normal form algorithm is chosen, then the kernel contains an additional field containing the normal form.
This normal form is \emph{lazily} computed, i.e., it will only be calculated when needed, and it is not computed if it is never used.
Once calculated, the result is stored directly in the kernel and can be reused.


\section{Partial Order}

The natural partial ordering between conjunctions is the semantic implication, i.e., $\Psi_1 \rightarrow \Psi_2$ if and only if $(\rho, \nu, \mu) \models \Psi_1$ whenever $(\rho, \nu, \mu) \models \Psi_2$.
This is the case if and only if $\Psi_1 \land \Psi_2$ is semantically equal to $\Psi_1$.
Therefore, it is possible to reduce the \emph{less equal} operation to a \emph{meet} and an \emph{equal} operation.
A kernel $k_1$ that represents a conjunction $\Psi_1$ is \emph{less or equal} to a kernel $k_2$ representing $\Psi_2$ if and only if $k_1 \meet k_2$ is semantically equal to $k_1$.
We have already described how semantic equality between two kernels can be decided.
In the next section, the \emph{meet} operation ($\meet$) is presented.


\section{Meet}


\section{Join}

Given two conjunctions $\Psi_1$ and $\Psi_2$, the least upper bound is a conjunction implied by both $\Psi_1$ and $\Psi_2$ that is smallest with respect to the partial order.
Intuitively, the least upper bound is $\Psi_1 \lor \Psi_2$.
However, disjunctions are not expressible in the 2-Pointer Logic, so we are looking for the most precise conjunction that is implied by $\Psi_1 \lor \Psi_2$.
The conjunction of all propositions implied by a disjunction is not always representable as a conjunction of finite size~\cite{join,2pointer}. Therefore, it is impossible to compute the exact least upper bound of two conjunctions.


\begin{example}\label{example:infinite-join}
    Adapting the original example by \textcite{join} to \cpo, consider the conjunctions $\Psi_1 \equiv (A = B)$ and $\Psi_2 \equiv (*A = A)\land (*B=B) \land (*(1 + A) = *(1 + B))$, represented by the automata $M_1$ and $M_2$ shown below:
    \begin{center}
        \begin{minipage}{0.3\textwidth}
            \centering
            \begin{tikzpicture}[shorten >=1pt, node distance=3cm, on grid, auto, state/.style={rectangle, rounded corners, draw, inner sep=5pt, align=center}]

                \node[state] (Q1) {$\begin{array}{c}
                            A,B
                        \end{array}$};
                \node [left=1.5cm of Q1] {$M_1$:};
            \end{tikzpicture}
        \end{minipage}
        \begin{minipage}{0.65\textwidth}
            \centering
            \begin{tikzpicture}[shorten >=1pt, node distance=3cm, on grid, auto, state/.style={rectangle, rounded corners, draw, inner sep=5pt, align=center}]

                \node[state] (Q1) {$\begin{array}{c}
                            A,*A,..
                        \end{array}$};

                \node[state, right=3cm of Q1] (Q2) {$\begin{array}{c}
                            B,*B,..
                        \end{array}$};

                \node[state, below right=2cm and 1.5cm of Q1] (Q3) {$\begin{array}{c}
                            *(1+A), *(1 + B),...
                        \end{array}$};

                \path[->]
                (Q1) edge[loop below] node[left] {$0,0$} (Q1)
                (Q2) edge[loop below] node[right] {$0,0$} (Q2)
                (Q1) edge[] node[right] {$1,0$} (Q3)
                (Q2) edge[] node[above left] {$1,0$} (Q3);

                \node [below left=1cm and 1.5cm of Q1] {$M_2$:};

            \end{tikzpicture}
        \end{minipage}
    \end{center}

    Then for each $n \geq 0$, the equality $*(1+*^n A) = *(1 + *^n B)$ is implied both by $\Psi_1$ and $\Psi_2$, where $*^m$ denotes $m$-fold dereference.
    It can be shown that the set of quantitative equalities implied both by $\Psi_1$ and $\Psi_2$ cannot be represented by a finite conjunction~\cite{join, 2pointer}.
\end{example}

This example shows that it is not possible to compute the exact least upper bound of two domain elements because of the limitations in the expressiveness of the congruence closure representation.
However, it is possible to define a sound over-approximation of the join operation.

Given that equalities only depend on the automaton and cannot follow from the disequalities, we first consider how to construct an automaton that represents the conjunction of two automata.
Later, we will consider the (block-)disequalities that follow from $\Psi_1$ and $\Psi_2$.

As we only compute an approximated join operation, multiple possibilities exist to define the join.
A compromise must be made between the precision of the join and the computational complexity.
Two different approaches will be presented in the following sections.
The first one computes the join based on the automata representation of the equalities, while the second only considers the partition $P$ of the terms $\T$.
This first approach is less efficient but more precise, as all terms of the set $\L(M_P)$ are considered, while in the second approach, only the terms of the set $\T$ are considered.
In particular, circular dependencies between terms are lost in the second approach.
However, these circular dependencies do not often occur in practice.
Therefore, we expect that the difference in precision is not significant.
%-------
\subsection{Join Using the Automaton}

We propose an implementation of the join algorithm of two quantitative finite automata.
It is based on the work of ~\textcite{join}, where they discuss join algorithms for congruence closure data structures.

Given two partitions $P_1 = (\T_1, \tau_1, \omega_1)$ and $P_2 = (\T_2, \tau_2, \omega_2)$ and the corresponding automata $M_{P_1}$ and $M_{P_2}$, we want to compute the partition $P_3$ and the automaton $M_{P_3}$ that represent the conjunction of the two automata.
The first step is to extend the set of terms of $P_1$ and $P_2$ to $\T = \T_1 \cup \T_2$.
Henceforth, we assume that both partitions are defined over the same set of terms $\T$,
i.e., $P_1 = (\T, \tau_1, \omega_1)$ and $P_2 = (\T, \tau_2, \omega_2)$.
Let $M_{P_1}$ and $M_{P_2}$ be the corresponding automata.

The join is defined as the product automaton of $M_1$ and $M_2$.

\begin{definition}
    Let $M_1 = (S_1, \otau_1, \eta_1, \delta_1)$ and $M_2 = (S_2, \otau_2, \eta_2, \delta_2)$ be two quantitative finite automata.
    The \emph{quantitative product automaton} $M_3 = (S_3, \otau_3, \eta_3, \delta_3)$ is defined as:
    \begin{itemize}
        \item If $\eta_1(a) = (z_1, s_1)$ and $\eta_2(a) = (z_2, s_2)$, then
              $\eta_3(a) = (z_1, \angl{s_1, s_2, z_2-z_1})$.
        \item If $\delta_1(z_1, s_1) = (z'_1, s'_1)$ and $\delta_2(z_2, s_2) = (z'_2, s'_2)$, then $\delta_3(z_1, \angl{s_1, s_2, z_2 - z_1}) =  (z'_1, \angl{s'_1, s'_2, z'_2 -z'_1})$.
        \item $S_3 \subseteq S_1 \times S_2 \times \Z$. $S_3$ contains all states that are reachable in $M_3$.
        \item For each $s \in S_3$, $\otau_3(s)$ is any term in $\L_{M_3}(s)$.
    \end{itemize}
\end{definition}

The product automaton is finite, as there are only a finite amount of reachable states in $S_1 \times S_2 \times \Z$.
The remaining states are unreachable, i.e., they neither occur in a return value of $\eta_3$ nor $\delta_3$.
This follows from the fact that $\eta_i$ and $\delta_i$ for $i = 1,2$ are defined only for a finite amount of values, as discussed in \cref{subsection:qfa}.

The product can be computed in polynomial time, as we need to add an initial state for all pairs of initial states of $M_1$ and $M_2$ and then add a transition for each pair of transitions of $M_1$ and $M_2$. Therefore, the runtime is equal to the size of $M_1$ times that of $M_2$.

\begin{lemma}\label{lemma:join-automata}
    Let $M_3$ be the product automaton of two quantitative automata $M_1$ and $M_2$.
    Then, for each term $t \in \L(M_1) \cap \L(M_2)$ it holds that $M_1[t] = (z_1, s_1)$ and $M_2[t] = (z_2, s_2)$ iff $M_3[t] = (z_1, \angl{s_1, s_2, z_2 - z_1})$.
\end{lemma}
\begin{proof}
    We prove it by induction over the structure of the term $t$.
\end{proof}

As shown in Example~\ref{example:infinite-join}, it is not possible to define a join algorithm that is complete.
In fact, we can only show for our definition of the join algorithm that it is at least complete over the terms in $\L(M_1) \cap \L(M_2)$.
The following proposition shows that our definition of join is complete over the set of terms $\L(M_1) \cap \L(M_2)$ and that it is sound.

\begin{proposition}\label{prop:join-automata}
    Let $M_3$ be the product automaton of two quantitative automata $M_1$ and $M_2$.
    Then, for all terms $t_1, t_2 \in \L(M_1) \cap \L(M_2)$ it holds that $M_3[t_1] = z + M_3[t_2]$ iff $M_1[t_1] = z + M_1[t_2]$ and $M_2[t_1] = z + M_2[t_2]$.
\end{proposition}
\begin{proof}
    This proposition follows from Lemma~\ref{lemma:join-automata}.
\end{proof}

As we mentioned before, we added all terms of $\T = \T_1 \cup \T_2$ to both automata before computing the product automaton, and after having added these terms, it holds that $\T \subseteq \L(M_1) \cap \L(M_2)$.
Therefore, the join of two automata is complete at least over the terms in $\T$.

Using the automaton $M_3$, we can compute the corresponding partition $P_3$.
Let $\T' = \{t \mid \exists s \in S_3 . \otau_3(s) = t\}$ be the set of terms chosen as representatives for the states of $M_3$.
The partition $P_3$ is defined as $P_3 = (\T' \cup \T, \tau_3, \omega_3)$,
where for each $t \in \T' \cup \T$, $\tau_3\,t = \otau_3\,(s)$ and $\omega_3\,t = z$ if $M_3[t] = (z,s)$.

\todo[inline]{examples}

\begin{example}
    We consider two automata $M_1$ and $M_2$, where $M_1$ represents the conjunction $A = B + 2 \land A = C + 3$ and $M_2$ represents the conjunction $A = B + 1 \land A = C + 2$.
    The automaton $M_1$ has one single state $s_0$ that represents the equivalence class $\{A, B+2, C+3\}$
    and $M_2$ has one single state $s_1$ that represents the equivalence class $s_1 = \{A, B+1, C+2\}$.
    The product automaton $M_3$ is made of no transitions and two states $\angl{s_0, s_1, 0}$ and $\angl{s_0, s_1, 1}$, where the first one represents the set $\{A\}$ and the second one the set $\{B, C+1\}$.
    Therefore the product automaton represents the proposition $B = C + 1$.
    The join operation keeps the relation between $B$ and $C$, as they have the same distance in $M_1$ and $M_2$.
    On the other hand, $A$ has not the same distance to $B$ or $C$ in the input automata, therefore we forget all information about $A$.
\end{example}

\begin{example}\label{example:join-subset}
    We consider two automata $M_1$ and $M_2$, where $M_1$ represents the conjunction $x = \&x$ and $M_2$ represents the conjunction $*x = \&x$.
    $M_1$ has one state $s_0$ and one transition from $s_0$ to itself. We have that $\L_{M_1}(s_0) =\{*^n\&x\mid n \geq 0\}$.
    The automaton $M_2$ has two states $s_1$ and $s_2$ and two transitions from $s_1$ to $s_2$ and from $s_2$ to $s_1$. We have that $\L_{M_2}(s_1) = \{*^{2n}\&x\mid n \geq 0\}$ and $\L_{M_2}(s_2) = \{*^{2n}x\mid n \geq 0\}$.
    We have that $\eta_1(\&x) = (0, s_0)$ and $\eta_2 (\&x) = (0, s_1)$.
    The product automaton $M_3$ has two states $\angl{s_0, s_1, 0}$ and $\angl{s_0, s_2, 0}$ and two transitions from the first state to the second and from the second state to the first state and $\eta_3(\&x) = (0, \angl{s_0, s_1, 0})$.
    Therefore the product automaton is isomorphic to the second automaton and it represents the same equality $*x = \&x$.
\end{example}

\subsection{Join Using the Partition}\label{subsection:join-partition}

We propose a second approach to compute the join of two quantitative automata.
This approach is less precise than the previous one, as it only considers the partition of the terms $\T$ and not the automaton.

As before, we first extend the two partitions to the same set of terms $\T$ that contains all terms of both partitions.

Let $P_1 = (\T, \tau_1, \omega_1)$ and $P_2 = (\T, \tau_2, \omega_2)$ be two partitions.
We define the partition $P_3 = (\T, \tau_3, \omega_3)$ that represents the join of $P_1$ and $P_2$.
The terms of $\T$ are partitioned in sets $Q_i$ such that for each two terms $t_1$ and $t_2$ in the same set, it holds that $\tau_1\,t_1 = \tau_1\,t_2$, $\tau_2\,t_1 = \tau_2\,t_2$ and $\omega_2\,t_1 -\omega_1\,t_1 = \omega_2\,t_2 - \omega_1\,t_2$.
Then for each equivalence class $Q_i$ we choose a representative term $t_i\in Q_i$ and
for each $t \in Q_i$ we define $\tau_3\,t = t_i$ and $\omega_3\,t = \omega_1\,t - \omega_1\,t_i$.

The corresponding automaton $M_{P_3}$ can be computed starting from the partition $P_3$,
as described in \cref{subsection:qfa}.

\subsection{Join of Two Kernels}

Given two kernels $k_1 = (P_1, M_{P_1},B_1, D_1)$ and $k_2 = (P_2, M_{P_2},B_2, D_2)$, we want to compute the kernel $k_3 = (P_3, M_{P_3},B_3, D_3)$ that represents the join of the two kernels.
We already discussed two different methods of computing the partition $P_3$ and the automaton $M_{P_3}$ that represents the join of the two automata.
Now we consider the (block-)disequalities that follow from both kernels,
assuming that $P_3$ and $M_{P_3}$ are the join of $P_1$ and $P_2$ and $M_{P_1}$ and $M_{P_2}$, respectively,
using either of the two methods described above.

\subsubsection{Block Disequalities}

In order to infer the block disequalities that follow from the conjunction of two kernels $k_1$ and $k_2$,
we rewrite the block disequalities of $k_1$ and $k_2$ to use the new representatives of the new equivalence classes of $P_3 = (\T, \tau_3, \omega_3)$.
Then, we take the intersection of these two sets of block disequalities.

Thus, $B_3$ is defined as
\[
\begin{array}{ll}
    B_3 & = \{\{\tau_3\,t_1, \tau_3\,t_2\}  \mid                                                \\
        & (\{\tau_1\,t_1, \tau_1\,t_2\} \in B_1 \land \{\tau_2\,t_1, \tau_2\,t_2\} \in B_2) \}.
\end{array}
\]

$B_3$ is computed by rewriting each equality $\{t, t'\} \in B_1$ to use the new representatives of the equivalence classes in $P_3$.
We consider the disequality $\{t_1, t_2\}$
for each terms $t_1$ and $t_2$ that are representatives in $P_3$ and for
which $t_1$ is in the same equivalence class of $t$ and $t_2$ is in the same equivalence class of $t'$ in both partition $P_1$.
There could be multiple such sets $\{t_1, t_2\}$,
as the equivalence classes of $P_1$ might be divided into multiple equivalence classes in $P_3$.
We add each of these disequalities to $B_3$ if $\{\tau_2\,t_1, \tau_2\,t_2\} \in B_2$.

\subsubsection{Disequalities}

In \cref{section:disequalities}, we defined the two types of disequalities that follow from a conjunction:
the set of disequalities that follow from the explicitly stored disequalities,
and the set of implicit disequalities that follow from the equalities or the block disequalities.
As before for the equalities, it is also impossible to define a join operation that computes all disequalities that follow from two conjunctions, as is shown in the following example.

\begin{example}
    Let $\Psi_1 \equiv A = B + 1$ and $\Psi_2 \equiv A = B + 2$. There are infinitely many equalities implied by $\Psi_1$ and $\Psi_2$, i.e., $A \neq B + z$ for all $z \in \Z \setminus \{1,2\}$.
    There is no way to implicitly represent these equalities via equalities or block disequalities, and we can only store a finite amount of explicit disequalities.
    Therefore, it is not always possible to compute the exact set of disequalities implied by two conjunctions.
\end{example}

We have seen that the intersection between the set of implicit disequalities that follow from two conjunctions $\Psi_1$ and $\Psi_2$ could be infinite.
Therefore, we define the join to keep only those disequalities that follow from an explicit disequality in $\Psi_1$ and from an explicit or implicit disequality in $\Psi_2$ or vice versa.

Let $P_3 = (\T, \tau_3, \omega_3)$.
We define
\[
\begin{array}{ll}
    D_3 & = \{(\tau_3\,t_1,  z + \omega_3\,t_2 - \omega_3\,t_1, \tau_3\,t_2) \mid                                                                            \\
        & ((\tau_1\,t_1, \tau_1\,t_2, z + \omega_1\,t_2 - \omega_1\,t_1) \in D_1 \lor (\tau_2\,t_1, \tau_2\,t_2, z + \omega_2\,t_2 - \omega_2\,t_1) \in D_2) \\
        & \land t_1 \nequiv_{k_1} t_2 + z \land t_1 \nequiv_{k_2} t_2 + z\},
\end{array}
\]

where $\nequiv_{k}$ symbolizes the disequalities that are (implicitly or explicitly) implied by $k$, i.e.,
it is equal to $\nequiv_{\F[k],\T}$.
They can be determined by following the rules described in \cref{section:disequalities}.

$D_3$ can be computed in polynomial time in the number of explicitly stored disequalities by first constructing the set containing all equalities of $D_1$ by computing their closure with respect to the \emph{closure under quantitative equalities} rule \labelcref{item:closure-under-quantitative-equalities}.
This is necessary because each equivalence class of $P_1$ and $P_2$ might now be divided into multiple equivalence classes in $P_3$.
Then, for each resulting disequality $t_1 \neq z + t_2$,
we need to rewrite it to use the new representatives of the equivalence classes in $P_3$,
so we consider the disequality $\tau_3\,t_1 \neq z' + \tau_3\,t_2$,
where $z' = z + \omega_3\,t_2 - \omega_3\,t_1$.
If this equality is implied (implicitly or explicitly) by $k_2$, then it is added to $D_3$.

The same is done for the disequalities $D_2$.

The only disequalities that are lost by this definition of join are those that are implied only by the implicit disequalities of $\Psi_1$ and the implicit disequalities of $\Psi_2$.

Thus, we can define the join of two kernels $k_1 = (P_1, M_{P_1},B_1, D_1)$ and $k_2 = (P_2, M_{P_2},B_2, D_2)$ as $k_1 \join k_2 \equiv (P_3, M_{P_3},B_3, D_3)$.


\section{Widening}

The join operation that we defined does not compute the exact least upper bound of two automata, but only an approximation.
However, the loss of information is not enough to guarantee that the join operation always reaches a fixpoint after a finite number of iterations.

\begin{example}
  Consider the sequence of kernels corresponding to the conjunctions
  \[
    \Psi_n \equiv (*^{2^n} \&x = \&x)\hspace{6pt} (n\geq 0).
  \]
  This could for example occur in a list structure, where each element points to its successor, and the last element points again to the first element of the list.
  Then we have for each $n \geq 0$, $\Psi_n \Longrightarrow \Psi_{n+1}$ and in particular the product automaton between $M_P[\Psi]$ and $M_P[\Psi_{n+1}]$ is $M_P[\Psi_{n+1}]$.
  If we use the second algorithm for join and compute the join of the partitions $P[\Psi_n]$ and $P[\Psi_{n+1}]$, we get the partition $P[\Psi_{n+1}]$.
\end{example}

Therefore, we need to define a widening operator~\cite{widening} that finds an even less precise upper bound of two automata than the join, such that it reaches a fixpoint after a finite number of iterations.

We decide to define the widening operator in a way that the result can't contain terms of arbitrary size, such that there can't be ascending chains of arbitrary size.
Therefore, we define the widening operator as first computing the join and then restricting the set of terms of the result to the set of terms that occur in the first input conjunction.

\begin{definition}
  For two conjunctions $\Psi_1$ and $\Psi_2$ over the sets of terms $\T_1$ and $\T_2$, respectively, we define the widening operation as
  $\Psi_1 \widen \Psi_2 = \restr{(\Psi_1 \join \Psi_2)}{\T_1}$.
\end{definition}

The widening operator depends on which join operator was chosen, and it
works for both possibilities.
However, if we choose to use the join operator that uses the partition, the one described in \cref{subsection:join-partition},
then it can be implemented in a more efficient way.
Instead of adding all the terms of the two kernels to both kernels at the beginning of the join,
we simply add the terms of the first kernel to the second kernel,
and we build the resulting partition starting from the set of terms of the first kernel.
This computes the same result as the above defined widening operator,
but it avoids computing a restriction.

For a given set $\T$ of terms, there is only a finite number of possible automata and sets of disequalities that contain only terms of $\T$.
The set of terms doesn't change after having applied the widening operation, therefore the widening operator will always reach a fixpoint after a finite amount of iterations.

\begin{example}\label{ex:widen}\todo{fix and add graph}
  If we consider the sequence of conjunctions $\Psi_n$ from \cref{ex:widen-ascending-chain}, the automaton representation $M_P$\todo{what is P} is a cycle of length $2^n$.
  If we apply widening to $\Psi_n$ and $\Psi_{n+1}$, the resulting automaton will contain a simple path from the equivalence class of $\&x$ to the equivalence class of $*^{2^{n}-1}\&x$,
  and no cycles.
  This means that each term is in its own equivalence class, we have lost all information about equalities.
  Therefore, the widening terminates in the next iteration,
  as we already reached the top element of our domain.
  % For two cycle-free automata $M_1$ and $M_2$, if $\L(M_2) \subseteq \L(M_1)$,
  % then the widen is equivalent to the join operation.
\end{example}

\section{Narrowing}

As shown in~\cite{2pointer}, the domain of 2-pointer propositions also has infinite strictly descending chains.

\begin{example}
	Consider the sequence of conjunctions
	\[
	\Phi_n \equiv\bigwedge_{i=1}^n (\&y = *(i+\&x))\qquad(n\geq 0)
	\]
	Then $\Phi_{n+1}\implies\Phi_n$ for all $n\geq 0$ and all are inequivalent.
	Thus, the $\Phi_n,n\geq 0$, constitute an infinite strictly descending chain in \cpo.
\end{example}

Similarly to the widening operation, we define the narrowing as the meet operation with a limited set of terms.
However, it is not sufficient to limit the terms of the result, we also need to limit the possible offsets
appearing in disequalities.\cite{new-paper-todo}

\begin{example}\label{e:narrow1}
	Consider the sequence of kernels corresponding to the propositions
	\[
	\Psi_n \equiv (y\neq n+\&x)\qquad(n\geq 0)
	\]
	Let $k_n$ denote the kernel of $\Psi_n$.
  Then all kernels consider the set $\T=\{\&x,\&y,y\}$ of terms.
	The sets $D_n$ of disequalities, are given by $D_n=\{(y,n,\&x)\}$.
  Consider the sequence $a_n, n\geq 0$ defined by $a_0 = k_0$ and $a_n = a_{n-1}\meet k_n$ for $n>0$.
  Each element $a_m$, is a kernel with set $D'_m$ of quantitative disequalities given by
	\[
	D'_m = \{(y,j,\&x)\mid 0\leq j\leq m\}.
	\]
\end{example}

To guarantee that the kernel does not grow in each iteration
and that a fixpoint is always reached, we only consider the propositions that contain terms of the first input conjunction
and disequalities that contain offsets that are in the range of offsets of the first input conjunction.

Let $k = \angl{P,M_P,B,D}$ be a kernel. We define $max\_offset(D)$ to be the biggest absolute value of an offset that appears in the disequalities of $D$:
\[
max\_offset(D) = \max\{ \lvert z \rvert \mid (t_1,z,t_2) \in D \}.
\]

Let $k_1$ and $k_2$ be the kernels.
As for the meet, if either $k_1 = \bot$ or $k_2 = \bot$, then $k_1 \meet k_2 = \bot$.
Otherwise, we consider the formula representations $\F[k_1]$, and we keep only the propositions that contain terms of $\T_1$.
We also filter out the disequalities in $k_2$ that contain offsets $z$ where $\lvert z \rvert > max\_offset(D_1)$.
Then we add each remaining proposition of $\F[k_1]$ to the kernel $k_2$,
as is done for the meet operation in \cref{section:meet}.

\begin{example}\todo{this is not a good example and it' wrong}
    If we consider the result of widen of \cref{ex:widen}, we assume that the result automaton $M$
    is, for example, a simple path from the equivalence class of $\&x$ to the equivalence class of $*x$.
    If we apply the narrowing operation to this automaton and the equality $\&x = *x$, this equality will
    be added to the automaton, as the terms $\&x$ and $*x$ are in the set of reachable terms of $M$.
    If on the other hand we apply the narrowing operation to the automaton and the equality $\&x = *(*x)$,
    this equality is ignored, as the term $*(*x)$ is not in the set of reachable terms of $M$.
\end{example}

