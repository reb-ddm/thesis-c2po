\chapter{Background}

The \cpo\ analysis described in this master's thesis is a static analysis that is based on abstract interprestation.
Before going into the details of the structure of the \cpo\ analysis,
it is useful to explain the key concepts of abstract interpretation and how it can be used
to design static program analyses.
This chapter also presents the \goblint\ abstract analyzer, where the \cpo\ analysis was implemented, and some of its analyzes that will later be
used by the \cpo\ analysis.

\section{Abstract Interpretation}
\todo[inline]{abstract semantics, transformer join ecc}


Abstract interpretation is a theoretical framework that is used to infer properties about programs
that are proven to be correct for all possible executions of the program.
It was first introduced by Patrick and Radhia Cousot in 1977 \cite{TODO} and is since then widely used
for program analysis and also other things. (TODO)
The analysis is done at compile-time (=statically), without having to execute the program.
This way it is also possible, for example, to prove the non-termination of a program.
The \cpo\ analysis described in this thesis, however, focuses on analyzing
relationships between pairs of pointer variables at each program point.

The analysis by abstract interpretation is constructed by replacing the concrete
semantics of a program with an abstract over-approximation of the program's behavior.
This abstract semantics is a simplified version of the concrete semantics
and can thus be computed in an efficient way.
Each step of the analysis needs to be proven proven to be a correct over-approximation of the concrete behaviour,
therefore the analysis is correct by construction.

\subsection{Concrete Semantics}

The program is represented by a control flow graph (CFG), where each node represents a program point and each edge represents a statement of the program, such as assignments, conditional branches and function calls.
A concrete state of the program is defined by functions that describe the values for each variable and each memory address.
\cref{fig:example-C-program} shows an example C program and \cref{ex:cfg} shows its corresponding CFG.

\begin{figure}
\begin{minted}{C}
main(){
  int x = 1;
  x = x + 1;
}
\end{minted}
\caption{An example of a C program.}
\label{fig:example-C-program}
\end{figure}
\begin{figure}
  \todo[inline]{CFG}
  \caption{The CFG of the program in \cref{fig:example-C-program}.}
  \label{ex:cfg}
\end{figure}

The concrete edge effect $\sem{l}$ describes the effect that an edge with label $l$ has on the concrete state.
For example in \cref{ex:cfg}, the concrete state at node $todo$ is $s_{todo} = \angl{x \rightarrow 1}$.
The state of node $todo2$ is equal to $\sem{x = x+1}\,s_{todo} = \angl{x \rightarrow 2}$.

\subsection{Abstract Semantics}

The goal of abstract interpretation is to over-approximate the concrete semantics at each node in the CFG.
This is done by choosing an abstract domain that can characterize specific properties of interest of the program.
For a node $p$ in the CFG; the abstract state is denoted as $X[p]$.
For example, the abstract domain of the \cpo\ analysis is a conjunction of equalities between pointers.
At each program point one such conjunction is computed, that over-approximates all possible concrete states at that point.
Another simpler example would be an abstract domain that remembers for each variable if it is even or odd.
An abstract state that only contains the variable $x$ is represented as $\angl{x \rightarrow \text{even}}$, $\angl{x \rightarrow \text{odd}}$ or $\angl{x \rightarrow \text{unknown}}$ if the parity of $x$ is not known.

The abstract edge effect $\sem{l}^\sharp$ describes the effect that an edge with label $l$ has on the abstract state.
For example, in the CFG in \cref{ex:cfg}, the abstract state at node $todo$ is $X[todo] = \angl{x \rightarrow \text{odd}}$.
After the edge with label $x = x+1$, the abstract state at node $todo2$ is $\sem{x = x+1}^\sharp\,X[todo] = \angl{x \rightarrow \text{even}}$.

The elements of the abstract domain must form a complete lattice, i.e., there is a partial order $\sqsubseteq$ between the elements and for each subset of elements there is a least upper bound (join, $\join$) and a greatest lower bound (meet, $\meet$).
Alternatively, if it is not possible to find the exact least upper bound, a sound over-approximation
can be used instead, without losing the correctness of the analysis.
A complete lattice also has a top element $\top$ and a bottom element $\bot$, which represent the least and greatest elements of the lattice, respectively.
During the analysis, the value $\top$ represents an abstract state that hold no information
about any variable, while $\bot$ represents a program point that is unreachable.

The concrete and the abstract semantics are related by the concretization function $\gamma$ and the abstraction function $\alpha$.
The concretization function maps an abstract element to the set of all concrete elements that are represented by the abstract element.
For example, the concretization of $\angl{x \rightarrow \text{even}}$ is the set of all concrete states where $x$ is an even number:
\[
\gamma(\angl{x \rightarrow \text{even}}) = \{\angl{x \rightarrow 0}, \angl{x \rightarrow 2}, \angl{x \rightarrow 4},\ldots\}
\]
The abstraction function maps a concrete element to the most precise abstract element that represent the concrete element. For example, $\alpha(\angl{x \rightarrow 0}) = \angl{x \rightarrow \text{even}}$.

The functions $\gamma$ and $\alpha$ form a Galois connection, that ensures the following property for each abstract state $a$ and each edge label $l$:
\[
\sem{l}\,\gamma(a) \subseteq \gamma(\sem{l}^\sharp\,a).
\]
This equality guarantees that the abstract edge effect of $l$ is an over-approximation
of the concrete edge effect.
Given that it holds for each $l$, we know that computing the semantics of a program in the abstract domain is a sound
over-approximation of the concrete semantics.

\subsection{Static Analysis by Abstract Interpretion}

This section describes more in detail the effects of each program statement on the abstract state.
\subsubsection{Assignment}
The abstract effect of an assignment $\sem{x = e}^\sharp$ computes the change in the abstract state when assigning the expression $e$ to the variable $x$.
For example, if $x$ and $e$ are pointers, the \cpo\ analysis removes all old equalities that were talking about $x$, as the vaue of $x$ might have changed.
Then it adds the equality $x = e$ to the abstract state.
\subsubsection{Conditionals}
When there is an if statement with the condition $c$, the CFG is split into two branches, one for the true branch ($Pos(c)$) and one for the false branch ($Neg(c)$).
The analysis can compute the abstract state $a_c$ corresponding to the condition and
then meet the current abstract state $a$ with $a_c$, resulting in the state $a \meet a_c$.
The same can be done for the false branch with the abstract state corresponding to $\neg c$.
For example, if the condition is $x = y$, the \cpo\ analysis adds the equality $x = y$
in the positive branch and the disequality $x \neq y$ in the negative branch.
\subsubsection{Loops}
For-loops and while-loops add backwards edges in the CFG.
The abstract value of a program point that has multiple incoming edges is
the least upper bound of the set of all abstract states of the incoming edges.
Formally, the abstract state $X[p]$ of a node $p$ is defined as
\[
X[p] \sqsupseteq \bigsqcup \{\sem{l}^\sharp\,X[p'] \mid (p,l,p') \text{ is an edge from $p$ to $p'$ in the CFG labeled with $l$}\}.
\]

\todo[inline]{fixpoint calculation,... therefore we need widening and narrowing}

\subsection{Widening and Narrowing}

Widening and narrowing are techniques used to ensure the termination of the abstract interpretation process.

- Widening (\(\triangledown\)): Used to accelerate the convergence of the fixpoint computation. It is applied at specific points to prevent infinite ascending chains in the abstract domain. Widening ensures that the iteration reaches a fixpoint in a finite number of steps.

  Formally, widening is a function \(\triangledown: A \times A \to A\) that satisfies:
  \[ \forall a_1, a_2 \in A: a_1 \leq_A a_1 \triangledown a_2 \quad \text{and} \quad a_2 \leq_A a_1 \triangledown a_2 \]

- Narrowing (\(\triangledown\)): Used to refine the result obtained by widening, making it more precise. Narrowing is applied after the widening phase to descend and reach a more precise fixpoint.

  Formally, narrowing is a function \(\triangledown: A \times A \to A\) that satisfies:
  \[ \forall a_1, a_2 \in A: a_1 \triangledown a_2 \leq_A a_1 \quad \text{and} \quad a_1 \triangledown a_2 \leq_A a_2 \]

By using widening and narrowing, abstract interpretation can provide sound and efficient approximations of the program's behavior.

\subsection{Interprocedural analysis}


Function calls are handled by the abstract enter function $\enter^\sharp$ and the abstract combine function $\combine^\sharp$.
The CFG adds an edge from the node before the function call to the entry point of the function and an edge from the exit point of the function to the node after the function call.
Let $s$ be the state before the function call.
The start state of the function is $\enter^\sharp\,s$.
Then the analysis is done on the function CFG.
Let the final state of the function be $s'$.
The state after the function call is $\combine^\sharp\,(s,s')$.

\todo[inline]{context sensitive

enter/combine}


\section{Goblint}\label{section:goblint}

The \goblint\ tool is an abstract interpretation framework for analyzing C programs.
It allows the definition of custom abstract domains by specifying the lattice operations and all abstract edge effects.
Then, \goblint\ can analyze C programs by transforming them into a set of CFGs, generating the system of inequalities arising from each edge, and solving these constraints using appropriate fixpoint algorithms.
There are already several abstract domains implemented in \goblint. These include relational and non-relational value analyses, concurrency and mutex analyses, as well as out-of-bounds access analyses.

The different analyses can communicate through \emph{queries}, which allows the combination of the results of different analyses, leading to an increase in the precision of each analysis.
The \cpo\ analysis utilizes information from two existing analyses of \goblint: the pointer analysis and the analysis of tainted variables.

\subsubsection{Pointer Analysis}

The pointer analysis in \goblint\ is a non-relational value analysis that tracks for each expression the set of possible addresses it can take.
The query \textsf{MayPointTo} determines the set of possible addresses an expression can refer to.
It is a may-analysis, meaning that the result is an over-approximation of the possible addresses, and it is impossible for the expression to point to an address that is not in the result of \textsf{MayPointTo}.
On the other hand, the \cpo\ analysis is a must-analysis, meaning that each proposition is definitely true for the current state. However, there might be even more true propositions that the analysis did not find.
The \textsf{MayPointTo} analysis complements the \cpo\ analysis by determining additional disequalities between terms.
Two terms are not equal if the \textsf{MayPointTo} sets of the two terms have an empty intersection.

\subsubsection{Tainted Variables Analysis}

An additional analysis used in the \cpo\ analysis is the tainted variables analysis.
Here, a \emph{tainted} variable is defined as a variable that was overwritten with a new value during the execution of a function.
The query \textsf{MayBeTainted} returns an over-approximation of the set of tainted variables in the current function.
\cpo\ uses this information to remove the tainted variables from the abstract state of the caller when returning from the function.
This analysis is necessary to remove only the changed values from the caller state, thus allowing us to keep the information about all the caller variables that were not modified during the function call.

