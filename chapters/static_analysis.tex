\chapter{Background}

\section{Abstract Interpretation}
\todo[inline]{abstract semantics, transformer join ecc}

Abstract interpretation is a theoretical framework for the static analysis of programs. It provides a means to reason about the behavior of programs by approximating their semantics. This section presents an overview of the main concepts involved in abstract interpretation, including lattices, abstract domains, concrete and abstract semantics, and the key operations used in this framework.

\subsection{Complete Lattice}

A lattice is a partially ordered set \((L, \leq)\) in which every pair of elements has a unique least upper bound (join, \(\sqcup\)) and a unique greatest lower bound (meet, \(\sqcap\)). A lattice is said to be complete if every subset \(S \subseteq L\) has both a least upper bound and a greatest lower bound in \(L\).

A complete lattice \((L, \leq, \sqcup, \sqcap, \top, \bot)\) is defined by:
- \(\top\) (top element): The least element in the lattice.
- \(\bot\) (bottom element): The greatest element in the lattice.
- Join (\(\sqcup\)): The least upper bound of a set of elements.
- Meet (\(\sqcap\)): The greatest lower bound of a set of elements.
- Partial order (\(\leq\)): A binary relation indicating the order between elements.

Formally, for any \(a, b \in L\):
\[ a \leq b \Leftrightarrow a \sqcup b = b \Leftrightarrow a \sqcap b = a \]

\subsection{Abstract and Concrete Domains}

In abstract interpretation, we work with two domains:
- The concrete domain \((C, \leq_C)\) represents the actual values or states of the program.
- The abstract domain \((A, \leq_A)\) is a simplification or abstraction of the concrete domain, designed to be finite and more tractable.

An abstract value is an element of the abstract domain that approximates one or more elements of the concrete domain.

\subsection{Concrete and Abstract Semantics}

The concrete semantics defines the exact behavior of the program, while the abstract semantics provides an approximation. Given a concrete semantic function \(F: C \to C\), the corresponding abstract semantic function \(F^\sharp: A \to A\) must satisfy:
\[ \gamma(F^\sharp(\alpha(c))) \supseteq F(c) \]
where \(\alpha: C \to A\) is the abstraction function and \(\gamma: A \to C\) is the concretization function.

\subsection{Abstract Transformers}

Abstract transformers are functions that approximate the effect of program operations in the abstract domain. Common abstract transformers include:
- Branch: Approximates conditional statements.
- Join: Approximates the merge of different execution paths. Defined as \(a_1 \sqcup a_2\).
- Assignment: Approximates variable assignment.
- Function Enter: Approximates the behavior when entering a function.
- Combine: Approximates the combination of different analysis results.

\subsection{Widening and Narrowing}

Widening and narrowing are techniques used to ensure the termination of the abstract interpretation process.

- Widening (\(\triangledown\)): Used to accelerate the convergence of the fixpoint computation. It is applied at specific points to prevent infinite ascending chains in the abstract domain. Widening ensures that the iteration reaches a fixpoint in a finite number of steps.

  Formally, widening is a function \(\triangledown: A \times A \to A\) that satisfies:
  \[ \forall a_1, a_2 \in A: a_1 \leq_A a_1 \triangledown a_2 \quad \text{and} \quad a_2 \leq_A a_1 \triangledown a_2 \]

- Narrowing (\(\triangledown\)): Used to refine the result obtained by widening, making it more precise. Narrowing is applied after the widening phase to descend and reach a more precise fixpoint.

  Formally, narrowing is a function \(\triangledown: A \times A \to A\) that satisfies:
  \[ \forall a_1, a_2 \in A: a_1 \triangledown a_2 \leq_A a_1 \quad \text{and} \quad a_1 \triangledown a_2 \leq_A a_2 \]

By using widening and narrowing, abstract interpretation can provide sound and efficient approximations of the program's behavior.

\subsection{Widening and Narrowing}

\subsection{Interprocedural analysis}

context sensitive
enter/combine


\section{Goblint}\label{section:goblint}

The \goblint\ tool is an abstract interpretation framework for analyzing C programs.
It allows the definition of custom abstract domains by specifying the lattice operations and all abstract edge effects.
Then, \goblint\ can analyze C programs by transforming them into a set of CFGs, generating the system of inequalities arising from each edge, and solving these constraints using appropriate fixpoint algorithms.
There are already several abstract domains implemented in \goblint. These include relational and non-relational value analyses, concurrency and mutex analyses, as well as out-of-bounds access analyses.

The different analyses can communicate through \emph{queries}, which allows the combination of the results of different analyses, leading to an increase in the precision of each analysis.
The \cpo\ analysis utilizes information from two existing analyses of \goblint: the pointer analysis and the analysis of tainted variables.

\subsubsection{Pointer Analysis}

The pointer analysis in \goblint\ is a non-relational value analysis that tracks for each expression the set of possible addresses it can take.
The query \textsf{MayPointTo} determines the set of possible addresses an expression can refer to.
It is a may-analysis, meaning that the result is an over-approximation of the possible addresses, and it is impossible for the expression to point to an address that is not in the result of \textsf{MayPointTo}.
On the other hand, the \cpo\ analysis is a must-analysis, meaning that each proposition is definitely true for the current state. However, there might be even more true propositions that the analysis did not find.
The \textsf{MayPointTo} analysis complements the \cpo\ analysis by determining additional disequalities between terms.
Two terms are not equal if the \textsf{MayPointTo} sets of the two terms have an empty intersection.

\subsubsection{Tainted Variables Analysis}

An additional analysis used in the \cpo\ analysis is the tainted variables analysis.
Here, a \emph{tainted} variable is defined as a variable that was overwritten with a new value during the execution of a function.
The query \textsf{MayBeTainted} returns an over-approximation of the set of tainted variables in the current function.
\cpo\ uses this information to remove the tainted variables from the abstract state of the caller when returning from the function.
This analysis is necessary to remove only the changed values from the caller state, thus allowing us to keep the information about all the caller variables that were not modified during the function call.

