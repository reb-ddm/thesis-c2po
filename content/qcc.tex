\section{Quantitative congruence closure}

\section{2-Pointer Logic}

In the following section, we describe the 2-pointer logic, which defines our new domain.
It is composed of finite conjunctions of propositions between terms.
It was first introduced by Seidl et al.\ in~\cite{2pointer} and we extend it here with an additional type of proposition.
The logic is made up of terms, formed from address constants and auxiliaries using the addition of constant offsets and dereferencing.
The address constants represent addresses of program variables.
The auxiliaries are variables whose address is never taken, therefore
we know of these variables that their address cannot be reached from other terms using address arithmetic or dereferencing.
The terms can form three types of propositions: Equalities and disequalities between terms and additionally a proposition which states that two terms do not belong to the same address block.
Let $\X$,$\A$ be disjoint finite sets of variables names and auxiliaries. The terms $t$ and propositions $p$ are defined by the following grammar:

\[
  \begin{array}{lll}
    t & {::=} & A \mid \&x \mid *(z+t)                                \\
    p & {::=} & t_1=z+t_2 \mid t_1\neq z+t_2\mid bl(t_1) \neq bl(t_2)
  \end{array}
\]

where $A\in \A$ is an auxiliary, $x \in \X$ is a variable with address $\&x$, $z \in \Z$ is an integer, $*$ denotes the dereferencing operator and $bl(t)$ represents the address block of $t$. We call terms $\&x$ or $A$ \emph{atoms}. We use $*t$ as an abbreviation for $*(0+t)$ and $x$ as an abbreviation for $*(0+\&x)$.
For a conjunction $\Psi$ of propositions, we denote the conjunctions of all the equalities in $\Psi$ by $\Psi_{=}$, the conjunction of disequalities between terms by $\Psi_{\neq}$ and the set of block disequalities of the type $bl(t_1) \neq bl(t_2)$ by $\Psi_{bl}$.

\subsection{Semantics}

We define the semantics of programs in a way that reflects the semantics of memory allocation in C.
In C, when a memory block is allocated with a library function such as \malloc, the pointer that is returned is only allowed to access the allocated memory block, and it can never access an address outside of this block.
(TODO cite C reference).
Therefore we modify the semantics of~\cite{2pointer} by interpreting the conjunctions $\Psi$ over a bi-dimensional address space.
Each address $(a, b) \in \ZZ$ belongs to a unique memory block $a$.
Intuitively, each address block belongs to a unique call to \malloc.

The semantics can be described by the following three functions: the mappings $\rho : \X \rightarrow \ZZ$ and $\nu : \mathcal{A} \rightarrow \ZZ$ that assign an address to each variable and auxiliary, respectively, and the map $\mu : \ZZ \rightarrow \ZZ$ which maps each address $(a, b)$ to the value $(a', b')$ which is stored at this address.
We are only interested in analyzing the values of pointers and addresses, therefore we interpret each value stored in the memory and in the auxiliaries as an address.

We also define the operator $(+) : \Z \rightarrow \ZZ \rightarrow \ZZ$ on addresses, where $z + (a,b) = (a, z+b)$.
It adds the offset only to the second element of the address, thus modeling the fact that it's not possible to leave a memory block using address arithmetic.
Moreover, we define the function $bl : \ZZ \rightarrow \Z$ that returns the address block identifier of an address, where $bl(a,b) = a$.

A term $t$ is interpreted as the value $\sem{t}(\rho, \nu, \mu)$ defined by:
\[
  \begin{array}{lll}
    \sem{\& x}\,(\rho,\nu,\mu)   & = & \rho\,x                          \\
    \sem{A}\,(\rho,\nu,\mu)      & = & \nu\,A                           \\
    \sem{*(z+t)}\,(\rho,\nu,\mu) & = & \mu\,(z+\sem{t}\,(\rho,\nu,\mu)) \\
  \end{array}
\]

The validity is defined as:
\[
  \begin{array}{ll}
    (\rho,\nu,\mu)\models t_1 = z+t_2          & \textbf{iff}
    \\ \multicolumn{2}{c}{\quad\quad\quad\quad\sem{t_1}\,(\rho,\nu,\mu) = z+\sem{t_2}\,(\rho,\nu,\mu)}	\\
    (\rho,\nu,\mu)\models t_1 \neq z+t_2       & \textbf{iff}
    \\ \multicolumn{2}{c}{\quad\quad\quad\quad\sem{t_1}\,(\rho,\nu,\mu) \neq z+\sem{t_2}\,(\rho,\nu,\mu)}	\\
    (\rho,\nu,\mu)\models bl(t_1) \neq bl(t_2) & \textbf{iff}
    \\ \multicolumn{2}{c}{\quad\quad\quad\quad bl(\sem{t_1}\,(\rho, \nu, \mu)) \neq bl(\sem{t_2}\,(\rho, \nu, \mu))}
  \end{array}
\]

If $(\rho, \nu, \mu)\models p$ for each proposition $p$ in $\Psi$, then we say that $(\rho, \nu, \mu) \models \Psi$. The domain of 2-Pointer Logic $\mathcal{P}_2[=]$ consists of all finite conjunctions
of propositions over terms up to semantic equivalence.
We define the concretization function $\gamma(\Psi)$ as the set of all $(\rho, \nu, \mu)$ with $(\rho, \nu, \mu) \models \Psi$.

\subsection{Quantitative Congruence Closure}\label{qcc}

In order to infer all equalities that are logically implied by a conjunction $\Psi$ of propositions, we compute the \emph{quantitative congruence closure} of the equalities in $\Psi$.
The representation of the quantitative congruence closure consists of a \emph{quantitative partition} $P = (\Pi,\tau,\omega)$ and a \emph{quantitative automaton} $M = (S, \eta, \delta)$, which will be described in this section.

Given that there can be an infinite number of equalities that follow from a certain conjunction $\Psi$, we only consider equalities that contain terms from a subterm-closed set $\T$ of \emph{interesting terms} that contains at least all the terms that occur in $\Psi$.

The \emph{quantitative partition} $(\Pi, \tau, \omega)$ consists of a partition $\Pi$ of $\T$, a function $\tau : \Pi \rightarrow \T$ that assigns a representative term to each equivalence class $Q \in \Pi$ and a mapping $\omega : \T \rightarrow \Z$, which assigns to each term its offset from the representative of its equivalence class.
We can derive the function $\otau : \T \rightarrow \T$ that maps each term to its representative, where $\otau t = \tau Q$ if $t \in Q$ for some $Q \in \Pi$.
These functions represent equalities of the form $t = \omega t + \otau t$.
For an equality $t_1 = z + t_2$, this is satisfied by a partition $(\Pi, \tau, \omega)$ iff $\otau t_1 = \otau t_2$ and $z = \omega t_1 - \omega t_2$.

This kind of partition can be represented in a similar fashion as a congruence closure data structure\cite{abstract-cc}, by additionally storing information about the integer offsets.
This is a dynamic data structure, which means that it is possible to add terms and equalities to the representation, such that the closure of equivalences is computed.
There exist multiple possible implementations of congruence closure\cite{cc-nelson, cc-shostak, cc-tarjan}.
In~\cite{2pointer} it is described how to implement the \emph{quantitative} version of it.
Here, we will assume that we can build a representation of the partition $(\Pi, \tau, \omega)$ starting from the conjunction $\Psi$, and that it is possible to add new terms and equalities to the representation.

The second part of the representation of our domain is the \emph{quantitative finite automaton} (QFA). It relates each equivalence class of the partition to the equivalence class that corresponds to the dereferenced terms.
It is defined by a triple $(S, \eta, \delta)$, where $S$ is a finite set of states, where each state represents an equivalence class from $\Pi$, $\eta : (\mathcal{A} \cup \{\&x | x \in \X\} \rightarrow \Z \times S)$ is a partial mapping that provides initial offsets and states for atoms, and $\delta : \Z \rightarrow S \rightarrow \Z \times S$ is the partial transition function.

We can derive the automaton directly from the partition $(\Pi, \tau, \omega)$ by defining $M[\Psi,\T] = (S, \eta, \delta)$ where $S = \{s_i | Q_i \in \Pi\}$, $\eta a = (\omega a, s_i)$ if $a \in Q_i$ for some $Q_i \in \Pi$, and $\delta z s_i = (\omega t', s_j)$ if there is a $t' \in Q_j$ with $Q_j \in \Pi$, such that $t' = *(z_1 + t_1)$ with $t_1 \in Q_i$ and $z = z_1 + (\omega t_1)$.
We remark that a transition $\delta z s_i$ could derive not only from a single term $t'$ but also from a second term $*(z_2 + t_2) \in \T$ with $t_2 \in Q_i$ and $z = z_2 + (\omega t_2)$.
Since $t_i \equiv (\omega t_i) + (\tau Q_i)$ for $i = 1,2$, it follows that $t_1 \equiv (z_2 - z_1)+ t_2$. Therefore $*(z_2+t_2)\in Q_j$ with offset $\omega(*(z_2+t_2)) = \omega t'$.
This shows that the result of $\delta z s_i$ is well defined and doesn't depend on which term $t'$ we choose for deriving a transition.
Additionally, $\delta$ is defined only for a finite amount of values, given that each term $*(z + t) \in \T$ defines at most one mapping for $\delta$.

We use $\oT$ to denote the set of \emph{all} terms with variable names from $\X$ and auxiliaries from $\A$.
We can extend the mappings $\eta$,$\delta$ to a partial mapping $M : \oT \rightarrow \Z \times S$ where $M[a] = \eta(a)$ for atoms $a$, and $M[*(z+t_1)] = \delta(z+z_1, s)$ for terms $t_1$ if $M[t_1] = (z_1,s)$.
We define the action $(+): \Z \rightarrow (\Z \times S) \rightarrow (\Z \times S)$ where $z_1 + (z_2,s) = (z_1 + z_2, s)$.
We define $M[\Psi] = M[\Psi,\T_\Psi]$, where $\T_\Psi$ is the set of terms and subterms occurring in $\Psi$.

We also define the set $\L(M)$ of terms $t \in \oT$ for which $M$ is defined.
For each state $s \in S$ we define the set $\L_M(s)$ of terms $t \in \oT$ for which it holds that $M[t] = (z, s)$ for some $z \in \Z$.
\begin{theorem}
  % This is Theorem 3 in the other paper
  Assume that $\Psi$ is a satisfiable conjunction of equalities, let $\T$ be a subterm-closed set of terms which contains all terms occuring in $\Psi$. We can construct a corresponding quantitative partition $(\Pi, \tau,\omega)$
  and a corresponding QFA $M = (S, \eta, \delta)$ such that for every $t_1, t_2 \in \L(M)$, the following statements are equivalent:
  \begin{enumerate}
    \item $M[t_1] = z + M[t_2]$
    \item $\Psi$ implies $(t_1 = z + t_2)$
  \end{enumerate}
\end{theorem}

This theorem was proven in~\cite{2pointer} for the case of a one-dimensional memory model.
The proof can easily be adapted for the two-dimensional memory model.

We remark that the only equalities that can be derived from $\Psi$ are derived from $\Psi_=$. We cannot derive equalities from any of the disequalities.

\subsection{Disequalities}\label{disequalities}

In order to implement a more precise abstract transformer for the assignment, it is important to know which terms definitely
point to different addresses. Let $\Psi$ be a conjunction of 2-pointer logic terms.
We define the binary relation $\nequivp$ which represents all disequalities that can be derived from $\Psi$.

\begin{enumerate}
  \item[(D0)] If $t_1 \neq z + t_2 \in \Psi$, then $t_1 \nequivp z + t_2$.
  \item[(D1)] If $t_1 \nequivp z + t_2$, then:
    \begin{enumerate}
      \item $t_2 \nequivp - z + t_1$.
      \item $t'_1\nequivp (z_1 + z) + t_2$ if $t_1' = z_1 + t_1$ is implied by $\Psi$.
      \item If $t_1 = z + t_2$ is implied by $\Psi$, then $\Psi$ is Unsat.
    \end{enumerate}
  \item[(D2)] If $*(z_1 + t_1) \nequivp *(z_2 + t_2)$ then also $t_1 \nequivp (z_2 - z_1) + t_2$.
  \item[(D3)] If $t_1 = z' + t_2$ is implied by $\Psi$ then $t_1 \nequivp z + t_2$ for all $z \neq z'$.
  \item[(D4)] If $bl(t_1) \neq bl(t_2) \in \Psi$, then:
    \begin{enumerate}
      \item $t_1 \nequivp z + t_2$ for all $z \in \Z$.
      \item If $t_1 = z + t_2$ for any $z \in \Z$, then $\Psi$ is Unsat.
    \end{enumerate}
\end{enumerate}

The set of disequalities following from (D3) and (D4) can be infinite, but we don't need to explicitly store them, as they implicitly derive from the congruence closure of $\Psi_=$ or respectively from the propositions in $\Psi_{bl}$.
We construct the set of all disequalities and block disequalities deriving from (D0), (D1), (D2) and (D4),but only those that contain representative terms of components $Q\in\Pi$.
This set is finite, because we only store equalities between different representatives and not the implicit equalities following from (D3) and (D4).
We denote these explicitly stored disequalities as $closure_\neq(\Psi)$.

This set can be computed in polynomial time in the size of the formula $\Psi$ by going through all the
disequalities $t_1 \neq z + t_2$ in $\Psi_\neq$ and storing the disequalities
$\otau t_1  \neq (\omega t_2 - \omega t_1 + z) + \otau t_2$ and $\otau t_2 \neq - (\omega t_2 - \omega t_1 + z) + \otau t_1$.
Then for each disequality of the type $*(z_1 + t_1) \nequivp *(z_2 + t_2)$ that follows from a stored disequality or from the congruence closure of $\Psi_=$ or from a block disequality,
the disequalities propagate backwards using the rule (D2).
This only adds a finite amount of disequalities to the set, given that each equality and each block disequality implies at most one disequality of the type $*(z_1 + t_1) \nequivp *(z_2 + t_2)$.

For each block disequality $bl(t_1) \neq bl(t_2) \in \Psi$, we store the disequalities $bl(\otau t_1) \neq bl(\otau t_2)$.
We call the set of these disequalities $closure_{bl}(\Psi)$.

\begin{lemma}\label{lemma:diseq_types}
  $t_1 \nequivp t_2 + z$ iff one of the following holds:

  \begin{enumerate}
    \item $\otau t_1 \neq (\omega t_1 - \omega t_2 + z) + \otau t_2 \in closure_\neq(\Psi)$.
    \item Either $bl(\otau t_1) \neq bl(\otau t_2) \in closure_{bl}(\Psi)$ or $bl(\otau t_2) \neq bl(\otau t_1) \in closure_{bl}(\Psi)$.
    \item $t_1 = z' + t_2$ for any $z' \neq z$ is implied by $\Psi$.
  \end{enumerate}
  We call the disequalities that follow from 1.\ the \emph{explicit} disequalities and those following from 2.\ and 3.\ the \emph{implicit} disequalities.
\end{lemma}
