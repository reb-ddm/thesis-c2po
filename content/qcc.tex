\section{Quantitative congruence closure}

Given a conjunction $\Psi$ of propositions, the quantitative congruence closure infers all equalities that are logically implied by $\Psi$.
As there are infinitely many such equalities, we consider only the equalities that contain terms from a specific set $\T$ of terms.
The set $\T$ must be subterm-closed and contain at least all the terms that occur in $\Psi$.
The equalities implied by $\Psi$ are described by the quantitative equivalence relation $\equivp$,
which is the smallest equivalence relation $\equivp$ that satisfies these rules:
\begin{enumerate}[label={[E\arabic*]}, ref={[E\arabic*]}]
  \setcounter{enumi}{-1}
\item\label{item:persistence} If $t_1 = z+t_2$ occurs in $\Psi$, then $t_1 \equivp z+t_2$;
\item\label{item:quantitative-reflexivity} $t_1 \equivp 0+t_1$ for all $t\in\T$(\emph{quantitative reflexivity});
\item\label{item:quantitative-symmetry} If $t_1 \equivp z+t_2$, then $t_2 \equivp -z+t_1$ (\emph{quantitative symmetry});
\item\label{item:quantitative-transitivity} If $t_1 \equivp z_1+t_2$ and $t_2 \equivp z_2 + t_3$,
		then $t_1 \equivp (z_1+z_2)+t_3$ (\emph{quantitative transitivity});
\item\label{item:dereferencing} If $t_1 \equivp (z_2 - z_1) + t_2$, then $*(z_1 + t_1) \equivp *(z_2 + t_2)$ holds as well, whenever $*(z_1 + t_1),  *(z_2 + t_2)$ are in $\T$ (\emph{dereferencing}).
\end{enumerate}

\subsection{Quantitative union-find}

We can represent the quantitative equivalence relation $\equivp$ by using a data structure similar to union-find, but extended with integer offsets.
The \emph{quantitative union-find} $(\Pi, \tau, \omega)$ consists of a partition $\Pi$ of $\T$,
a function $\tau : \Pi \rightarrow \T$ that assigns a representative term to each equivalence class $Q \in \Pi$ and a mapping $\omega : \T \rightarrow \Z$,
which assigns to each term its offset from the representative of its equivalence class.
A set $Q \in \Pi$ contains all terms that are equivalent up to an integer offset.

There exist efficient algorithms for the union-find data structure,
in particular the well-known representation proposed by~\textcite{uf-tarjan},
which we will present in this section.
We extend this approach to handle integer offsets as well.

The union-find partition is represented as an acyclic directed weighted graph,
where each node represents a term $t \in \T$ and
each connected component is a tree, which represents an equivalence class $Q \in \Pi$.
Each tree has a root, that is the representative $\tau Q$ of the equivalence class $Q$.
All the edges in the tree are directed towards the root.
Each edge from $t_1$ to $t_2$ is weighted with an integer $z$,
which means that $t_1 \equivp z + t_2$.
$\omega t$ can be computed by following the path from $t$ to the root of the tree and summing up the weights of the edges.
The \find{t} operation returns the pair $(\otau t, \omega t)$ for a given term $t$.

The union-find data structure is initialized with a partition $\Pi$ that contains all the terms from $\T$ as singletons.
The corresponding graph contains a node for each $t \in \T$ and no edges.
Then for each equality $t_1 = z + t_2$ in $\Psi$, we perform the \union{t_1}{t_2}{z} operation, which merges two trees by adding an edge from $\tau\,t_1$ to $\tau\,t_2$ with weight $z - \omega\,t_1 + \omega\,t_2$.

\todo{Is it worth it to mention path compression and union by rank?}

\begin{example}
Consider the conjunction $\Psi \equiv A = 1 + B \land \&x = 2 + D \land  = 3 + A$.
\end{example}

For an equality $t_1 = z + t_2$, this is satisfied by a partition $(\Pi, \tau, \omega)$ iff $\otau t_1 = \otau t_2$ and $z = \omega t_1 - \omega t_2$.
We can derive the function $\otau : \T \rightarrow \T$ that maps each term to its representative, where $\otau t = \tau Q$ if $t \in Q$ for some $Q \in \Pi$.
The partition is constructed in such a way that for each $t \in \T$,
the equality $t \equivp \omega t + \otau t$ is derivable from the rules above.



This kind of partition can be represented in a similar fashion as a congruence closure data structure\cite{abstract-cc}, by additionally storing information about the integer offsets.
This is a dynamic data structure, which means that it is possible to add terms and equalities to the representation, such that the closure of equivalences is computed.
There exist multiple possible implementations of congruence closure\cite{cc-nelson, cc-shostak, cc-tarjan}.
In~\cite{2pointer} it is described how to implement the \emph{quantitative} version of it.
Here, we will assume that we can build a representation of the partition $(\Pi, \tau, \omega)$ starting from the conjunction $\Psi$, and that it is possible to add new terms and equalities to the representation.

The second part of the representation of our domain is the \emph{quantitative finite automaton} (QFA). It relates each equivalence class of the partition to the equivalence class that corresponds to the dereferenced terms.
It is defined by a triple $(S, \eta, \delta)$, where $S$ is a finite set of states, where each state represents an equivalence class from $\Pi$, $\eta : (\mathcal{A} \cup \{\&x | x \in \X\} \rightarrow \Z \times S)$ is a partial mapping that provides initial offsets and states for atoms, and $\delta : \Z \rightarrow S \rightarrow \Z \times S$ is the partial transition function.

We can derive the automaton directly from the partition $(\Pi, \tau, \omega)$ by defining $M[\Psi,\T] = (S, \eta, \delta)$ where $S = \{s_i | Q_i \in \Pi\}$, $\eta a = (\omega a, s_i)$ if $a \in Q_i$ for some $Q_i \in \Pi$, and $\delta z s_i = (\omega t', s_j)$ if there is a $t' \in Q_j$ with $Q_j \in \Pi$, such that $t' = *(z_1 + t_1)$ with $t_1 \in Q_i$ and $z = z_1 + (\omega t_1)$.
We remark that a transition $\delta z s_i$ could derive not only from a single term $t'$ but also from a second term $*(z_2 + t_2) \in \T$ with $t_2 \in Q_i$ and $z = z_2 + (\omega t_2)$.
Since $t_i \equiv (\omega t_i) + (\tau Q_i)$ for $i = 1,2$, it follows that $t_1 \equiv (z_2 - z_1)+ t_2$. Therefore $*(z_2+t_2)\in Q_j$ with offset $\omega(*(z_2+t_2)) = \omega t'$.
This shows that the result of $\delta z s_i$ is well defined and doesn't depend on which term $t'$ we choose for deriving a transition.
Additionally, $\delta$ is defined only for a finite amount of values, given that each term $*(z + t) \in \T$ defines at most one mapping for $\delta$.

We use $\oT$ to denote the set of \emph{all} terms with variable names from $\X$ and auxiliaries from $\A$.
We can extend the mappings $\eta$,$\delta$ to a partial mapping $M : \oT \rightarrow \Z \times S$ where $M[a] = \eta(a)$ for atoms $a$, and $M[*(z+t_1)] = \delta(z+z_1, s)$ for terms $t_1$ if $M[t_1] = (z_1,s)$.
We define the action $(+): \Z \rightarrow (\Z \times S) \rightarrow (\Z \times S)$ where $z_1 + (z_2,s) = (z_1 + z_2, s)$.
We define $M[\Psi] = M[\Psi,\T_\Psi]$, where $\T_\Psi$ is the set of terms and subterms occurring in $\Psi$.

We also define the set $\L(M)$ of terms $t \in \oT$ for which $M$ is defined.
For each state $s \in S$ we define the set $\L_M(s)$ of terms $t \in \oT$ for which it holds that $M[t] = (z, s)$ for some $z \in \Z$.
\begin{theorem}
  % This is Theorem 3 in the other paper
  Assume that $\Psi$ is a satisfiable conjunction of equalities, let $\T$ be a subterm-closed set of terms which contains all terms occuring in $\Psi$. We can construct a corresponding quantitative partition $(\Pi, \tau,\omega)$
  and a corresponding QFA $M = (S, \eta, \delta)$ such that for every $t_1, t_2 \in \L(M)$, the following statements are equivalent:
  \begin{enumerate}
    \item $M[t_1] = z + M[t_2]$
    \item $\Psi$ implies $(t_1 = z + t_2)$
  \end{enumerate}
\end{theorem}

This theorem was proven in~\cite{2pointer} for the case of a one-dimensional memory model.
The proof can easily be adapted for the two-dimensional memory model.

We remark that the only equalities that can be derived from $\Psi$ are derived from $\Psi_=$. We cannot derive equalities from any of the disequalities.

\subsection{Disequalities}\label{disequalities}

In order to implement a more precise abstract transformer for the assignment, it is important to know which terms definitely
point to different addresses. Let $\Psi$ be a conjunction of 2-pointer logic terms.
We define the binary relation $\nequivp$ which represents all disequalities that can be derived from $\Psi$.

\begin{enumerate}
  \item[(D0)] If $t_1 \neq z + t_2 \in \Psi$, then $t_1 \nequivp z + t_2$.
  \item[(D1)] If $t_1 \nequivp z + t_2$, then:
    \begin{enumerate}
      \item $t_2 \nequivp - z + t_1$.
      \item $t'_1\nequivp (z_1 + z) + t_2$ if $t_1' = z_1 + t_1$ is implied by $\Psi$.
      \item If $t_1 = z + t_2$ is implied by $\Psi$, then $\Psi$ is Unsat.
    \end{enumerate}
  \item[(D2)] If $*(z_1 + t_1) \nequivp *(z_2 + t_2)$ then also $t_1 \nequivp (z_2 - z_1) + t_2$.
  \item[(D3)] If $t_1 = z' + t_2$ is implied by $\Psi$ then $t_1 \nequivp z + t_2$ for all $z \neq z'$.
  \item[(D4)] If $bl(t_1) \neq bl(t_2) \in \Psi$, then:
    \begin{enumerate}
      \item $t_1 \nequivp z + t_2$ for all $z \in \Z$.
      \item If $t_1 = z + t_2$ for any $z \in \Z$, then $\Psi$ is Unsat.
    \end{enumerate}
\end{enumerate}

The set of disequalities following from (D3) and (D4) can be infinite, but we don't need to explicitly store them, as they implicitly derive from the congruence closure of $\Psi_=$ or respectively from the propositions in $\Psi_{bl}$.
We construct the set of all disequalities and block disequalities deriving from (D0), (D1), (D2) and (D4),but only those that contain representative terms of components $Q\in\Pi$.
This set is finite, because we only store equalities between different representatives and not the implicit equalities following from (D3) and (D4).
We denote these explicitly stored disequalities as $closure_\neq(\Psi)$.

This set can be computed in polynomial time in the size of the formula $\Psi$ by going through all the
disequalities $t_1 \neq z + t_2$ in $\Psi_\neq$ and storing the disequalities
$\otau t_1  \neq (\omega t_2 - \omega t_1 + z) + \otau t_2$ and $\otau t_2 \neq - (\omega t_2 - \omega t_1 + z) + \otau t_1$.
Then for each disequality of the type $*(z_1 + t_1) \nequivp *(z_2 + t_2)$ that follows from a stored disequality or from the congruence closure of $\Psi_=$ or from a block disequality,
the disequalities propagate backwards using the rule (D2).
This only adds a finite amount of disequalities to the set, given that each equality and each block disequality implies at most one disequality of the type $*(z_1 + t_1) \nequivp *(z_2 + t_2)$.

For each block disequality $bl(t_1) \neq bl(t_2) \in \Psi$, we store the disequalities $bl(\otau t_1) \neq bl(\otau t_2)$.
We call the set of these disequalities $closure_{bl}(\Psi)$.

\begin{lemma}\label{lemma:diseq_types}
  $t_1 \nequivp t_2 + z$ iff one of the following holds:

  \begin{enumerate}
    \item $\otau t_1 \neq (\omega t_1 - \omega t_2 + z) + \otau t_2 \in closure_\neq(\Psi)$.
    \item Either $bl(\otau t_1) \neq bl(\otau t_2) \in closure_{bl}(\Psi)$ or $bl(\otau t_2) \neq bl(\otau t_1) \in closure_{bl}(\Psi)$.
    \item $t_1 = z' + t_2$ for any $z' \neq z$ is implied by $\Psi$.
  \end{enumerate}
  We call the disequalities that follow from 1.\ the \emph{explicit} disequalities and those following from 2.\ and 3.\ the \emph{implicit} disequalities.
\end{lemma}
