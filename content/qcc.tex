\section{Quantitative congruence closure}\label{chapter:qcc}

Given a conjunction $\Psi$ of propositions, the quantitative congruence closure infers all equalities that are logically implied by $\Psi$.
As there are infinitely many such equalities, we consider only the equalities that contain terms from a specific set $\T$ of terms.
The set $\T$ must be subterm-closed and contain at least all the terms that occur in $\Psi$.
The equalities implied by $\Psi$ are described by the quantitative equivalence relation $\equivp$,
which is the smallest equivalence relation $\equivp$ that satisfies these rules:
\begin{enumerate}[label={[E\arabic*]}, ref={[E\arabic*]}]
  \setcounter{enumi}{-1}
\item\label{item:persistence} If $t_1 = z+t_2$ occurs in $\Psi$, then $t_1 \equivp z+t_2$;
\item\label{item:quantitative-reflexivity} $t_1 \equivp 0+t_1$ for all $t\in\T$ (\emph{quantitative reflexivity});
\item\label{item:quantitative-symmetry} If $t_1 \equivp z+t_2$, then $t_2 \equivp -z+t_1$ (\emph{quantitative symmetry});
\item\label{item:quantitative-transitivity} If $t_1 \equivp z_1+t_2$ and $t_2 \equivp z_2 + t_3$,
		then $t_1 \equivp (z_1+z_2)+t_3$ (\emph{quantitative transitivity});
\item\label{item:dereferencing} If $t_1 \equivp (z_2 - z_1) + t_2$, then $*(z_1 + t_1) \equivp *(z_2 + t_2)$ holds as well, whenever $*(z_1 + t_1),  *(z_2 + t_2)$ are in $\T$ (\emph{dereferencing}).
\end{enumerate}

\subsection{Quantitative partition}\label{subsection:quantitative-union-find}

We can represent the quantitative equivalence relation $\equivp$ as a partition.
The \emph{quantitative partition} $P = (\T, \tau, \omega)$ consists of a set of terms $\T$,
a function $\tau : \T \rightarrow \T$ that assigns a representative term to each term $t \in \T$ and a mapping $\omega : \T \rightarrow \Z$,
which assigns to each term its offset from the representative of its equivalence class.
This mean that for each $t \in \T$, it holds that $t \equivp \omega\,t + \tau\,t$.
All terms that have the same representative belong to the same equivalence class, i.e., they are equivalent up to an integer offset.
A representative term is a term $t \in \T$ that occurs on the right-hand side of the function $\tau$,
i.e., there exists a $t'\in \T$ such that $\tau\,t' = t$.
For each representative term $t$, it holds that $\tau\,t=t$ and $\omega\,t = 0$.

\subsubsection{Implementation}

This partition is represented by using a data structure similar to union-find, but extended with integer offsets.
The union-find data structure can be represented as a forest of trees,
where each tree represents an equivalence class.
Each node in the tree corresponds to a term $t \in \T$, and each tree is rooted at a representative element $\tau\,t$,
which serves as the identifier for the entire set.\cite{uf-tarjan}
All the edges in the tree are directed towards the root.
In order to represent the integer offset, we store a weight for each edge.

The forest is represented as a map $parent : \T \rightarrow \Z \times \T$ that maps each
term $t$ to its parent node in the tree.
The nodes that have themselves as a parent are the representative terms.
If $parent(t_1) = (z,t_2)$, then there is an edge from $t_1$ to $t_2$ with weight $z$
in the graph and it holds that $t_1 \equivp z + t_2$.

The $\find{t}$ operation returns the pair $(\omega\,t,\tau\,t)$.
The offset $\omega t$ can be computed by following the path from $t$ to the root of the tree and summing up the weights of the edges.
The representative $\tau\,t$ is the root of the tree.

Initially, each term $t\in\T$ is its own parent, and equalities $t_1 = z + t_2$ between terms can be added by performing the
$\union{t_1}{z}{t_2}$ operation.
This operation merges two trees by adding an edge from $\tau\,t_1$ to $\tau\,t_2$ with weight $z - \omega\,t_1 + \omega\,t_2$,
thus merging the two equivalence classes of $t_1$ and $t_2$.
If $\tau\,t_1 = \tau\,t_2$, but $z - \omega\,t_1 + \omega\,t_2 \neq 0$, then the conjunction is
unsatisfiable.

\begin{example}
Consider the conjunction $\Psi \equiv B = 1 + A \land D = 2 + \&x \land A = 3 + \&x$.
The union-find algorithm computes the following tree after having performed the three union operations:

\begin{tikzpicture}[->, node distance=2cm, auto]
    \node (x) {$\&x$};
    \node (A) [below left of=x] {$A$};
    \node (D) [below right of=x] {$D$};
    \node (B) [below=1cm of A] {$B$};

    \path (A) edge node {3} (x);
    \path (B) edge node {1} (A);
    \path (D) edge node {2} (x);
\end{tikzpicture}

From this representation, we can for example derive that $B = 4 + \&x$.
Depending on the choice of the representatives, the tree can look different.
\end{example}

In a practical implementation, the representatives are chosen in such a way that the trees have the lowest possible height,
thus making the $\find{t}$ operation more efficient.
This optimization is called \emph{union by rank} and consists in adding the edge from
the smallest to the biggest equivalence class.
Another possible optimization is to set the parent node of a term to the representative of the tree,
each time the $\find{t}$ operation is traverses the node.
This way, the tree is flattened, which makes the $\find{t}$ operation
more efficient if we call it again for the same term $t$ or a term in the same equivalence class.
This is the so-called \emph{path compression}.~\cite{uf-tarjan}

\subsection{Quantitative finite automaton}\label{subsection:qfa}

A partition $P = (\T, \tau, \omega)$ is useful for representing equalities between atoms
with an addition of integer offsets and it computes the closure under the rules
\labelcref{item:persistence,item:quantitative-reflexivity,item:quantitative-symmetry,item:quantitative-transitivity}
by construction.
In order to take into account the dereferencing operation, we need to
keep track for each term $t$, which equivalence classes contain terms of the form $*(z+t)$.
This way, we can efficiently compute the equalities deriving from rule \labelcref{item:dereferencing}.
For example if $\T = \{A, B, *A, *B\}$ and we perform a union of $A$ and $B$, the
closure rule~\labelcref{item:dereferencing} tells us, that also the equivalence classes of $*A$ and $*B$ should be merged.
Therefore, we introduce an alternative representation of the partition $P$ as a \emph{quantitative finite automaton} (QFA) $M_P$.

The QFA is defined by a triple $M_P = (S, \otau, \eta, \delta)$, where $S$ is a finite set of states, where each state represents an equivalence class of the partition.
In order to know which state represents which equivalence class, the mapping $\otau : S \rightarrow \T$
assigns a representative term to each state.
The partial mapping $\eta : (\mathcal{A} \cup \{\&x | x \in \X\} \rightarrow \Z \times S)$ provides initial offsets and states for atoms, meaning that if $\eta\,a = (z,s)$, then $a$ is in the equivalence class represented by $s$.
$\delta : \Z \rightarrow S \rightarrow \Z \times S$ is the partial transition function.
Intuitively, if $\delta\,z_1\,s_1 = (z_2, s_2)$, this means that it holds that $*(z_1 + \otau\,s_1) \equivp z_2 + \otau\,s_2$.

The QFA is constructed from the partition  $(\T, \tau, \omega)$ by defining $M_P = (S, \otau, \eta, \delta)$ where $S$ contains a state $s$ for each representative term $t$ of $P$ and we set $\otau\,s = t$.
$\eta\,a = (\omega\,a, s_1)$ if $s_1 \in S$ and $\otau\,s = \tau\,a$,
and $\delta\,z\,s_1 = (\omega\,t', s_2)$ if there is a $\tau\,t'=\otau\,s_2$, such that $t' \equiv *(z_1 + t_1)$ with $\tau \,t_1= \otau\,s_1$ and $z = z_1 + \omega\,t_1$.

We remark that a transition $\delta\,z\,s_1$ could derive not only from a single term $t'$ but also from a second term $*(z_2 + t_2) \in \T$ with $\tau\,t_2=\otau\,s_1$ and $z = z_2 + \omega\,t_2$.
Since $t_i \equivp (\omega t_i) + (\tau Q_i)$ for $i = 1,2$, it follows that $t_1 \equivp (z_2 - z_1)+ t_2$.
Therefore $\tau\,*(z_2+t_2)=\otau\,s_2$ with offset $\omega(*(z_2+t_2)) = \omega\,t'$.
This shows that the result of $\delta\,z\,s_1$ is well defined and doesn't depend on which term $t'$ we choose for deriving a transition.
Additionally, $\delta$ is defined only for a finite amount of values, given that each term $*(z + t) \in \T$ defines at most one mapping for $\delta$.

We use $\oT$ to denote the set of \emph{all} terms with variable names from $\X$ and auxiliaries from $\A$.
We can extend the mappings $\eta$ and $\delta$ to a partial mapping $M : \oT \rightarrow \Z \times S$ where $M_P[a] = \eta(a)$ for atoms $a$, and $M_P[*(z+t_1)] = \delta(z+z_1, s)$ for terms $t_1$ if $M_P[t_1] = (z_1,s)$.

Additionally, we define the set $\L(M_P)$ of terms $t \in \oT$ for which $M$ is defined.
For each state $s \in S$, the set $\L_M(s)$ is the set of terms $t \in \oT$ for which it holds that $M_P[t] = (z, s)$ for some $z \in \Z$.

Using the automata, the $\closure{t_1}{z}{t_2}$\todo{add result type and $P$ and $M_P$ to the parameters}  operation performs the union of $t_1$ and $t_2$ and
then computes the closure using the rule \labelcref{item:dereferencing}.
It is defined as follows:

\begin{enumerate}
  \item Case 1: $\tau\,t_1 = \tau\,t_2$. If $\omega\,t_1 = z + \omega\,t_2$ we are done.
  If $\omega\,t_1 \neq z + \omega\,t_2$, then the conjunction is unsatisfiable.
  \item Case 2: $\tau\,t_1 \neq \tau\,t_2$. We call $\union{t_1}{t_2}{z}$.
  Now either $t_1$ or $t_2$ has a new representative, so the transitions of the QFA are updated accordingly.
  Let $t$ be the term between $t_1$ and $t_2$ that has a new representative,
  and let $P = (\T, \tau, \omega)$ and $M_P = (S, \otau, \eta, \delta)$ be the partition and qutomaton before the union operation
  and $P' = (\T, \tau', \omega')$ and $M_{P'} = (S', \otau', \eta', \delta')$ be the partition after the union operation.
  For each outgoing transition $\delta(z_1, s_1) = \delta(z_2, s_2)$ where $t \in \L(s_1)$,
  we set $\delta'(z_1 - \omega\,(\tau\,t), s_1) = \delta'(z_2, s_2)$.
  \todo[inline]{Then use merge for the successors.}
  \todo{change $z_2$ in incoming transitions? how deep should I go in the details?}
\end{enumerate}

\todo[inline]{This is the same as congruence closure (cite) but restricted to a unary uninterpreted function symbol $*$ and extended with integer offsets as in (cite Oliveras Nieuwenhuis).}

\begin{theorem}
  % This is Theorem 3 in the other paper
  Assume that $\Psi$ is a satisfiable conjunction of equalities, let $\T$ be a subterm-closed set of terms which contains all terms occuring in $\Psi$.
  The corresponding quantitative partition $(\T, \tau,\omega)$ and a corresponding QFA $M_P = (S, \otau, \eta, \delta)$ are constructed by applying the closure operation with all equalities in $\Psi$.
  Then $\T \subseteq \L(M_P)$ and for every $t_1, t_2 \in \L(M_P)$ the following statements are equivalent:
  \begin{enumerate}
    \item $M_P[t_1] = z + M_P[t_2]$,
    \item $\tau\,t_1 = \tau\,t_2$ and $\omega\,t_1 = z + \omega\,t_2$,
    \item $\Psi$ implies $(t_1 = z + t_2)$.
  \end{enumerate}
\end{theorem}

This theorem was proven in~\cite{2pointer} for the case of a one-dimensional memory model.
The proof can easily be adapted for the two-dimensional memory model.

We remark that the only equalities that can be derived from $\Psi$ are derived from $\Psi_=$. We cannot derive equalities from any of the disequalities.
