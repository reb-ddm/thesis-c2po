

\section{Representation of C Expressions as 2-Pointer Logic Terms}

Since the \cpo\ analysis only considers C expressions with pointer types, they have a uniform size.
For instance, on an x86-64 architecture, pointers occupy 64 bits in the memory.

However, the offsets added to pointers vary depending on the pointer's type.
For example, if we have a pointer $p$ of type $int32_t\,*p$, the expression $p + z$ represents an offset of $32 \cdot z$ bits.
If the same pointer $p$ is cast to a pointer of type $int64_t\,*p$, the offset $z$ is of $64 \cdot z$ bits.
It is essential to consider this to know the relative positioning of
different expressions in the memory and to discover possible overlapping data.
Therefore, in our implementation, the offsets in the abstract terms representing C expressions are expressed in bits rather than the value present in the code.
For instance, if $p$ is a pointer to a 32-bit integer and $q$ is a pointer to a 64-bit integer, then the analysis represents the C expression $*(p + 1)$ as $*(p + 32)$, and $*(q + 1)$ as $*(q + 64)$.

In some cases, it is impossible to determine the offset of a pointer expression, e.g., if the offset is an unknown variable instead of a constant.
Here, the \goblint\ constant propagation analysis is used to infer the offset value.
If the value remains indeterminable, the expression is treated as unknown.

Furthermore, arrays and structs are handled similarly to pointers in C, allowing us to express properties not only about pointers but also about arrays and structs.
For example, in C, the expression $a[1]$ is equivalent to the expression $*(a + 1)$.
Hence, the analysis treats arrays as pointers to their first element.
A two-dimensional array in C with dimensions $n$ and $m$ can be viewed as a one-dimensional array with $n \cdot m$ elements.
The expression $a[i][j]$ is treated as $*(a + (i \cdot m + j) \cdot s)$, where $s$ is the size in bits of the array elements.
Similarly, arrays with more dimensions are treated as pointers to a one-dimensional array.

For a struct $s$, the analysis treats $\&s$ as a pointer to the first field of the struct.
For a field $f$ of the struct $s$ with an offset of $z$ bits,
the expression \textsf{s.f} is treated as $*(\&s + z)$.
In C, the offset of a struct element may not always be a multiple of 8 bits.
Thus, offsets are represented in bits rather than bytes.
The analyzer \goblint\ can determine the bit offsets of fields within a struct in most cases.
\cpo\ leverages \goblint\ to find these offsets, and in the rare instances where \goblint\ cannot determine the exact offset, the expression is treated as unknown.


\begin{example}
    Consider this C struct:
    \begin{minted}{C}
        struct S {
            int32_t first;
            int32_t arr[5];
        };
    \end{minted}
    Let $x$ be a variable of type \textsf{struct S}.
    The expression \textsf{x.arr[4]} is treated as $*(\&x + 160)$.
    Inside the struct, the offset of $arr$ is $32$ and the offset of $arr[4]$ is $32\cdot 4 = 128$.
    In total, the offset is $32 + 128 = 160$.
\end{example}
%struct S x[3][2];
%struct S {
% int first;
% int arr[5];
%   };
%x[2][1].arr[4]
%*(1120+x[Ptr64])[Int32]
