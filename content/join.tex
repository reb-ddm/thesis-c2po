
\section{Join}

Given two conjunctions $\Psi_1$ and $\Psi_2$, the least upper bound is a conjunction implied by both $\Psi_1$ and $\Psi_2$ that is smallest with respect to the partial order.
Intuitively, the least upper bound is $\Psi_1 \lor \Psi_2$.
However, disjunctions are not expressible in the 2-Pointer Logic, so we are looking for the most precise conjunction that is implied by $\Psi_1 \lor \Psi_2$.
The conjunction of all propositions implied by a disjunction is not always representable as a conjunction of finite size~\cite{join,2pointer}. Therefore, it is impossible to compute the exact least upper bound of two conjunctions.


\begin{example}\label{example:infinite-join}
    Adapting the original example by \textcite{join} to \cpo, consider the conjunctions $\Psi_1 \equiv (A = V)$ and $\Psi_2 \equiv (*A = A)\land (*V=V) \land (*(1 + A) = *(1 + V))$, represented by the automata $M_1$ and $M_2$ shown in \cref{qfa-infinite-join}.
    \begin{figure}
\centering
        \begin{minipage}{0.3\textwidth}
            \centering
            \begin{tikzpicture}[shorten >=1pt, node distance=3cm, on grid, auto, state/.style={rectangle, rounded corners, draw, inner sep=5pt, align=center}]

                \node[state] (Q1) {$\begin{array}{c}
                            A=V
                        \end{array}$};
                \node [left=1.5cm of Q1] {$M_1$:};
            \end{tikzpicture}
        \end{minipage}
        \begin{minipage}{0.65\textwidth}
            \centering
            \begin{tikzpicture}[shorten >=1pt, node distance=3cm, on grid, auto, state/.style={rectangle, rounded corners, draw, inner sep=5pt, align=center}]

                \node[state] (Q1) {$\begin{array}{c}
                            A=*A=..
                        \end{array}$};

                \node[state, right=3cm of Q1] (Q2) {$\begin{array}{c}
                            V=*V=..
                        \end{array}$};

                \node[state, below right=2cm and 1.5cm of Q1] (Q3) {$\begin{array}{c}
                            *(1+A)= *(1 + V),...
                        \end{array}$};

                \path[->]
                (Q1) edge[loop below] node[left] {$0,0$} (Q1)
                (Q2) edge[loop below] node[right] {$0,0$} (Q2)
                (Q1) edge[] node[right] {$1,0$} (Q3)
                (Q2) edge[] node[above left] {$1,0$} (Q3);

                \node [below left=1cm and 1.5cm of Q1] {$M_2$:};

            \end{tikzpicture}
        \end{minipage}
    \caption{QFAs for $\Psi_1 \equiv (A = V)$ and $\Psi_2 \equiv (*A = A)\land (*V=V) \land (*(1 + A) = *(1 + V))$}\label{qfa-infinite-join}
\end{figure}
    Then for each $n \geq 0$, the equality $*(1+*^n A) = *(1 + *^n V)$ is implied both by $\Psi_1$ and $\Psi_2$, where $*^m$ denotes $m$-fold dereference.
    It can be shown that the set of quantitative equalities implied both by $\Psi_1$ and $\Psi_2$ cannot be represented by a finite conjunction~\cite{join, 2pointer}.
\end{example}

This example shows that it is not possible to compute the exact least upper bound of two domain elements because of the limitations in the expressiveness of the congruence closure representation.
However, it is possible to define a sound over-approximation of the join operation.

Given that equalities only depend on the automaton and cannot follow from the disequalities, we first consider how to construct an automaton that represents the conjunction of two automata.
Later, we will consider the (block-)disequalities that follow from $\Psi_1$ and $\Psi_2$.

As we only compute an approximated join operation, multiple possibilities exist to define the join.
A compromise must be made between the precision of the join and the computational complexity.
Two different approaches will be presented in the following sections.
The first one computes the join based on the automata representation of the equalities, while the second only considers the partition $P$ of the terms $\T$.
This first approach is less efficient but more precise, as all terms of the set $\L(M)$ are considered, while in the second approach, only the terms of the set $\T$ are considered.
In particular, circular dependencies between terms are lost in the second approach.
However, these circular dependencies do not often occur in practice.
Therefore, we expect that the difference in precision is not significant.

\subsection{Join Using the Automaton}

We propose an implementation of the join algorithm of two quantitative finite automata.
It is based on the work of ~\textcite{join}, where they discuss join algorithms for congruence closure data structures.

Given two partitions $P_1 = (\T_1, \tau_1, \omega_1)$ and $P_2 = (\T_2, \tau_2, \omega_2)$ and the corresponding automata $M_1$ and $M_2$, we want to compute the partition $P_3$ and the automaton $M_3$ that represent the conjunction of the two automata.
The first step is to extend the set of terms of $P_1$ and $P_2$ to $\T = \T_1 \cup \T_2$.
Henceforth, we assume that both partitions are defined over the same set of terms $\T$,
i.e., $P_1 = (\T, \tau_1, \omega_1)$ and $P_2 = (\T, \tau_2, \omega_2)$.
Let $M_1$ and $M_2$ be the corresponding automata.

The join is defined as the product automaton of $M_1$ and $M_2$.

\begin{definition}
    Let $M_1 = (S_1, \otau_1, \eta_1, \delta_1)$ and $M_2 = (S_2, \otau_2, \eta_2, \delta_2)$ be two quantitative finite automata.
    The \emph{quantitative product automaton} $M_3 = (S_3, \otau_3, \eta_3, \delta_3)$ is defined as:
    \begin{itemize}
        \item If $\eta_1(a) = (z_1, s_1)$ and $\eta_2(a) = (z_2, s_2)$, then
              $\eta_3(a) = (z_1, \angl{s_1, s_2, z_2-z_1})$.
        \item If $\delta_1(z_1, s_1) = (z'_1, s'_1)$ and $\delta_2(z_2, s_2) = (z'_2, s'_2)$, then $\delta_3(z_1, \angl{s_1, s_2, z_2 - z_1}) =  (z'_1, \angl{s'_1, s'_2, z'_2 -z'_1})$.
        \item $S_3 \subseteq S_1 \times S_2 \times \Z$. $S_3$ contains all states that are reachable in $M_3$.
        \item For each $s \in S_3$, $\otau_3(s)$ is any term in $\L_{M_3}(s)$.
    \end{itemize}
\end{definition}

The product automaton is finite, as there are only a finite amount of reachable states in $S_1 \times S_2 \times \Z$.
The remaining states are unreachable, i.e., they neither occur in a return value of $\eta_3$ nor $\delta_3$.
This follows from the fact that $\eta_i$ and $\delta_i$ for $i = 1,2$ are defined only for a finite amount of values, as discussed in \cref{subsection:qfa}.

The product can be computed in polynomial time, as we need to add an initial state for all pairs of initial states of $M_1$ and $M_2$ and then add a transition for each pair of transitions of $M_1$ and $M_2$. Therefore, the runtime is equal to the size of $M_1$ times that of $M_2$.

\begin{lemma}\label{lemma:join-automata}
    Let $M_3$ be the product automaton of two quantitative automata $M_1$ and $M_2$.
    Then, for each term $t \in \L(M_1) \cap \L(M_2)$ it holds that $M_1[t] = (z_1, s_1)$ and $M_2[t] = (z_2, s_2)$ iff $M_3[t] = (z_1, \angl{s_1, s_2, z_2 - z_1})$.
\end{lemma}
\begin{proof}
    We prove it by induction over the structure of the term $t$.
\end{proof}

As shown in \cref{example:infinite-join}, it is not possible to define a join algorithm that is complete.
In fact, we can only show for our definition of the join algorithm that it is at least complete over the terms in $\L(M_1) \cap \L(M_2)$.
The following proposition shows that our definition of join is complete over the set of terms $\L(M_1) \cap \L(M_2)$ and that it is sound.

\begin{proposition}\label{prop:join-automata}
    Let $M_3$ be the product automaton of two quantitative automata $M_1$ and $M_2$.
    Then, for all terms $t_1, t_2 \in \L(M_1) \cap \L(M_2)$ it holds that $M_3[t_1] = z + M_3[t_2]$ iff $M_1[t_1] = z + M_1[t_2]$ and $M_2[t_1] = z + M_2[t_2]$.
\end{proposition}
\begin{proof}
    This proposition follows from Lemma~\ref{lemma:join-automata}.
\end{proof}

As we mentioned before, we added all terms of $\T = \T_1 \cup \T_2$ to both automata before computing the product automaton, and after having added these terms, it holds that $\T \subseteq \L(M_1) \cap \L(M_2)$.
Therefore, the join of two automata is complete at least over the terms in $\T$.

Using the automaton $M_3$, we can compute the corresponding partition $P_3$.
Let $\T' = \{t \mid \exists s \in S_3 . \otau_3(s) = t\}$ be the set of terms chosen as representatives for the states of $M_3$.
The partition $P_3$ is defined as $P_3 = (\T' \cup \T, \tau_3, \omega_3)$,
where for each $t \in \T' \cup \T$, $\tau_3\,t = \otau_3\,(s)$ and $\omega_3\,t = z$ if $M_3[t] = (z,s)$.

\begin{example}\label{join-example}
    We consider again the two automata of \cref{example:infinite-join}.
    First, we add the terms of the set $\T = \{A, V, *A,*V,*(1+A)$, $*(1 + V)\}$ to the automaton $M_1$, resulting in the automata $M_1'$.
    Let $M_3$ be the product automata of $M_1'$ and $M_2$.
    The automata $M_1'$ and $M_3$ are shown in \cref{m1-m3-qfa}
\begin{figure}
    \centering
        \begin{minipage}{0.45\textwidth}
            \centering
            \begin{tikzpicture}[shorten >=1pt, node distance=3cm, on grid, auto, state/.style={rectangle, rounded corners, draw, inner sep=5pt, align=center}]

                \node[state] (Q1) {$\begin{array}{c}
                            A= V
                        \end{array}$};
                \node[state, below=2cm of Q1] (Q2) {$\begin{array}{c}
                            *(1+A)= *(1+V)
                        \end{array}$};
                \node[state, below left=1cm and 2cm of Q1] (Q3) {$\begin{array}{c}
                            *A
                        \end{array}$};
                \node[state, below right=1cm and 2cm of Q1] (Q4) {$\begin{array}{c}
                            *V
                        \end{array}$};
                \path[->]
                (Q1) edge node {$1,0$} (Q2)
                (Q1) edge node {$0,0$} (Q3)
                (Q1) edge node {$0,0$} (Q4);


                \node [below left=-0.5cm and 1.7cm of Q1] {$M_1'$:};
            \end{tikzpicture}
        \end{minipage}
        \begin{minipage}{0.45\textwidth}
            \centering
            \begin{tikzpicture}[shorten >=1pt, node distance=3cm, on grid, auto, state/.style={rectangle, rounded corners, draw, inner sep=5pt, align=center}]

                \node[state] (Q1) {$\begin{array}{c}
                            A
                        \end{array}$};

                \node[state, right=2cm of Q1] (Q2) {$\begin{array}{c}
                            V
                        \end{array}$};

                \node[state, below right=2cm and 1cm of Q1] (Q3) {$\begin{array}{c}
                            *(1+A)= *(1 + V)
                        \end{array}$};
                \node[state, below left=1cm and 2cm of Q1] (Q4) {$\begin{array}{c}
                            *A
                        \end{array}$};
                \node[state, below right=1.1cm and 2cm of Q2] (Q5) {$\begin{array}{c}
                            *V
                        \end{array}$};
                \path[->]
                (Q1) edge[] node[left] {$1,0$} (Q3)
                (Q2) edge[] node[right] {$1,0$} (Q3)
                (Q1) edge[] node[above left=0cm and -0.2cm] {$1,0$} (Q4)
                (Q2) edge[] node[above] {$1,0$} (Q5);

                \node [below left=-0.5cm and 1.7cm of Q1] {$M_3$:};

            \end{tikzpicture}
        \end{minipage}

    \caption{Visualization of the automata $M_1'$ and $M_3$.}\label{m1-m3-qfa}
\end{figure}
    From $M_3$ we can infer the equality $*(1 + A) = *(1 + V)$, while the equalities $*(1 + *^n A) = *(1 + *^n V)$ for $n \geq 1$ are implied by $M_1'$ and $M_2$, but not by $M_3$.
\end{example}

\subsection{Join Using the Partition}\label{subsection:join-partition}

We propose a second approach to compute the join of two quantitative automata.
This approach is less precise than the previous one, as it only considers the partition of the terms $\T$ and not the automaton.

As before, we first extend the two partitions to the same set of terms $\T$ that contains all terms of both partitions.

Let $P_1 = (\T, \tau_1, \omega_1)$ and $P_2 = (\T, \tau_2, \omega_2)$ be two partitions.
We define the partition $P_3 = (\T, \tau_3, \omega_3)$ that represents the join of $P_1$ and $P_2$.
The terms of $\T$ are partitioned in sets $Q_i$ such that for each two terms $t_1$ and $t_2$ in the same set, it holds that $\tau_1\,t_1 = \tau_1\,t_2$, $\tau_2\,t_1 = \tau_2\,t_2$ and $\omega_2\,t_1 -\omega_1\,t_1 = \omega_2\,t_2 - \omega_1\,t_2$.
Then for each equivalence class $Q_i$ we choose a representative term $t_i\in Q_i$ and
for each $t \in Q_i$ we define $\tau_3\,t = t_i$ and $\omega_3\,t = \omega_1\,t - \omega_1\,t_i$.

The corresponding automaton $M_3$ can be computed starting from the partition $P_3$,
as described in \cref{subsection:qfa}.


\begin{example}\label{ex:join-cycle}

\begin{figure}\begin{subfigure}{0.5\textwidth}
    \centering
      \begin{tikzpicture}[shorten >=1pt, node distance=3cm, on grid, auto, state/.style={rectangle, rounded corners, draw, inner sep=5pt, align=center}]

            \node[state] (Q1) {$\&x= *x$};
            \node[state, right=of Q1] (Q2) {$x= {*}{*}x$};

            \path[->]
              (Q1) edge[bend left, above] node {$0,0$} (Q2)
              (Q2) edge[bend left, below] node {$0,0$} (Q1);

          \end{tikzpicture}


      \caption{The QFA of the conjunction $\&x = *x$.}\label{fig:join-diff-example}
    \end{subfigure}
    \begin{subfigure}{0.5\textwidth}
        \centering
          \begin{tikzpicture}[shorten >=1pt, node distance=3cm, on grid, auto, state/.style={rectangle, rounded corners, draw, inner sep=5pt, align=center}]

                \node[state] (Q1) {$\&x= {*}{*}x$};
                \node[state, right=of Q1] (Q2) {$x$};
                \node[state, below right=of Q1] (Q3) {$*x$};


                \path[->]
                  (Q1) edge[bend left] node {$0,0$} (Q2)
                  (Q2) edge[bend left] node {$0,0$} (Q3)
                  (Q3) edge[bend left] node {$0,0$} (Q1);

              \end{tikzpicture}


          \caption{The QFA of the conjunction $\&x = {*}{*}x$.}\label{fig:join-diff-example2}
        \end{subfigure}
        \caption[An example of quantitative finite automata.]{QFAs for the conjunctions $\&x = *x$ and $\&x = {*}{*}x$.}
    \end{figure}

    The main difference between the two join algorithms is the behavior when the automata contain cycles.
    For example, let $\Psi_1 \equiv \&x = *x$ and $\Psi_2 \equiv \&x = {*}{*}x$.
    Assume that $\T = \{\&x, x, *x, {*}{*}x\}$.
    The automaton $M[\Psi_1]$ is a cycle of length 2 (\cref{fig:join-diff-example}) and the automaton $M[\Psi_2]$ is a cycle of length 3 (\cref{fig:join-diff-example2}).
    If we use the join of the automata, the resulting automaton is a cycle of length 6.
    However, the join of the partitions only considers the terms in $\T$, which are all in distinct equivalence classes in the join.
    Therefore, all information about any equalities are lost with the partitions-join.

\end{example}

\subsection{Join of Two Kernels}

Given two kernels $k_1 = (P_1, M_1,B_1, D_1)$ and $k_2 = (P_2, M_2,B_2, D_2)$, we want to compute the kernel $k_3 = (P_3, M_3,B_3, D_3)$ that represents the join of the two kernels.
We already discussed two different methods of computing the partition $P_3$ and the automaton $M_3$ that represents the join of the two automata.
Now we consider the (block-)disequalities that follow from both kernels,
assuming that $P_3$ and $M_3$ are the join of $P_1$ and $P_2$ and $M_1$ and $M_2$, respectively,
using either of the two methods described above.

\subsubsection{Block Disequalities}

In order to infer the block disequalities that follow from the conjunction of two kernels $k_1$ and $k_2$,
we rewrite the block disequalities of $k_1$ and $k_2$ to use the new representatives of the new equivalence classes of $P_3 = (\T, \tau_3, \omega_3)$.
Then, we take the intersection of these two sets of block disequalities.

Thus, $B_3$ is defined as
\[
    \begin{array}{ll}
        B_3 & = \{\{\tau_3\,t_1, \tau_3\,t_2\}  \mid                                                \\
            & (\{\tau_1\,t_1, \tau_1\,t_2\} \in B_1 \land \{\tau_2\,t_1, \tau_2\,t_2\} \in B_2) \}.
    \end{array}
\]

$B_3$ is computed by rewriting each equality $\{t, t'\} \in B_1$ to use the new representatives of the equivalence classes in $P_3$.
We consider the disequality $\{t_1, t_2\}$
for each pair of terms $t_1$ and $t_2$ that are representatives in $P_3$ and for
which $t_1$ is in the same equivalence class of $t$ and $t_2$ is in the same equivalence class of $t'$ in both partition $P_1$.
There could be multiple such sets $\{t_1, t_2\}$,
as the equivalence classes of $P_1$ might be divided into multiple equivalence classes in $P_3$.
We add each of these disequalities to $B_3$ if $\{\tau_2\,t_1, \tau_2\,t_2\} \in B_2$.

\subsubsection{Disequalities}

In \cref{section:disequalities}, we defined the two types of disequalities that follow from a conjunction:
the set of disequalities that follow from the explicitly stored disequalities,
and the set of implicit disequalities that follow from the equalities or the block disequalities.
As before for the equalities, it is also impossible to define a join operation that computes all disequalities that follow from two conjunctions, as is shown in the following example.

\begin{example}
    Let $\Psi_1 \equiv A = V + 1$ and $\Psi_2 \equiv A = V + 2$. There are infinitely many equalities implied by $\Psi_1$ and $\Psi_2$, i.e., $A \neq V + z$ for all $z \in \Z \setminus \{1,2\}$.
    There is no way to implicitly represent these equalities via equalities or block disequalities, and we can only store a finite amount of explicit disequalities.
    Therefore, it is not always possible to compute the exact set of disequalities implied by two conjunctions.
\end{example}

We have seen that the intersection between the set of implicit disequalities that follow from two conjunctions $\Psi_1$ and $\Psi_2$ could be infinite.
Therefore, we define the join to keep only those disequalities that follow from an explicit disequality in $\Psi_1$ and from an explicit or implicit disequality in $\Psi_2$ or vice versa.

Let $P_3 = (\T, \tau_3, \omega_3)$.
We define
\[
    \begin{array}{ll}
        D_3 & = \{(\tau_3\,t_1,  z + \omega_3\,t_2 - \omega_3\,t_1, \tau_3\,t_2) \mid                                                                            \\
            & ((\tau_1\,t_1, \tau_1\,t_2, z + \omega_1\,t_2 - \omega_1\,t_1) \in D_1 \lor (\tau_2\,t_1, \tau_2\,t_2, z + \omega_2\,t_2 - \omega_2\,t_1) \in D_2) \\
            & \land t_1 \nequiv_{k_1} t_2 + z \land t_1 \nequiv_{k_2} t_2 + z\},
    \end{array}
\]

where $\nequiv_{k}$ symbolizes the disequalities that are (implicitly or explicitly) implied by $k$, i.e.,
it is equal to $\nequiv_{\F[k],\T}$.
They can be determined by following the rules described in \cref{section:disequalities}.

$D_3$ can be computed in polynomial time in the number of explicitly stored disequalities by first constructing the set containing all equalities of $D_1$ by computing their closure with respect to the \emph{closure under quantitative equalities} rule \labelcref{item:closure-under-quantitative-equalities}.
This is necessary because each equivalence class of $P_1$ and $P_2$ might now be divided into multiple equivalence classes in $P_3$.
Then, for each resulting disequality $t_1 \neq z + t_2$,
we need to rewrite it to use the new representatives of the equivalence classes in $P_3$,
so we consider the disequality $\tau_3\,t_1 \neq z' + \tau_3\,t_2$,
where $z' = z + \omega_3\,t_2 - \omega_3\,t_1$.
If this equality is implied (implicitly or explicitly) by $k_2$, then it is added to $D_3$.

The same is done for the disequalities $D_2$.

The only disequalities that are lost by this definition of join are those that are implied only by the implicit disequalities of $\Psi_1$ and the implicit disequalities of $\Psi_2$.

Thus, we can define the join of two kernels $k_1 = (P_1, M_1,B_1, D_1)$ and $k_2 = (P_2, M_2,B_2, D_2)$ as $k_1 \join k_2 \equiv (P_3, M_3,B_3, D_3)$.
