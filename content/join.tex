
\section{Join}

Given two conjunctions $\Psi_1$ and $\Psi_2$, the least upper bound is a conjunction that is implies by both $\Psi_1$ and $\Psi_2$,
and which is smallest with respect to the partial order.
An exact least upper bound is $\Psi_1 \lor \Psi_2$.
However, disjunctions are not expressible in the 2-pointer logic.
The conjunction of all propositions that are implied by a disjunction is not always representable as a conjunction of finite size\cite{join,2pointer},
therefore it is not possible to compute the exact least upper bound of two conjunctions.

\begin{example}\label{example:infinite-join}
    Adapting the original example by \textcite{join} to \cpo, consider the conjunctions $\Psi_1 \equiv (A = B)$ and $\Psi_2 \equiv (*A = A)\land (*B=B) \land (*(1 + A) = *(1 + B))$, represented by the automata $M_1$ and $M_2$ shown below:
    \begin{center}
    \begin{minipage}{0.3\textwidth}
        \centering
        \begin{tikzpicture}[shorten >=1pt, node distance=3cm, on grid, auto, state/.style={rectangle, rounded corners, draw, inner sep=3pt, align=center}]

          \node[state] (Q1) {$\begin{array}{c}
              A,B
          \end{array}$};
          \node [left=1.5cm of Q1] {$M_1$:};
      \end{tikzpicture}
    \end{minipage}
    \begin{minipage}{0.65\textwidth}
        \centering
        \begin{tikzpicture}[shorten >=1pt, node distance=3cm, on grid, auto, state/.style={rectangle, rounded corners, draw, inner sep=3pt, align=center}]

          \node[state] (Q1) {$\begin{array}{c}
              A,*A,..
          \end{array}$};

          \node[state, right=3cm of Q1] (Q2) {$\begin{array}{c}
             B,*B,..
          \end{array}$};

          \node[state, below right=2cm and 1.5cm of Q1] (Q3) {$\begin{array}{c}
              *(1+A), *(1 + B),...
          \end{array}$};

          \path[->]
          (Q1) edge[loop below] node[left] {$0,0$} (Q1)
          (Q2) edge[loop below] node[right] {$0,0$} (Q2)
          (Q1) edge[] node[right] {$1,0$} (Q3)
          (Q2) edge[] node[above left] {$1,0$} (Q3);

          \node [below left=1cm and 1.5cm of Q1] {$M_2$:};

      \end{tikzpicture}
      \end{minipage}
    \end{center}

    Then for each $n \geq 0$, the equality $*(1+*^n A) = *(1 + *^n B)$ is implied both by $\Psi_1$ and $\Psi_2$, where $*^m$ denotes $m$-fold dereference.
    It can be shown that the set of quantitative equalitities implied both by $\Psi_1$ and $\Psi_2$ cannot be represented by a finite conjunction~\cite{join, 2pointer}.
\end{example}

This example shows that it is not possible to compute the exact least upper bound of two domain elements,
because of the limitations in the expressiveness of the congruence closure representation.
However, it is possible to define a sound overapproximation of the join operation.

Given that equalities only depend on the automaton and cannot follow from the disequalities, we first consider how to construct an automaton that represents the conjunction of two automata.
Later, we will consider the (block-)disequalities that follow from $\Psi_1$ and $\Psi_2$.

As we only compute an approximated join operation, there are multiple possibilities to define the join.
A compromise must be made between the precision of the join and the computational complexity.
Two different approaches will be presented in the following sections.
The first one computes the join based on the automata representation of the equalities,
while the second one only considers the partition of the terms $\T$.
This first approach is less efficient, but more precise,
as all terms of the set $\L(M_P)$ are considered, while in the second approach only the terms of the set $\T$ are considered.
In particular, circular dependencies between terms are lost in the second approach.
However, these circular dependencies don't often occur in practice, therefore it is
considered as a possible option for the join operation.

\subsection{Join Using the Automaton}

We propose an implementation of the join algorithm of two quantitative finite automata.
It is based on the work of Gulwani et al.~\cite{join}, where they discuss join algorithms for congruence closure data structures.

Given two partitions $P_1 = (\T_1, \tau_1, \omega_1)$ and $P_2 = (\T_2, \tau_2, \omega_2)$ and the corresponding automata $M_{P_1}$ and $M_{P_2}$, we want to compute the partition $P_3$ and the automaton $M_{P_3}$ that represent the conjunction of the two automata.
The first step is to extend the set of terms of $P_1$ and $P_2$ to $\T = \T_1 \cup \T_2$.
From now on we assume that both partitions are defined over the same set of terms $\T$,
i.e., $P_1 = (\T, \tau_1, \omega_1)$ and $P_2 = (\T, \tau_2, \omega_2)$.
Let $M_{P_1}$ and $M_{P_2}$ be the corresponding automata.

The join is defined as the product automaton of $M_1$ and $M_2$.

\begin{definition}
    Let $M_1 = (S_1, \otau_1, \eta_1, \delta_1)$ and $M_2 = (S_2, \otau_2, \eta_2, \delta_2)$ be two quantitative finite automata. The \emph{quantitative product automaton} $M_3 = (S_3, \otau_3, \eta_3, \delta_3)$ is defined as:
    \begin{itemize}
        \item If $\eta_1(a) = (z_1, s_1)$ and $\eta_2(a) = (z_2, s_2)$, then
              $\eta_3(a) = (z_1, \angl{s_1, s_2, z_2-z_1})$.
        \item If $\delta_1(z_1, s_1) = (z'_1, s'_1)$ and $\delta_2(z_2, s_2) = (z'_2, s'_2)$, then $\delta_3(z_1, \angl{s_1, s_2, z_2 - z_1}) =  (z'_1, \angl{s'_1, s'_2, z'_2 -z'_1})$.
        \item $S_3 \subseteq S_1 \times S_2 \times \Z$. $S_3$ contains all states that are reachable in $M_3$.
        \item For each $s \in S_3$, $\otau_3(s)$ is any term in $\L_{M_3}(s)$.
    \end{itemize}
\end{definition}

The product automaton is a finite automaton, as there is only a finite amount of states in $S_1 \times S_2 \times \Z$ which are reachable.
All the remaining states are unreachable, i.e.\ they neither occur in a return value of $\eta_3$, nor of $\delta_3$.
This follows from the fact that $\eta_i$ and $\delta_i$ for $i = 1,2$ are defined only for a finite amount of values, as discussed in \cref{subsection:qfa}.

The product can be computed in polynomial time, as we need to add an initial state for all pairs of initial states of $M_1$ and $M_2$ and then add a transition for each pair of transitions of $M_1$ and $M_2$. Therefore the runtime is equal to the size of $M_1$ times the size of $M_2$.

\begin{lemma}\label{lemma:join-automata}
    Let $M_3$ be the product automaton of two quantitative automata $M_1$ and $M_2$.
    Then, for each term $t \in \L(M_1) \cap \L(M_2)$ it holds that $M_1[t] = (z_1, s_1)$ and $M_2[t] = (z_2, s_2)$ iff $M_3[t] = (z_1, \angl{s_1, s_2, z_2 - z_1})$.
\end{lemma}
\begin{proof}
    We prove it by induction over the structure of the term $t$.
\end{proof}

As shown in Example~\ref{example:infinite-join}, it is not possible to define a join algorithm that is complete.
In fact, we can only show for our definition of the join algorithm that it is at least complete over the terms in $\L(M_1) \cap \L(M_2)$, which will be shown in the following proposition.
Additionally, the proposition shows that our definition of join is sound.

\begin{proposition}\label{prop:join-automata}
    Let $M_3$ be the product automaton of two quantitative automata $M_1$ and $M_2$.
    Then, for all terms $t_1, t_2 \in \L(M_1) \cap \L(M_2)$ it holds that $M_3[t_1] = z + M_3[t_2]$ iff $M_1[t_1] = z + M_1[t_2]$ and $M_2[t_1] = z + M_2[t_2]$.
\end{proposition}
\begin{proof}
    This proposition follows from Lemma~\ref{lemma:join-automata}.
\end{proof}

As we mentioned before, we added all terms of $\T_1 \cup \T_2$ to both automata before computing the product automaton, and after having added these terms, it holds that $\T \subseteq \L(M_1) \cap \L(M_2)$.
Therefore, the join of two automata is complete at least over the terms in $\T$.

From the automata $M_3$ we can compute the corresponding partition $P_3$.
Let $\T' = \{t \mid \exists s \in S_3 . \otau_3(S_3) = t\}$ be the set of terms that were chosen
as representatives for the states of $M_3$.
Then the partition $P_3$ is defined as $P_3 = (\T' \cup \T, \tau_3, \omega_3)$,
where for each $t \in \T' \cup \T$, $\tau_3\,t = \otau_3\,(s)$ and $\omega_3\,t = z$ if $M_3[t] = (z,s)$.

\begin{example}
    We consider two automatas $M_1$ and $M_2$, where $M_1$ represents the conjunction $A = B + 2 \land A = C + 3$ and $M_2$ represents the conjunction $A = B + 1 \land A = C + 2$.
    The automaton $M_1$ has one single state $s_0$ that represents the equivalence class $\{A, B+2, C+3\}$
    and $M_2$ has one single state $s_1$ that represents the equivalence class $s_1 = \{A, B+1, C+2\}$.
    The product automaton $M_3$ is made of no transitions and two states $\angl{s_0, s_1, 0}$ and $\angl{s_0, s_1, 1}$, where the first one represents the set $\{A\}$ and the second one the set $\{B, C+1\}$.
    Therefore the product automaton represents the proposition $B = C + 1$.
    The join operation keeps the relation between $B$ and $C$, as they have the same distance in $M_1$ and $M_2$.
    On the other hand, $A$ has not the same distance to $B$ or $C$ in the input automata, therefore we forget all information about $A$.
\end{example}

\begin{example}\label{example:join-subset}
    We consider two automata $M_1$ and $M_2$, where $M_1$ represents the conjunction $x = \&x$ and $M_2$ represents the conjunction $*x = \&x$.
    $M_1$ has one state $s_0$ and one transition from $s_0$ to itself. We have that $\L_{M_1}(s_0) =\{*^n\&x\mid n \geq 0\}$.
    The automaton $M_2$ has two states $s_1$ and $s_2$ and two transitions from $s_1$ to $s_2$ and from $s_2$ to $s_1$. We have that $\L_{M_2}(s_1) = \{*^{2n}\&x\mid n \geq 0\}$ and $\L_{M_2}(s_2) = \{*^{2n}x\mid n \geq 0\}$.
    We have that $\eta_1(\&x) = (0, s_0)$ and $\eta_2 (\&x) = (0, s_1)$.
    The product automaton $M_3$ has two states $\angl{s_0, s_1, 0}$ and $\angl{s_0, s_2, 0}$ and two transitions from the first state to the second and from the second state to the first state and $\eta_3(\&x) = (0, \angl{s_0, s_1, 0})$.
    Therefore the product automaton is isomorphic to the second automaton and it represents the same equality $*x = \&x$.
\end{example}

\subsubsection{Disequalities}
Given two kernels $k_1 = (P_1, M_{P_1},B_1, D_1)$ and $k_2 = (P_2, M_{P_2},B_2, D_2)$, we want to compute the kernel $k_3 = (P_3, M_{P_3},B_3, D_3)$ that represents the join of the two kernels.
We already discussed two different methods of how to compute the partition $P_3$ and the automaton $M_{P_3}$ that represents the join of the two automata.
Now we consider the (block-)disequalities that follow from both kernels.

In \cref{section:disequalities} we defined the two types of disequalities that follow from a conjunction:
the set of disequalities that follow from the explicitly stored disequalities,
and the set of implicit disequalities that follow from the equalities or the block disequalities.
As before with the equalities, it is also not possible to define a join operation that computes all disequalities that follow from two conjunctions, as is shown in the following example.

\begin{example}
    Let $\Psi_1 \equiv A = B + 1$ and $\Psi_2 \equiv A = B + 2$. There are infinitely many equalities that are implied by $\Psi_1$ and $\Psi_2$ i.e. $A \neq B + z$ for all $z \in \Z \backslash \{1,2\}$.
    There is no way to represent these equalities implicitly via equalities or block disequalities, and we can only store a finite amount of explicit disequalities. Therefore, it is not always possible to compute the exact set of disequalities for the join.
\end{example}

We have seen that the intersection between the set of implicit disequalities that follow from two conjunctions $\Psi_1$ and $\Psi_2$ could be infinite.
Therefore, we define the join to keep only those disequalities that follow from an explicit disequality in $\Psi_1$ and from an explicit or implicit disequality in $\Psi_2$ or vice versa.

\begin{definition}
    Let $\Psi_1$ and $\Psi_2$ be two conjunctions of 2-pointer propositions. We define $\Psi_1 \join_\neq \Psi_2 = \{t_1 \neq t_2 + z | t_1 \neq t_2 + z \text{ is an explicit disequality of } \Psi_1 \text{ or of } \Psi_2 \text{ and } t_1 \nequiv_{\Psi_1} t_2 + z \text{ and } t_1 \nequiv_{\Psi_2} t_2 + z\}$.
\end{definition}

This join can be computed in polynomial time in the number of explicitly stored disequalities by first constructing the set containing all equalities of $closure_\neq(\Psi_1)$.
For each equality $t \neq t' + z$ in this set, we need to rewrite it to use the new representatives of the equivalence classes in $\Pi_3$.
However, each equivalence class of $\Pi_1$ and $\Pi_2$ might now be divided into multiple equivalence classes in $\Pi_3$.
Therefore for each term in each explicit disequality of $\Psi_1$ and $\Psi_2$, we need to consider all new representatives for each derived sub-equivalence class.

Let $M_3$ be the join automaton of the automata $M_1 = M_P[\Psi_1]$ and $M_2 = M_P[\Psi_2]$.
For a state $s_1$ in $M_1$, there could be multiple states of the form $s_3 = \angl{s_1, s_2', z'} \in S_3$ for some $s_2' \in S_2$ and some $z' \in \Z$.
For each of these states, it holds that $\L_{M_3}(s_3) \subseteq \L_{M_1}(s_1)$.
Therefore, for each disequality of the form $t_1 \neq z + t_2 \in closure_\neq(\Psi_1)$, such that $M_1[t_1] = (z_1, s_1)$ and $M_1[t_2] = (z_2, s_2)$ for some $z_1,z_2\in \Z$, we need to consider
each of the states $s_1', s_2' \in S_3$ such that the first component of $s_1'$ is $s_1$ and the first component of $s_2'$ is $s_2$. Let $Q_1', Q_2' \in \Pi_3$ be the corresponding equivalence classes of $s1'$ and $s2'$, respectively.
Then the equality $\tau_3 Q_1' \neq (z + \omega_1 (\tau_3 Q_2') - \omega_1 (\tau_3 Q_1')) +\tau_3 Q_2'$ is one of the equalities that follow from the explicitly stored disequalities of $\Psi_1$.
For each of these disequalities that we found, we need to check if it is implied by $\Psi_1$ and $\Psi_2$.

We do the same thing for each disequality of the form $t_1 \neq z + t_2 \in closure_\neq(\Psi_2)$:
let $M_2[t_1] = (z_1, s_1)$ and $M_2[t_2] = (z_2, s_2)$ for some $z_1,z_2\in \Z$.
We need to consider the equalities $\tau_3 Q_1' \neq (z + \omega_2 (\tau_3 Q_2') - \omega_2 (\tau_3 Q_1')) +\tau_3 Q_2'$ for each of the states $s_1', s_2' \in S_3$ such that the second component of $s_1'$ is $s_1$ and the second component of $s_2'$ is $s_2$ and $Q_1', Q_2' \in \Pi_3$ being the corresponding equivalence classes.

The only disequalities that are lost by this definition of join are those that are implied only by the implicit disequalities of $\Psi_1$ and the implicit disequalities of $\Psi_2$.

\begin{proposition}
    Let $\Psi_1$ and $\Psi_2$ be conjunctions of 2-pointer propositions.
    Then $\Psi_1 \Longrightarrow \Psi_1 \join_\neq \Psi_2$ and $\Psi_2 \Longrightarrow \Psi_1 \join_\neq \Psi_2$.
\end{proposition}

\begin{proof}
    By definition of $\join_\neq$, we add only equalities that follow both from $\Psi_1$ and from $\Psi_2$.
\end{proof}

\subsubsection{Join of block disequalites}
We have shown how to compute all equalities and disequalities that follow from two conjunctions $\Psi_1$ and $\Psi_2$.
The only missing propositions are the block equalities.
Exactly as with the disequalities, we need to rewrite them such that they use the new representatives of the new equivalence classes.

As before, for each disequality $bl(t_1) \neq bl(t_2) \in \Psi_1$, let $M_1[t_1] = (z_1, s_1)$ and $M_1[t_2] = (z_2, s_2)$ for some $z_1,z_2\in \Z$.
For each pair of states $s_1', s_2'$, where the first component of $s_1'$ is $s_1$ and the first component of $s_2'$ is $s_2$ and their corresponding equivalence classes $Q_1', Q_2' \in \Pi_3$, we add the disequality $bl(\tau Q_1') \neq bl(\tau Q_2')$.
We do the same thing for $\Psi_2$ and then intersect the set of block disequalities that derive from the two conjunctions.

\subsection{Join Without The Automaton}
