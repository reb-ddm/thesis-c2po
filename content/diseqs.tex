\section{Block Disequalities}\label{section:block-disequalities}

The 2-Pointer Logic considers not only equalities but also block disequalities and disequalities between terms.
These are necessary to infer which terms are not
modified when an assignment occurs.
Given a conjunction $\Psi$ of propositions and the corresponding equivalence relation $\equivp$, the block disequalities implied by $\Psi$ can be derived by the following rules:\todo{cite new paper?}

\begin{enumerate}[label={[B\arabic*]}, ref={[B\arabic*]}]
    \setcounter{enumi}{-1} % to start with B0
    \item If $bl(t_1)\neq bl(t_2)$ then $bl(t_1)\nequivp bl(t_2)$
    \item If $bl(t_1)\nequivp bl(t_2)$ then $bl(t_2)\nequivp bl(t_1)$
          (\emph{symmetry});
          \item\label{item:closure-under-quantitative-equalities}
          If $bl(t_1)\nequivp bl(t_2)$, $t'_1 \equivp z_1+t_1$ and $t'_2\equivp z_2+t_2$, then
          also $bl(t'_1)\nequivp bl(t'_2)$ (\emph{closure under quantitative equalities});
          \item\label{item:distinct-vars-block-neq} $bl(\&x) \neq bl(\&y)$ for all distinct $x, y \in \X$.
\end{enumerate}

The block disequalities are stored as a map $B[\Psi]$ that maps representative terms to a set of representative terms that are not in the same block.
It is sufficient to only store the disequalities between representatives, i.e.,
the normalized disequalities $norm_{P[\Psi]}(p)$.
The block disequalities between other terms are derived from the congruence closure of $\Psi$ and the map $B[\Psi]$.
The disequalities deriving from rule~\ref{item:distinct-vars-block-neq} are considered as \emph{implicit} block disequalities and are therefore not stored in the map $B[\Psi]$.

\begin{example}
    Consider the conjunction $\Psi \equiv A = 1 + B \land bl(B) \neq bl(*(1 + B))$.\todo{change A and B to something else please because of the map B}
    The QFA deriving from $\Psi$ is shown in \cref{fig:closure-example}.
    As $\tau\,B = A$ and $\tau\,(*(1+B)) = *A$, the map $B[\Psi]$ would contain the entries $A \mapsto \{*A\}$ and $*A \mapsto \{A\}$.
\end{example}
\todo{define them more explicitly and also with examples in order to understand the meet ecc.}

\section{Disequalities}\label{section:disequalities}

Let $\Psi$ be a conjunction of 2-pointer logic terms.
We define the binary relation $\nequivp$ which represents all disequalities that can be derived from $\Psi$, given the relations $\equivp$ and $bl(\cdot)\nequivp bl(\cdot)$.\todo{cite new paper?}

\begin{enumerate}[label={[D\arabic*]}, ref={[D\arabic*]}]
    \setcounter{enumi}{-1} % to start with D0
    \item\label{item:neq-quantitative-equalities}
    If $t_1 \neq z + t_2$ occurs in $\Psi$, and $t_1 \equivp z'_1 + t'_1$ and $t_2\equivp z'_2+t'_2$,
    then $t'_1\nequivp (z'_2-z'_1+z)+t'_2$ (\emph{closure under quantitative equalities});
    \item\label{item:neq-quantitative-symmetry} If $t_1 \nequivp z + t_2$, then $t_2 \nequivp -z + t_1$ (\emph{quantitative symmetry});
    %
    \item\label{item:inverse-deref}
    If $*(z_1 + t_1) \nequivp *(z_2 + t_2)$, then $t_1 \nequivp (z_2 - z_1) + t_2$
    (\emph{inverse dereferencing});
    \item\label{item:eq-neq} If $t_1 \equivp z + t_2$, then $t_1\nequivp z'+t_2$ for all $z'\neq z$;
    \item\label{item:block-neq} If $bl(t_1) \nequivp bl(t_2) \in \Psi$, then $t_1 \nequivp z + t_2$ for all $z \in \Z$.
\end{enumerate}

As for the block disequalities, we store a map $D[\Psi]$ of disequalities between representative terms.
A disequality $t_1 \nequivp z + t_2$ is represented in $D[\Psi]$ as the mapping $t_1 \mapsto S$ and $(z, t_2) \in S$, as well as the mapping $t_2 \mapsto S$ and $(-z, t_1) \in S$.
The set of disequalities following from rule~\ref{item:eq-neq} and rule~\ref{item:block-neq} can be infinite, but we do not need to explicitly store them, as they implicitly derive from the relations $\equivp$ and $bl(\cdot)\nequivp bl(\cdot)$, which are stored in the partition $P[\Psi]$ and the map $B[\Psi]$.

The list $D[\Psi]$ consists of all disequalities and block disequalities deriving from
\labelcref{item:neq-quantitative-equalities,item:neq-quantitative-symmetry,item:inverse-deref,item:eq-neq,item:block-neq},
but only those that contain representative terms.
This set is finite because we only store equalities between different representatives and not the implicit equalities following from \labelcref{item:eq-neq,item:block-neq}.

The disequalities $D = D[\Psi]$ can be computed in polynomial time in the size of the formula $\Psi$ in the following way:

\begin{itemize}
    \item For each equality $*(z_1 + t_1) \equivp z + *(z_2 + t_2)$ for $z \neq 0$
          that follows from $\Psi$,
          add $norm_{P[\Psi]}(t_1 = (z_2 - z_1) + t_2)$ to $D$.
    \item For each disequality $bl(*(z_1 + t_1)) \nequivp bl(*(z_2 + t_2))$ that follows from $\Psi$, add $norm_{P[\Psi]}(t_1 \neq (z_2 - z_1) + t_2)$ to $D$.
          These two first rules only add a finite amount of disequalities to the set, given that each equality and each block disequality implies at most one disequality of the type $*(z_1 + t_1) \nequivp *(z_2 + t_2)$.
    \item Compute the closure of the disequalities in $D$ using rule~\labelcref{item:inverse-deref}, and add the normalized disequalities to $D$.
    \item For each disequality $p \in \Psi$, add $norm_{P[\Psi]}(p)$ to $D$.
    \item Again, compute the closure of the disequalities in $D$ using rule~\labelcref{item:inverse-deref}.
\end{itemize}

Given a conjunction $\Psi$, a set $\T$ and the corresponding partition $P = P[\Psi]$, let $p \equiv t_1 \neq z + t_2$ and $norm_{P[\Psi]}(p) = t'_1 \neq z' + t'_2$.
Then
it holds that $\Psi$ implies $p$ if and only if $t_1 \nequivp z + t_2$ and if and only if one of the following holds:

\begin{enumerate}
    \item\label{item:diseqs} There is a mapping $t_1' \mapsto S$ in $D$ and $(z', t_2') \in S$;
    \item\label{item:bl-diseqs} or there is a mapping $t_1 \mapsto S$ in $B$ and $t_2 \in S$;
    \item\label{item:eqs} or $t_1 \equivp z' + t_2$ for any $z' \neq z$.
\end{enumerate}
We call the disequalities that follow from \labelcref{item:diseqs}.\ the \emph{explicit} disequalities and those following from \labelcref{item:bl-diseqs}.\ and \labelcref{item:eqs}.\ the \emph{implicit} disequalities.
