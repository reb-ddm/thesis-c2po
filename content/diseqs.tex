\section{Block Disequalities}\label{section:block-disequalities}

The 2-Pointer Logic considers not only equalities, but also block disequalities and disequalities between terms.
These are necessary in order to infer which terms are definitely not
modified when an assignment takes place.
Given a conjunction $\Psi$ of propositions, the block disequalities implied by the $\Psi_{bl}$
and $\equivp$ can be derived by the following rules:

\begin{enumerate}[label={[B\arabic*]}, ref={[B\arabic*]}]
    \setcounter{enumi}{-1} % to start with B0
    \item	If $bl(t_1)\neq bl(t_2)$ then $bl(t_2)\nequivp bl(t_1)$ (\emph{symmetry});
          \item\label{item:closure-under-quantitative-equalities}
          If $bl(t_1)\nequivp bl(t_2)$, $t'_1 \equivp z_1+t_1$ and $t'_2\equivp z_2+t_2$, then
          also $bl(t'_1)\nequivp bl(t'_2)$ (\emph{closure under quantitative equalities});
          \item\label{item:distinct-vars-block-neq} $bl(\&x) \neq bl(\&y)$ for all distinct $x, y \in \X$.
\end{enumerate}

The block disequalities are stored as a list $B$ of disequalities between representative terms
of the congruence closure.
The disequalities deriving from rule~\ref{item:distinct-vars-block-neq} are considered as \emph{implicit} block disequalities.
The disequalities between other kinds of terms derive implicitly from the list $B$ and the congruence closure of $\Psi$.

\section{Disequalities}\label{section:disequalities}

Let $\Psi$ be a conjunction of 2-pointer logic terms.
We define the binary relation $\nequivp$ which represents all disequalities that can be derived from $\Psi$.

\begin{enumerate}[label={[D\arabic*]}, ref={[D\arabic*]}]
    \setcounter{enumi}{-1} % to start with D0
    \item\label{item:neq-quantitative-equalities}
    If $t_1 \neq z + t_2$ occurs in $\Psi$, and $t_1 \equivp z'_1 + t'_1$ and $t_2\equivp z'_2+t'_2$,
    then $t'_1\nequivp (z'_2-z'_1+z)+t'_2$ (\emph{closure under quantitative equalities});
    \item\label{item:neq-quantitative-symmetry} If $t_1 \nequivp z + t_2$, then $t_2 \nequivp -z + t_1$ (\emph{quantitative symmetry});
          %
          \item\label{item:inverse-deref}
          If $*(z_1 + t_1) \nequivp *(z_2 + t_2)$, then $t_1 \nequivp (z_2 - z_1) + t_2$
          (\emph{inverse dereferencing});
          \item\label{item:eq-neq} If $t_1 \equivp z + t_2$, then $t_1\nequivp z'+t_2$ for all $z'\neq z$;
          \item\label{item:block-neq} If $bl(t_1) \neq bl(t_2) \in \Psi$, then $t_1 \nequivp z + t_2$ for all $z \in \Z$.
\end{enumerate}

As for the block disequalities, we store a list $D$ of disequalities between representative terms.
The set of disequalities following from rule~\ref{item:eq-neq} and rule~\ref{item:block-neq} can be infinite, but we don't need to explicitly store them, as they implicitly derive from the relation $\equivp$ or respectively from the block disequalities in the list $B$.

The list $D$ consists of all disequalities and block disequalities deriving from
\labelcref{item:neq-quantitative-equalities,item:neq-quantitative-symmetry,item:inverse-deref,item:eq-neq,item:block-neq},
but only those that contain representative terms of components $Q\in\Pi$.
This set is finite, because we only store equalities between different representatives and not the implicit equalities following from \labelcref{item:eq-neq,item:block-neq}.

Given a disequality $p \equiv t_1 \neq z + t_2$ and a partition $P = (\Pi, \tau, \omega)$,
the normalized disequality $norm_P(p)$ is an equivalent disequality that only contains representatives of $P$,
i.e., $\otau\,t_1 = z - \omega\,t_1 + \omega\,t_2 + \otau\,t_2$.
This set can be computed in polynomial time in the size of the formula $\Psi$ in the following way:

\begin{itemize}
    \item For each equality $*(z_1 + t_1) \equivp z + *(z_2 + t_2)$ for $z \neq 0$
    that follows from $\Psi$,
    add the normalized version of the disequality $t_1 = (z_2 - z_1) + t_2$ to $D$.
    \item For each disequality $bl(*(z_1 + t_1)) \neq bl(*(z_2 + t_2))$ that follows from $\Psi$, add the normalized version of the disequality $t_1 \neq (z_2 - z_1) + t_2$ to $D$.
    these two first rules only add a finite amount of disequalities to the set, given that each equality and each block disequality implies at most one disequality of the type $*(z_1 + t_1) \nequivp *(z_2 + t_2)$.
    \item Compute the closure of the disequalities in $D$ using rule~\labelcref{item:inverse-deref}.
    \item For each disequality $p \in \Psi$, add $norm_P(p)$ to $D$.
    \item Again, compute the closure of the disequalities in $D$ using rule~\labelcref{item:inverse-deref}.
\end{itemize}

Given a conjunction $\Psi$, a set $\T$ and the corresponding partition $P$,
it holds that $t_1 \nequivp z + t_2$ iff one of the following holds:

\begin{enumerate}
    \item The normalized disequality $norm_P(t_1 \neq z + t_2)$ is in $D$;
    \item Either $bl(\otau\,t_1) \neq bl(\otau\,t_2) \in B$;
    \item $t_1 \equivp z' + t_2$ for any $z' \neq z$.
\end{enumerate}
We call the disequalities that follow from 1.\ the \emph{explicit} disequalities and those following from 2.\ and 3.\ the \emph{implicit} disequalities.
