\section{Assignment}

In the following, we will describe the abstract semantics $\sem{s_1\,{:=}\,s_2}^\sharp\,k$ of the assignment $s_1\,{:=}\,s_2$ over the kernel $k$, for the each of the three possibilities of $s_2$.
If $k \equiv \bot$, then $\sem{s_1\,{:=}\,s_2}^\sharp\,k \equiv \bot$.
In the following, we assume that $k \neq \bot$.

\subsection{Indefinite Assignment}

We consider an assignment of the form $s_1\,{:=}\,?$.
After this assignment, the value of $s_1$ changes to an unknown value,
so we must forget all the equalities and disequalities that contain the term $s_1$,
but also all other terms that may be modified by overwriting $s_1$.
Only the terms where we know that each subterm is definitely not equal to the address of $s_1$ are not modified by overwriting $s_1$.
In the following, we will describe how to find the set of terms
that are not modified by the assignment.

\ignore{When overwriting $s_1$, all the terms whose value is written in the memory at the same address as $s_1$ are also modified.
In the case that $s_1$ is an auxiliary $B \in \A$, then we can't reach the address of $B$ by dereferencing any other term.
Therefore, the only terms that are modified by overwriting $B$ are the terms that contain $B$ as a subterm.
On the other hand, if $s_1$ is a pointer term $*(z+t)$, then we need to forget all the terms that contain any subterm that may be equal to $z+t$.
Then we need to restrict the kernel $k$ to forget all the equalities and disequalities that contain the prohibited terms.}

Given a kernel $k = \angl{P, M_P, B, D}$, a partition $P = (\T, \tau, \omega)$ and a term $t$,
we want to compute the set $\restr{\T}{\neg t}$ that contains all terms $t' \in \T$ such that each subterm of $t'$ definitely does not alias with $t$.
This is exactly the set of terms that are not modified when the value of $t$ changes.
We differentiate between two cases: the case where $t$ is an atom and the case where $t$ is a dereferenced term.

If $t$ is an atom, then it is impossible to reach the address of $t$ by dereferencing.
Therefore, $\restr{\T}{\neg t}$ is the set of all terms that do not contain $t$ as a subterm.

If $t \equiv *(z + t')$ is a dereferenced term, then $\restr{\T}{\neg t}$ contains all the terms of $\T$ where for each subterm of the form $*(z' + v)$ it holds that $t' \nequivp (z' - z) + v$.
\todo{add definition of $\nequiv_k$}
\todo[inline]{If it only was that easy. What about overlapping terms? What about MayPointTo?}
As defined in \cref{section:disequalities}, we can derive the diesqualities from the disequalities in $D$, from the equalities
implied by $P$ and from the set of block disequalities $B$.

To simplify the notation, the kernel that is restricted to the terms in $\restr{\T}{\neg t}$ is denoted $\restr{k}{\neg t}$.
It represents all propositions that follow from $M$ and that are still valid after overwriting the value of $t$.
Therefore we can define the assignment as follows:

\[
	\sem{s_1\,{:=}\;?}^\sharp\,k \equiv \restr{k}{\neg s_1}.
\]

\begin{proposition}\label{p:ass-unknown}
	For every kernel $k$ and assignment $s_1\,{:=}\,?$,
	\[
		\sem{s_1\,{:=}\,?}\,(\gamma\,k) \subseteq \gamma\,(\sem{s_1\,{:=}\,?}^\sharp\,k)
	\]
\end{proposition}

\todo{prove}
The proof of this proposition for the case of a one-dimensional memory model can be found in~\cite{2pointer}.
This proof can easily be adapted to the two-dimensional model.

\subsection{Definite Assignment}

We consider an assignment of the form $s_1\,{:=}\,z_1+s$.
As before, we need to forget all the equalities and disequalities about terms that may be modified by overwriting $s_1$.
After this, is not sufficient to simply add the equality $s_1 = z_1+s$ to the set of equalities,
because the term $z_1 + s$ may also be modified by the assignment. For example, when we assign $B\,{:=}\,1+B$, the equality $B = 1 + B$ would incorrectly lead to an unsatisfiable conjunction.
Therefore we introduce a fesh auxiliary $A$ and first assign $A$ to the right-hand side of the assignment, then restrict the automaton, and then assign $s_1$ to $A$.
This way we can remember the original value of $z_1 + s$ trough the auxiliary $A$ and reconstruct the equalities that still hold after the assignment.

We define the abstract effect of the definite assignment as follows:

\[
	\sem{s_1\,{:=}\,z_1+s}^\sharp\,k\;\equiv\;
	\restr{((s_1=A)\land\restr{((A=z_1+s)\land\Psi)}{\neg s_1})}{\neg A}
\]

\begin{proposition}\label{p:ass-definite}
	For every kernel $k$ and any assignment of the form $s_1\,{:=}\,z_1+s$,
	\[
		\sem{s_1\,{:=}\,z_1+s}\,(\gamma\,k) \subseteq \gamma\,(\sem{s_1\,{:=}\,z_1+s}^\sharp\,k)
	\]
\end{proposition}

\todo{As before, todo prove}
As before, the proof of this proposition for the case of a one-dimensional memory model can be found in~\cite{2pointer}.

\subsection{Memory Allocation}

We consider an assignment of the form $s_1\,{:=}\,\malloc$.
The term $s_1$ is assigned to a fresh address, therefore we know that after the assignment,
the block of the term $s_1$ is different from the block of any other term in the set $\T$ of the partition,
but also different than any other term $t \in \oT$ that is not (yet) in the partition.
However, we cannot remember an infinite number of equalities that contain an infinite number of possible terms,
therefore we choose to only remember that this new address is different than the address of any other term in $\T$.
Let $k = \angl{P, M_P, B,D}$, and $\restr{k}{\neg s_1} = \angl{P', M_P, B, B}$.
For each equivalence class in the partition $P'$, we add a disequality that states
that the block of the term $s_1$ is different from the block of the representative of the equivalence class.

\[
	\sem{s_1\,{:=}\,\text{\malloc}}^\sharp\,k \equiv
	\restr{k}{\neg s_1}\;\land
	\bigwedge_{t\in \T,s_1\notin Q'}(bl(s_1) \neq bl(\tau\,t))
\]

\begin{proposition}\label{p:ass-malloc}
	For every kernel $k$ and any assignment of the form $s_1\,{:=}\,\malloc$,
	\[
		\sem{s_1\,{:=}\,\malloc}\,(\gamma\,k) \subseteq \gamma\,(\sem{s_1\,{:=}\,\malloc}^\sharp\,k)
	\]
\end{proposition}

\begin{proof}
	We only consider the case where $s_1$ equals the expression $*(z+t)$.
	The other case where $s_1 = B$ is analogous.
	Let $(\rho',\nu',\mu') \in \sem{s_1\,{:=}\,\malloc}\,(\gamma\,k)$, then we know that there exists $(\rho,\nu,\mu) \in \gamma\,k$ such that $\rho'=\rho$, $\nu'=\nu$ and $\mu' =\mu\oplus\{(z+\sem{t}\,(\rho,\nu,\mu))\mapsto a\}$ and $fresh(a, (\rho, \nu, \mu))$.
	From Proposition~\ref{p:ass-unknown} it follows that $(\rho',\nu',\mu') \models \restr{k}{\neg s_1}$.
	From $fresh(a,( \rho,\nu,\mu))$ it follows that all addresses stored in any term which is not $s_1$ do not have the same address block as $a$, therefore $(\rho',\nu',\mu') \models \bigwedge_{t\in \T,s_1\notin Q'}(bl(s_1) \neq bl(\tau\,t))$.
	It directly follows that $(\rho',\nu',\mu') \in \gamma\,(\sem{s_1\,{:=}\,\malloc}^\sharp\,k)$
\end{proof}
