\section{Extension to a Superset of Terms}

When new terms appear during the analysis, the partition $P$ and the automaton $M_P$ need to be updated to express properties about the new terms.
It is not sufficient to simply add a fresh equivalence class for the new terms, as they sometimes need to be added to existing equivalence classes.

\begin{example}
    Let $\Psi \equiv A = B$ and $\T = \{A, B, *A\}$.
    The equivalence classes of $P[\Psi]$ are $\{A, B\}$ and $\{*A\}$.
    If we want to add the term $*B$ to $\T$, we need to add it to the same equivalence class as $*A$, as the two terms are equivalent according to the rules defined in \cref{chapter:qcc}.
    \todo{The same example is already somewhere else.}
\end{example}

Let $P = (\T, \tau, \omega)$ be a partition with the corresponding QFA $M_P = (S, \otau, \eta, \delta)$ and let $\T'$ be a superset of $\T$ and closed under subterms.
We define the operation $\ext{k}{\T'}$ that extends the kernel $k = \angl{P, M_P, B, D}$ to the set of terms $\T'$, without altering the semantics of the representation.
Only $P$ and $M_P$ need to be modified, while $B$ and $D$ remain unchanged.

The extension is built inductively based on the structure of the terms in $\T'$.
We first describe how to add a new atom and then consider terms of the form $*(z+t)$, where $t$ is already a part of $\T$.
This is repeated for all terms in $\T'$.

In the case that $\T' = \T \cup \{a\}$ for an atom $a \notin \T$, we simply add a new equivalence class with representative $a$ to $P$, by including a new state $s$ to the states $S$ of $M_P$ and defining $\tau\,a = a$, $\omega\,a=0$, $\eta(a) = (0,s)$, $\otau\,s = a$.

If $\T' = \T \cup \{*(z+t)\}$ for a term $t \in \T$, but $*(z + t) \notin \T$, we add a new equivalence class $\{*(z+t)\}$ to $P$ and then determine if it needs to be merged with an existing equivalence class.
We define $\tau\,(*(z+t)) = *(z+t)$ and $\omega\,(*(z+t))=0$.
Let the resulting partition be $P'$ and let $s \in S$ be the state of the subterm $t$, i.e., for which it holds that $\otau\,s = \tau\,t$.
If $\delta(z + \omega\,t,s)$ is not defined, then we add a new state $s'$ to $S$ and define $\delta(z + \omega\,t, s) = (0, s')$ and $\otau\,s' = *(z+t)$.
Otherwise, let $\delta(z + \omega\,t, s) = (z', s')$.
Then we perform the operation $\closure{*(z + t)}{z'}{\otau\,s'}{P'}$.
In this case, we do not need to add a new state to the automaton,
as the new term is added to the existing equivalence class of $s'$.
