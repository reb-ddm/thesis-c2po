\section{Interprocedural Analysis}

At the beginning (enter) we add for each parameter $p$ of the function a
shadow variable $p'$ that represents during the whole analysis of the function
the initial value of $p$ at the beginning of the function.
This way, if we have an example code:

\begin{minted}{C}
void f(int** p, int** q) {
    *p = *q;
    p = NULL;
    q = NULL;
}
int main(){
    int** a;
    int** b;
    f(a,b);
}
\end{minted}

call $f(a)$, we can assign $a$ to $a'$ at the

\subsection{Concrete Semantics}
Formally, the $\enter$ function is defined as:
\todo[inline]{TODO}
\[
    \enter \sem{(\rho, \nu, \mu)} = (\rho|_{local}+assign,\nu|_{local}+assign, \mu)
\]

At the end of the function, the $\combine$ function is called when traversing the edge from the returning point of the function to the caller.
It takes as an argument the concrete state of the caller before having invoked the function,
and the final state of the function.
It keeps the information about the local variables of the caller and forgets about the local variables of the function.
The values of the global variables and the content of the memory are equal to the state after the function call.

\todo[inline]{TODO example}

\begin{tikzpicture}[
        node distance=1cm and 4cm,
        every node/.style={font=\small},
        call/.style={->, >=Stealth, thick, dashed, blue},
        combine/.style={->, >=Stealth, thick, dashed, red},
        block/.style={draw, rectangle, rounded corners, align=center}
    ]

    % Nodes for main function
    \node[block] (start) {start};
    \node[below=of start] (usw1) {...};
    \node[block, below=of usw1] (beforecall) {};
    \node[block, below=of beforecall] (aftercall) {};
    \node[below=of aftercall] (usw2) {...};
    \node[block, below=of usw2] (end) {end};

    % Nodes for function f
    \node[block, right=of usw1] (fstart) {f start};
    \node[below=of fstart] (usw3) {...};
    \node[block, below=of usw3] (fend) {f return};

    % Edges for main function
    \draw[->] (start) -- node[right] {...} (usw1);
    \draw[->] (usw1) -- node[right] {...} (beforecall);
    \draw[->] (beforecall) -- node[right] {f(...)} (aftercall);
    \draw[->] (aftercall) -- node[right] {y = 10} (usw2);
    \draw[->] (usw2) -- node[right] {y = 10} (end);

    % Edges for function f
    \draw[->] (fstart) -- node[right] {a = x} (usw3);
    \draw[->] (usw3) -- node[right] {return} (fend);

    % Enter and combine edges
    \draw[call] (beforecall.east) to node[above] {enter} (fstart.west);
    \draw[combine] (fend.west) to node[below] {combine} (aftercall.east);

\end{tikzpicture}



\subsection{Abstract Semantics}

When entering a function we need to forget local variables of the caller, but keep the information about
global variables and the content of the memory.
However, if we simply perform the analysis of a function and then return to the caller,
it is necessary to restore the propositions between the return value or global vars and
the parameters given to the function in the beginning.
While the parameters may have been overwritten during the function execution,
when we return to the caller, only the values that were handed over as parameters to the
function are still unchanged.
Therefore it is necessary to keep track of the initial value of the parameters in a function.
This will be illustrated in the following example.

\begin{minted}[linenos=true, escapeinside=||]{C}
    int* f(int* a, int* b) { |\label{testlabel3}|
        int *g = a; |\label{testlabel}|
        a = NULL; |\label{testlabel4}|
        return g;
    }
    int main(){
        int* c;
        int* d;
        int* e = f(c,d);
        assert(e == c);
    }
\end{minted}

\begin{example}
If we simply remove all local variables of the caller from the abtract state, then
after the abtract enter function ($\enter^\sharp$), the abstact state is an empty conjunction
at the beginning of the function in line~\ref{testlabel}.
After line~\ref{testlabel3}, the abstract state is $g = a$.
We lose all information after line~\ref{testlabel4}.
When returning to \textsf{main}, we cannot retrieve any information from the function,
and we lose the information that $e = c$.

To solve this problem, we add additional shadow variables for each parameter of the function,
that represent the initial values of the parameter at the beginning of the function.
This way, in line~\ref{testlabel3}, we add the variables $a'$ and $b'$ and the equalities $a = a'$ and $b = b'$
to the abstract state.
After line~\ref{testlabel}, we add the equality $a = g$.
After line~\ref{testlabel4}, we lose information about $a$, but we still know that $g = a'$.
When returning to \textsf{main},
we know that the parameters with ehich we called $f$ were $c$ and $d$,
so we can add the equalities $c = a'$ and $d = b'$.
Then we still need to remove the local variables of $f$ from the abstract state,
as well as the shadow variables $a'$ and $b'$.
But the  equality $e = c$ is still preserved.
\end{example}

The abstract enter function $\enter^\sharp$ performs the following steps:
\begin{enumerate}
    \item Remove all local variables of the caller from the abstract state.
    \item For each parameter $p$ of the function, add a shadow variable $p'$ to the abstract state,
as well as the equality $p = p'$.
\end{enumerate}

The abstract combine function $\combine^\sharp$ performs the following steps:
\begin{enumerate}
    \item Remove all tainted \todo{explain tainted} variables from the caller state
    \item Meet the caller state with the state of the function.
    \item For each parameter $p$, if the function was called with the
    expression $e$ as a value for the parameter $p$, and if this expression is expressible in 2-pointer logic,
     add the equality $p' = e$ to the resulting state.
    \item If the return value is assigned to the variable $x$ in the caller,
    and the function returns an expression $e'$ and this expression is expressible in 2-pointer logic,
    then add the assignment $\sem{x = e'}^\sharp$ is performed.
    \item Remove all local variables of the function from the abstract state,
    as well as the shadow variables.
\end{enumerate}
TODO soundness?
