\section{Goblint}

The \goblint\ tool is an abstract interpretation framework for C programs.
It allows to define custom abstract domains by specifying an abstract domain and all abstract edge effects.
Then, \goblint\ is able to analyze C programs by transforming them in a CFG and
generating the system of inequalities that come from each edge, and solving
this system of consstraint using appropriate fixpoint algorithms.
There are already several abstract domains implemented in \goblint, such as
relational and non-relational value analyses, concurrency and mutex analyses,
out-of-bounds access analysis, etc.

The different analyses can communicate with each other through \emph{queries}.
This is useful to increase the precision of each analysis by combining the results of different analyses.
The \cpo\ analysis uses information from two already existing analyses of \goblint, the pointer analysis and the tainted variables analysis.

%\subsection{Pointer Analysis}
The pointer analysis in \goblint\ is a non-relational value analysis that tracks for each expression the set of possible addresses it can take.
The query \textsf{MayPointTo} is used to ask for the set of possible addresses an expression can point to.
It is a may-analysis, meaning that the result is an overapproximation of the possible addresses and it is impossible for the expression to point to an address that is not in the result.
In opposition, the \cpo\ analysis is a must-analysis, meaning that each of the propositions
are definitely true for the current state, but there might be even more true propsitions that are not found by our analysis.
Even though the pointer analysis is not relational, it complements the \cpo\ analysis
by determining additional disequalities between terms.
These disequalities follow from the fact that if two may-point-to address sets of two terms
have an empty intersection, then the terms are definitely not equal.

%\subsection{Tainted Variables Analysis}

An additional analysis that is used in the \cpo\ analysis is the tainted variables analysis.
During the execution of a function, each variable or address that is overwritten, s called a \emph{tainted} variable.
The query \textsf{MayBeTainted} returns an over-approximation of the set of variables that are tainted in the function call.
This information is used to remove the tainted variables from the abstract state of the caller when returning from the function.
Without this analysis, all the global variables and terms that represent addresses in the memory would be removed from the abstract state of the caller, resulting in a very imprecise analysis.
