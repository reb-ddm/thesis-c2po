\section{Restriction to a Subset of Terms}

During an assignment, it is necessary to forget information about a specific set of terms.
For instance, when assigning a value to a term, all terms in the domain that this assignment might modify---those that may alias with this term---must be removed.
This section explains how to restrict a kernel $k = \angl{P, M, B, D}$
to keep only the propositions containing terms from a specified set $\Set$ of allowed terms.

Let $P = (\T, \tau, \omega)$ be a partition and its corresponding automaton $M = (S, \otau, \eta, \delta)$ and $\Set \subseteq \L(M)$ a subterm-closed set of terms.
This section describes how to compute the restricted automaton $\restr{M}{\Set} = (S', \otau', \eta', \delta')$ such that $\L(\restr{M}{\Set})$ only includes terms from $\Set$.
For some states $s \in S$, the representative $\otau\,s$ might not be in $\Set$.
If this is the case, we must find a new representative $\otau'\,s$ from the set $\L_{M}(s) \cap \Set$.
If this set is empty, the state $s$ must be removed from the automaton.

From $\restr{M}{\Set}$, we will see how to derive the restricted partition $\restr{P}{\Set} = (\T', \tau', \omega')$, where $\T'$ contains the terms in $\T \cap \Set$ and additionally all terms that appear in the codomain of the function $\otau'$.
We do not define $\T' = \Set$, as $\Set$ may be an infinite set.
However, defining $\T' = \T \cap \Set$ would not be sufficient, as there may be some states $s \in S'$ for which
$\T \cap \Set$ does not contain any term of $\L_{M}(s)$, i.e.,
$\L_{M}(s) \cap \T \cap \Set = \emptyset$.
If such a state $s$ exists, it would mean that $s$ does not have a corresponding equivalence class in $\restr{P}{\Set}$.
To avoid this, we add for each $s \in S'$ the term $\otau'\,s$ to $\T'$, ensuring that there is at least one element in the equivalence class of $s$.

\begin{example}
    Consider the conjunction $\Psi \equiv (\&x = \&z) \land (*x = \&y)$.
    We choose $\T = \{\&x, \&y, \&z, x, *x\}$. The corresponding automaton $M$ can be visualized as follows:
    \todo{should I make this Figure 4.3 and reference it, or is it ok like this?}
    \begin{center}
        \begin{tikzpicture}[shorten >=1pt, node distance=3cm, on grid, auto,
                state/.style={rectangle, rounded corners, draw, inner sep=5pt, align=center}]
            \node[state] (q_0) {$\&x, \&z$};
            \node[below=0.5cm of q_0] {$s_1$};
            \node[state] (q_1) [right=of q_0] {$x$};
            \node[below=0.5cm of q_1] {$s_2$};
            \node[state] (q_2) [right=of q_1] {$*x, \&y$};
            \node[below=0.5cm of q_2] {$s_3$}; % Label for the third state

            \path[->]
            (q_0) edge node [above] {$0,0$} (q_1)
            (q_1) edge node [above] {$0,0$} (q_2);
        \end{tikzpicture}
    \end{center}
    Let $\Set$ be the (infinite) set of terms that do not contain the variable $x$.
    We want to compute $\restr{M}{\Set}$.
    It holds that $\T \cap \Set = \{\&y, \&z\}$.
    The set $\T \cap \Set$ does
    not contain any terms from $\L_{M}(s_2) = \{x,z\}$.
    However, $\L_{M}(s_2)$ contains $z$, which is in $\Set$, and is therefore allowed to appear in $\restr{M}{\Set}$.
    Therefore, we can choose $z$ as the new representative $\tau'\,s_2$.
    The new set of terms after the restriction is $\T' = (\T \cap \Set) \cup \{z\} = \{\&y, \&z, z\}$.
    The resulting automaton can be visualized as follows:

    \begin{center}
        \begin{tikzpicture}[shorten >=1pt, node distance=3cm, on grid, auto,
                state/.style={rectangle, rounded corners, draw, inner sep=5pt, align=center}]
            \node[state] (q_0) {$\&z$};
            \node[below=0.6cm of q_0] {$s_1$};
            \node[state] (q_1) [right=of q_0] {$z$};
            \node[below=0.6cm of q_1] {$s_2$};
            \node[state] (q_2) [right=of q_1] {$\&y$};
            \node[below=0.6cm of q_2] {$s_3$};

            \path[->]
            (q_0) edge node [above] {$0,0$} (q_1)
            (q_1) edge node [above] {$0,0$} (q_2);
        \end{tikzpicture}
    \end{center}
    It represents the formula $*z = \&y$.
\end{example}

The restriction $\restr{M}{\Set} = (S', \otau', \eta', \delta')$ is computed using a breadth-first search (BFS) on the automaton $M$ to determine which paths remain reachable when keeping only the terms in $\Set$.
The algorithm proceeds as follows, with $\V$ representing the set of visited states and $Q$ being a queue of states to be processed:

\begin{enumerate}
    \item Start with the initial states $\eta(a) = (z, s)$ for each atom $a \in \T \cap \Set$:
          \begin{itemize}
              \item If $s$ has not been visited, i.e., $s \notin \V$, set $a$ as the representative for $s$:
                    $\otau'(s) = a$ and $\eta'(a) = (0, s)$.
                    Add $s$ to $\V$ and to $Q$.
              \item If $s$ has already been visited, i.e., $s \in \V$, there exists a $t$ such that $\otau'(s) = t$.
                    Let $M[t] = (z', s)$.
                    Define $\eta'(a) = (z - z', s)$.
          \end{itemize}

          \item\label{item:remove-from-Q} Remove a state $s_1$ from the queue $Q$.
          Consider all outgoing transitions $\delta(z_1, s_1) = (z_2, s_2)$.
          \begin{itemize}
              \item If a term $*(z+t) \in \L(M) \cap \Set$ exists such that $M[t] = (z_1', s_1)$ and $z = z_1 - z_1'$,
                    then the transition remains valid.
                    \begin{itemize}
                        \item If $s_2 \in \V$, then $\otau'(s_2)$ is already defined.
                              Otherwise, set $\otau'(s_2) = *(z+t)$ and add $s_2$ to $\V$ and to $Q$.
                        \item Now let $M[\otau'(s_1)] = (z_3, s_1)$ and $M[\otau'(s_2)] = (z_4, s_2)$.
                              Define the transition $\delta'(z_1 - z_3, s_1) = (z_2 - z_4, s_2)$.
                    \end{itemize}
              \item If no such term exists, the transition does not appear in $\restr{M}{\Set}$, and $s_2$ is not added to the set of visited states.
          \end{itemize}
    \item If $Q$ is not empty, return to step \labelcref{item:remove-from-Q}.\ and continue with the next state in $Q$.
\end{enumerate}

Essentially, the algorithm retains the transitions between the states that remain reachable using only terms from $\Set$ and adjusts the weights of these transitions to the new representatives.
The set $S'$ equals the set of visited states $\V$.

The new partition $\restr{P}{\Set} = (\T', \tau', \omega')$ is defined as follows:
\begin{itemize}
    \item $\T' = (\T \cap \Set) \cup \{t \mid \exists s \in S' . \otau'\,s = t\}$,
    \item For each $t \in \T'$, if $\restr{M}{\Set}[t] = (z, s)$, then $\tau'\,t = \otau'\,s$ and $\omega'\,t = z$.
\end{itemize}

\begin{theorem}\label{restriction}
    \todo[inline]{Not only prove this, but also explain it}
    Let $M = (S, \otau, \eta, \delta)$ be a QFA and let $\restr{M}{\Set} = (S', \otau', \eta', \delta')$ be the automaton that is obtained by restricting $M$ as described above.
    Then,
    \begin{enumerate}
        \item\label{item:lemma-restriction} for each term $t \in \L(\restr{M}{\Set})$, it holds that $M[t] = (z, s)$ and $M[\otau'\,s] = (z', s')$ iff $\restr{M}{\Set}[t] = (z - z', s)$ and $s = s'$,
        \item $\L(\restr{M}{\Set}) = \L(M) \cap \Set$ and
              \item\label{item:correctness-restriction} for each term $t_1, t_2 \in \L(\restr{M}{\Set})$, it holds that $M[t_1] = z + M[t_2]$ iff $\restr{M}{\Set}[t_1] = z + \restr{M}{\Set}[t_2]$.
    \end{enumerate}
\end{theorem}
\begin{proof}
    The first point can be proven by induction over the structure of the term $t$.
    \todo[inline]{Proof of point 2}
    Point~\labelcref{item:correctness-restriction} follows from point~\labelcref{item:lemma-restriction}.
\end{proof}

\subsubsection{(Block-)Disequalities}
Given a kernel $k = \angl{P, M, B, D}$, we can restrict it to a set of terms $\Set$ by setting $\restr{k}{\Set} = \angl{\restr{P}{\Set}, \restr{M}{\Set}, \restr{B}{\Set}, \restr{D}{\Set}}$.
$\restr{B}{\Set}$ and $\restr{D}{\Set}$ represent the block disequalities and quantitative disequalities that are still valid when considering only the terms in $\T'$.
We keep only the propositions between representatives of the equivalence classes that
still have a corresponding state in the automaton $\restr{M}{\Set}$, but we need to update the terms in the disequalities to the new representatives.
Disequalities containing terms that are not in $\T'$ are removed.

For $k = \bot$, the restriction $\restr{k}{\Set}$ is $\bot$.
