\section{Restriction to a subset of terms}

For the definition of the assignment operation, it is necessary to be able to forget information about specific sets of terms.
For example, when we assign a value to a term, we need to remove all terms from the domain that may be modified by this assignment, i.e., all the terms that may alias with this term.
In the following, we will describe how to restrict the automaton and the disequalities to only keep the propositions that contain terms from a specific set.

Given the set $\restr{\T}{\neg t}$, we can find the restricted automaton $\restr{M}{\neg t} = (S', \eta', \delta')$ by conducting a breadth-first search on the automaton $M$ and computing which paths of the automaton are still reachable by using only the terms in $\restr{\T}{\neg t}$.
In order to correctly update the weights in the automaton and later find representative terms of the equivalence classes that derive from the automaton, we will need to choose a representative term for each state of the automaton, which we will denote as the function $\repr : S' \rightarrow \Z \times \restr{\T}{\neg t}$.

We start with the initial states of $M$:
for each atom $a \in \restr{\T}{\neg t}$ where $\eta(a) = (z, s)$, if we haven't already visited $s$, then we need to choose a new representative for $s$. We simply define $\repr s = (z, a)$ and $\eta'(a) = (0, s)$.
If we have already visited $s$, then let $\repr s = (z', t)$. We define $\eta'(a) = (z - z', s)$.

Then for each visited state $s_1$, we consider all outgoing transitions: for each $\delta(z_1, s_1) = (z_2, s_2)$,
we define $\repr s_2$ if we haven't visited $s_2$ yet.
For this, we find a term $*(z+t)$ such that $M[t] = (z_1', s_1)$ and $z = z_1 - z_1'$ for some $z_1' \in \Z$.
If we manage to find such a term, we define $\repr s_2 = *(z+t)$.
Otherwise, this transition is not valid in $\restr{M}{\neg t}$ anymore, and we don't add $s_2$ to the set of visited states.
Now let $\repr s_1 = (z_1', t_1)$ and $\repr s_2 = (z_2', t_2)$.
We can define the corresponding transition $\delta'(z_1 - z_1', s_1) = (z_2 - z_2', s_2)$.
We add $s_2$ to the set of reachable states, and we continue the search for all outgoing transitions for the remaining visited states.

Essentially, we keep the transitions between the states that are still reachable after removing the terms, and we adjust the weights of the transitions to the new representatives.

The new partition $\restr{P}{\neg t} = (\Pi', \tau', \omega')$ is defined as $\Pi' = \{\L(s) \mid s \in S, s \text{ is reachable in $\restr{M}{\neg t}$}\}$,
$\tau' (\L(s)) = t$ if $\repr s = (z, t)$, and
for each $t \in \L(\restr{M}{\neg t})$ we define $\omega'(t) = z'$ if $\restr{M}{\neg t}[t] = (z', t')$.

This definition of restriction can be applied in the same way by using any subterm-closed set $\T' \subseteq \L(M)$ of terms and restricting the automaton $M$ to the automaton $\restr{M}{\T'}$ that only remembers the equalities that only contain terms from $\T'$.

\begin{lemma}\label{restriction}
    Let $M = (S, \eta, \delta)$ be a QFA and let $\restr{M}{\T'} = (S', \eta', \delta')$ and $\repr : S' \rightarrow \Z \times \T'$ be the automaton and the function that are obtained by restricting $M$ as described above.
    Then for each term $t_1 \in \T'$, it holds that $M[t_1] = (z, s)$ iff $\restr{M}{\T'}[t_1] = (z - \repr s, s)$.
\end{lemma}
\begin{proof}
    This lemma can be proven by induction over the structure of the term $t_1$.
\end{proof}

\begin{proposition}
    Let $M = (S, \eta, \delta)$ be a QFA and let $\restr{M}{\T'} = (S', \eta', \delta')$ be the automaton that is obtained by restricting $M$ as described above.
    Then for each term $t_1, t_2 \in \T'$, it holds that $M[t_1] = z + M[t_2]$ iff $\restr{M}{\T'}[t_1] = z + \restr{M}{\T'}[t_2]$.
\end{proposition}
\begin{proof}
    This proposition follows from Lemma~\ref{restriction}.
\end{proof}

We can also find the set of disequalities and the block disequalities that are still valid when considering only the terms $\T'$.
We keep only the propositions between representatives of the equivalence classes that are still reachable in the automaton $\restr{M}{\T'}$, but we need to update the representatives to the new representatives.
The remaining disequalities are removed.

Given a conjunction $\Psi$, we denote the restriction of the automaton and disequalities derived from $\Psi$ to the set of terms $\T'$ by $\restr{\Psi}{\T'}$.
Correspondingly, $\restr{\Psi}{\neg t}$ is the restriction of $\Psi$ to the terms of $\restr{\T}{\neg t}$, where $\T = \L(M)$ and $M$ is the automaton that derives from $\Psi$.
