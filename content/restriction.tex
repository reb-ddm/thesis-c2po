\section{Restriction to a subset of terms}

For the definition of the assignment operation, it is necessary to be able to forget information about specific sets of terms.
For example, when we assign a value to a term, we need to remove all terms from the domain that may be modified by this assignment, i.e., all the terms that may alias with this term.
In the following, we will describe how to restrict the automaton and the disequalities to forget all propositions that contain terms from a specific set.

As a first step, we define the restriction of a partition and its corresponding automaton to forget information about the terms of a set $\Set$.
The complement of this set need to be closed under subterms, such that the
resulting partition will have a set that is still closed under subterms. \todo{Is this right?}
Let $P = (\T, \tau, \omega)$ be a partition and $M_P = (S, \otau, \eta, \delta)$ its corresponding QFA.\@
Given a set $\Set$, we can find restricted
automaton $\restr{M_P}{-\Set} = (S', \otau', \eta', \delta')$ such that $\L(\restr{M_P}{-\Set})$ does not contain terms from
$\Set$, but still contains the equalities between all other terms of $\L(M_P) \backslash \Set$.
From this we can derive the restricted partition $\restr{P}{-\Set} = (\T', \tau', \omega')$, where $\T'$ contains
the terms in $\T \backslash \Set$ and additionally all terms that appear in the codomain of the function $\otau'$.
This is necessary in order to make sure that each state in $M_P$ corresponds to an equivalence class in $P$,
even when $\T \backslash \Set$ does not contain any elements of the equivalence class for some of the resulting states.

\begin{example}
\todo{see implementation/saturating for variable removal -> not a good example}
\end{example}

The restriction $\restr{M_P}{-\Set}$ is computed by conducting a breadth-first search on the automaton $M_P$
and computing which paths of the automaton are still reachable after forgetting all the terms in $\Set$.
For states $s \in S$, it may be that the representative $\otau\,s$ is in $\Set$.
In this case, we need to find a new representative for $s$, which is one of the elements in the set
$\L(M_P) \backslash \Set$. On the other hand, if this set is empty,
we need to remove this state from the automaton.

We start with the initial states of $M_P$:
for each atom $a \in \T \backslash \Set$ where $\eta(a) = (z, s)$, if we haven't already visited $s$, then we need to choose a new representative for $s$. We simply define $\otau'\,s = a$ and $\eta'(a) = (0, s)$.
We also add $s$ to $S'$.
If we have already visited $s$, then let $\otau'\,s = t$. We define $\eta'(a) = (\omega\,a - \omega\,t, s)$.

\todo{do this more schematically/pseudocodemäßig}
Then for each visited state $s_1$, we consider all outgoing transitions: for each $\delta(z_1, s_1) = (z_2, s_2)$,
we define $\otau'\,s_2$ if we haven't visited $s_2$ yet.
\todo{are you sure about that? check the code}
For this, we find a term $*(z+t)\in \L(M)\backslash \Set$ such that $M_P[t] = (z_1', s_1)$ and $z = z_1 - z_1'$.
If we manage to find such a term, we define $\otau\,s_2 = *(z+t)$.
Otherwise, this transition is not valid in $\restr{M_P}{-\Set}$ anymore, and we don't add $s_2$ to the set of visited states.
Now let $\otau'\,s_1 = (z_1', t_1)$ and $\otau\,s_2 = (z_2', t_2)$.
We define the corresponding transition $\delta'(z_1 - z_1', s_1) = (z_2 - z_2', s_2)$.
We add $s_2$ to $S'$ and to the set of visited states, and we continue the search for all outgoing transitions for the remaining visited states.

Essentially, we keep the transitions between the states that are still reachable after removing the terms, and we adjust the weights of the transitions to the new representatives.

The new partition $\restr{P}{-\Set} = (\T', \tau', \omega')$ is defined as:
\begin{itemize}
    \item $\T' = (\T \backslash \Set) \cup \{t \mid \exists s \in S' . \otau'\,s = t\}$,
    \item $\tau'\,t = \otau'\,(s)$ and $\omega'\,t = z$ if $\restr{M_P}{-\Set}[t] = (z,s)$ and
\end{itemize}

\todo{adapt the proofs}
\begin{lemma}\label{restriction}
    Let $M = (S, \otau, \eta, \delta)$ be a QFA and let $\restr{M_P}{-\Set} = (S', \otau', \eta', \delta')$ be the automaton that is obtained by restricting $M$ as described above.
    Then for each term $t_1 \in \T'$, it holds that $M_P[t_1] = (z, s)$ iff $\restr{M_P}{-\Set}[t_1] = (z - \repr s, s)$.
\end{lemma}
\begin{proof}
    This lemma can be proven by induction over the structure of the term $t_1$.
\end{proof}

\begin{proposition}
    Let $M = (S, \eta, \delta)$ be a QFA and let $\restr{M}{\T'} = (S', \eta', \delta')$ be the automaton that is obtained by restricting $M$ as described above.
    Then for each term $t_1, t_2 \in \T'$, it holds that $M_P[t_1] = z + M_P[t_2]$ iff $\restr{M}{\T'}[t_1] = z + \restr{M}{\T'}[t_2]$.
\end{proposition}
\begin{proof}
    This proposition follows from Lemma~\ref{restriction}.
\end{proof}

We can also find the set of disequalities and the block disequalities that are still valid when considering only the terms $\T'$.
We keep only the propositions between representatives of the equivalence classes that
still have a corresponding state in the automaton $\restr{M_P}{-\Set}$, but we need to update the representatives to the new representatives.
The remaining disequalities are removed.

Given a conjunction $\Psi$, we denote the restriction of the corresponding kernel $k$ to forget propostions about terms in $\Set$ by $\restr{\Psi}{-\Set}$.
