\section{Restriction to a subset of terms}

During an assignment, it is necessary to forget information about a specific set of terms.
For example, when we assign a value to a term, we need to remove all terms from the domain that may be modified by this assignment, i.e., all the terms that may alias with this term.
In the following, we will describe how to restrict a kernel $k = \angl{P, M_P, B, D}$
to forget all propositions that contain terms from a specific set.

As a first step, we define the restriction of a partition $P = (\T, \tau, \omega)$ and its corresponding automaton $M_P = (S, \otau, \eta, \delta)$ to forget information about the terms of a set $\Set$.
The complement of $\Set$ needs to be closed under subterms, such that the
resulting partition will contain terms from a set that is still closed under subterms. \todo{Is this right?}
Given a set $\Set$, we can find restricted
automaton $\restr{M_P}{-\Set} = (S', \otau', \eta', \delta')$ such that $\L(\restr{M_P}{-\Set})$ does not contain terms from
$\Set$, but still contains the equalities between all other terms of $\L(M_P) \backslash \Set$.
From this we can derive the restricted partition $\restr{P}{-\Set} = (\T', \tau', \omega')$, where $\T'$ contains
the terms in $\T \backslash \Set$ and additionally all terms that appear in the codomain of the function $\otau'$.
This is necessary in order to make sure that for each state $s \in S'$ it is possible to find a representative term in $\T'$,
for states $s$ for which $\L_{\restr{M_P}{-\Set}}(s) \neq \emptyset$ where
even when $\T \backslash \Set$ does not contain any term in $\L_{\restr{M_P}{-\Set}}(s)$.

\begin{example}
    Consider the conjunction $\Psi \equiv \&x = \&z \land *x = \&y$.
    We choose $\T = \{\&x, \&y, \&z, x, *x\}$. The corresponding automaton $M_P$ can be visualized as follows:

    \begin{center}
        \begin{tikzpicture}[shorten >=1pt, node distance=3cm, on grid, auto,
                state/.style={rectangle, rounded corners, draw, inner sep=3pt, align=center}]
            \node[state] (q_0) {$\&x, \&z$};
            \node[below=0.5cm of q_0] {$s_1$};
            \node[state] (q_1) [right=of q_0] {$x$};
            \node[below=0.5cm of q_1] {$s_2$};
            \node[state] (q_2) [right=of q_1] {$*x, \&y$};
            \node[below=0.5cm of q_2] {$s_3$}; % Label for the third state

            \path[->]
            (q_0) edge node {} (q_1)
            (q_1) edge node {} (q_2);
        \end{tikzpicture}
    \end{center}

    Let $\Set$ be the set of terms that contain the variable $x$.
    If we simply remove all terms from $\Set$ from the automaton, the state $s_2$ would
    not contain any terms from $\T \backslash \S$.
    However, the set $\L_{M_P}(s)$ contains not only $x$, but also $z$, which is not in $\Set$.
    Therefore, we can choose $z$ as the new representative for $s_2$.
    The new set of terms after the restriction is $\T' = (\T \backslash \Set) \cup \{z\} = \{\&y, \&z, z\}$.
    The resulting automaton can be visualized as follows:

    \begin{center}
        \begin{tikzpicture}[shorten >=1pt, node distance=3cm, on grid, auto,
                state/.style={rectangle, rounded corners, draw, inner sep=3pt, align=center}]
            \node[state] (q_0) {$\&z$};
            \node[below=0.5cm of q_0] {$s_1$};
            \node[state] (q_1) [right=of q_0] {$z$};
            \node[below=0.5cm of q_1] {$s_2$};
            \node[state] (q_2) [right=of q_1] {$\&y$};
            \node[below=0.5cm of q_2] {$s_3$};

            \path[->]
            (q_0) edge node {} (q_1)
            (q_1) edge node {} (q_2);
        \end{tikzpicture}
    \end{center}
    It represents the formula $*z = \&y$.
    \todo{add edge weights}
\end{example}

The restriction $\restr{M_P}{-\Set} = (S', \otau', \eta', \delta')$ is computed by conducting a breadth-first search on the automaton $M_P$
and computing which paths of the automaton are still reachable after forgetting all the terms in $\Set$.
For states $s \in S$, it may be that the representative $\otau\,s$ is in $\Set$.
In this case, we need to find a new representative for $s$, which is one of the elements in the set
$\L_{M_P}(s) \backslash \Set$. On the other hand, if this set is empty,
we need to remove $s$ from the automaton.

The algorithm is as follows, where $\V$ is the set of visited states and $Q$ is a queue of states that need to be considered:
\begin{enumerate}
    \item We start with the initial states of $M_P$ and we consider each atom $a \in \T \backslash \Set$ where $\eta(a) = (z, s)$:
          \begin{itemize}
              \item If we haven't already visited $s$, then we need to choose a new representative for $s$. We simply define $\otau'\,s = a$ and $\eta'(a) = (0, s)$.
                    We also add $s$ to $\V$ and to $Q$.
              \item Otherwise, if we have already visited $s$, then there is a $t$ such that $\otau'\,s = t$.
                    Let $M[t] = (z', s)$.
                    We define $\eta'(a) = (z - z', s)$.
          \end{itemize}
          \item\label{item:next-from-queue} Then we remove a state $s_1$ from the queue $Q$.
          We consider all outgoing transitions $\delta(z_1, s_1) = (z_2, s_2)$.
    \item If we manage to find a term $*(z+t)\in \L(M)\backslash \Set$ such that $M_P[t] = (z_1', s_1)$ and $z = z_1 - z_1'$, then the transition is still valid.
          \begin{itemize}
              \item  If we haven't visited $s_2$ yet, we define $\otau'\,s_2 = *(z+t)$ and we add $s_2$ to the set of visited states $\V$ and to $Q$.
              \item Otherwise, if we have already visited $s_2$, then $\otau'\,s_2$ is already defined.
              \item Now let $M_P[\otau'\,s_1] = (z_3, s_1)$ and $M_P[\otau'\,s_2] = (z_4, s_2)$.
                    We define the corresponding transition $\delta'(z_1 - z_3, s_1) = (z_2 - z_4, s_2)$.
          \end{itemize}
    \item Otherwise, this transition is not valid in $\restr{M_P}{-\Set}$ anymore, and we don't add $s_2$ to the set of visited states.
    \item If $Q$ is empty, we are done. Otherwise, we go back to step~\labelcref{item:next-from-queue}.
\end{enumerate}

Essentially, we keep the transitions between the states that are still reachable after removing the terms, and we adjust the weights of the transitions to the new representatives.
The set $S'$ is equal to the set of visited states $\V$.

The new partition $\restr{P}{-\Set} = (\T', \tau', \omega')$ is defined as:
\begin{itemize}
    \item $\T' = (\T \backslash \Set) \cup \{t \mid \exists s \in S' . \otau'\,s = t\}$,
    \item For each $t \in \T'$, if $\restr{M_P}{-\Set}[t] = (z, s)$, then $\tau'\,t = \otau'\,s$ and $\omega'\,t = z$.
\end{itemize}

\begin{theorem}\label{restriction}
    Let $M_P = (S, \otau, \eta, \delta)$ be a QFA and let $\restr{M_P}{-\Set} = (S', \otau', \eta', \delta')$ be the automaton that is obtained by restricting $M_P$ as described above.
    Then,
    \begin{enumerate}
        \item\label{item:lemma-restriction} for each term $t \in \L(\restr{M_P}{-\Set})$, it holds that $M_P[t] = (z, s)$ and $M[\otau'\,s] = (z', s')$ iff $\restr{M_P}{-\Set}[t] = (z - z', s)$ and $s = s'$,
        \item $\L(\restr{M_P}{-\Set}) = \L(M_P) \backslash \Set$ and
              \item\label{item:correctness-restriction} for each term $t_1, t_2 \in \L(\restr{M_P}{-\Set})$, it holds that $M_P[t_1] = z + M_P[t_2]$ iff $\restr{M_P}{-\Set}[t_1] = z + \restr{M_P}{-\Set}[t_2]$.
    \end{enumerate}
\end{theorem}
\begin{proof}
    The first point can be proven by induction over the structure of the term $t$.
    \todo[inline]{Proof of point 2}
    Point~\labelcref{item:correctness-restriction} follows from point~\labelcref{item:lemma-restriction}.
\end{proof}
%--------------
\subsubsection{(Block-)Disequalities}
We can also find the set of disequalities and the block disequalities that are still valid when considering only the terms $\T'$.
We keep only the propositions between representatives of the equivalence classes that
still have a corresponding state in the automaton $\restr{M_P}{-\Set}$, but we need to update the representatives to the new representatives.
The remaining disequalities are removed.

Given a conjunction $\Psi$, we denote the restriction of the corresponding kernel $k$ to forget propostions about terms in $\Set$ by $\restr{\Psi}{-\Set}$.
